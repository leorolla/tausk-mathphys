
% !TeX spellcheck = en_US
% !TEX encoding = UTF-8 Unicode

\documentclass[oneside,a4paper,11pt]{amsbook}

\usepackage{dsfont}
\usepackage[all]{xy}
\usepackage{amssymb}

\title[Notes on Mathematical Physics]{Notes on Mathematical Physics for Mathematicians}

\author{Daniel V. Tausk}

\date{May 27th, 2010}

\newcommand{\R}{\mathds R}
\newcommand{\C}{\mathds C}
\newcommand{\Z}{\mathds Z}
\newcommand{\dd}{\mathrm d}
\newcommand{\compl}{\mathrm c}
\newcommand{\Id}{\mathrm{Id}}
\newcommand{\I}{\mathrm I}
\newcommand{\anul}{\mathrm o}
\newcommand{\transp}{\mathrm t}
\newcommand{\microspace}{\mskip1mu}

\DeclareMathOperator{\Aut}{Aut}
\DeclareMathOperator{\Ker}{Ker}
\DeclareMathOperator{\Img}{Im}
\DeclareMathOperator{\Aff}{Aff}
\DeclareMathOperator{\GL}{GL}
\DeclareMathOperator{\Or}{O}
\DeclareMathOperator{\SO}{SO}
\DeclareMathOperator{\so}{\mathfrak{so}}
\DeclareMathOperator{\Gr}{gr}
\DeclareMathOperator{\Dom}{dom}
\DeclareMathOperator{\Span}{span}
\DeclareMathOperator{\sgn}{sgn}
\DeclareMathOperator{\Dim}{dim}
\DeclareMathOperator{\Diff}{Diff}

\theoremstyle{remark}\newtheorem{exercise}{Exercise}[chapter]
\swapnumbers
\theoremstyle{plain}\newtheorem{teo}{Theorem}[section]
\theoremstyle{plain}\newtheorem{lem}[teo]{Lemma}
\theoremstyle{plain}\newtheorem{prop}[teo]{Proposition}
\theoremstyle{plain}\newtheorem{cor}[teo]{Corollary}
\theoremstyle{definition}\newtheorem{defin}[teo]{Definition}
\theoremstyle{remark}\newtheorem{rem}[teo]{Remark}
\theoremstyle{definition}\newtheorem{notation}[teo]{Notation}
\theoremstyle{definition}\newtheorem{convention}[teo]{Convention}
\theoremstyle{definition}\newtheorem{example}[teo]{Example}

\numberwithin{section}{chapter}
\numberwithin{equation}{section}

\begin{document}

\frontmatter
\maketitle

\chapter*{License}
\noindent
\copyright\ 2010 Daniel V. Tausk.
\\
Licensed under CC~BY-SA 4.0.

%\addtocounter{tocdepth}{1}
\renewcommand{\contentsline}[3]{\csname novo#1\endcsname{#2}{#3}}
\newcommand{\novochapter}[2]{\bigskip\hbox to \hsize{\vbox{\advance\hsize by -1cm\baselineskip=12pt\parfillskip=0pt\leftskip=3cm\noindent\hskip -2cm #1\leaders\hbox{.}\hfil\hfil\par}$\,$#2\hfil}}
\newcommand{\novosection}[2]{\medskip\hbox to \hsize{\vbox{\advance\hsize by -1cm\baselineskip=12pt\parfillskip=0pt\leftskip=3.5cm\noindent\hskip -2cm #1\leaders\hbox{.}\hfil\hfil\par}$\,$#2\hfil}}
\newcommand{\novosubsection}[2]{\baselineskip=12pt}
%\newcommand{\novosubsection}[2]{\medskip\hbox to \hsize{\vbox{\advance\hsize by -1cm\baselineskip=12pt\parfillskip=0pt\leftskip=4cm\noindent\hskip -2cm #1\leaders\hbox{.}\hfil\hfil\par}$\,$#2\hfil}}
\tableofcontents

\mainmatter

\begin{chapter}{Classical Mechanics}

There are three basic ingredients for the formulation of a physical theory: spacetime, ontology and dynamics.
Spacetime is represented, within the theory, by a set endowed with some extra structure; the points of such
set are called {\em events}. For Classical Mechanics, the adequate type of spacetime is {\em Galilean spacetime},
which consists of a four dimensional real affine space endowed with global absolute time and Euclidean
metric structure over the simultaneity hyperplanes defined by the absolute time function. We will give the details
of the definition in Section~\ref{sec:Galileanspacetime}. To describe the {\em ontology\/} of the theory means
to say what are the objects whose existence is posited by the theory, i.e., the thing the theory is all about.
Such objects should be in some way connected to spacetime. For Classical Mechanics, the ontology consists of
{\em particles}, having {\em worldlines\/} which are one-dimensional submanifolds of spacetime. Finally, the
{\em dynamics\/} says how the ontology is allowed to behave. In the case of Classical Mechanics,
this means specifying when a given set of worldlines for the particles is admissible by the theory. This is done
by means of a second order ordinary differential equation. We present the ontology and
the dynamics of Classical Mechanics in Section~\ref{sec:ontologydynamics}.

\begin{section}{Galilean spacetime}
\label{sec:Galileanspacetime}

Recall that an {\em affine space\/} consists of a non empty set $E$, a vector space $V$ and a transitive free action
$V\times E\to E$ of the additive group $(V,+)$ on the set $E$. The elements of $E$ are called {\em points\/}
and the elements of $V$ are called {\em vectors}; the action of a vector $v\in V$ upon a point $e\in E$ is denoted
by $v+e\in E$ or, alternatively, by $e+v\in E$. The fact that the action is free and transitive means that given
points $e_1,e_2\in E$, there exists a unique vector $v\in V$ such that $e_2=v+e_1$; such vector $v$ is denoted by
$e_2-e_1$. We normally refer to the set $E$ as the affine space and to $V$ as the {\em underlying vector space\/} of $E$.
The {\em dimension\/} of $E$ and the {\em scalar field\/} of $E$ are, by definition, the dimension of $V$ and
the scalar field of $V$, respectively.

\begin{defin}
A {\em Galilean spacetime\/} consists of:
\begin{itemize}
\item a four-dimensional real affine space $E$;
\item a non zero linear functional $\mathfrak t:V\to\R$ on the underlying vector space $V$ of $E$;
\item a (positive definite) inner product $\langle\cdot,\cdot\rangle$ on the kernel $\Ker(\mathfrak t)$ of $\mathfrak t$.
\end{itemize}
We call $\mathfrak t$ the {\em time functional}.
\end{defin}

Given events $e_1,e_2\in E$, the {\em elapsed time\/} from $e_1$ to $e_2$ is defined as $\mathfrak t(e_2-e_1)$.
Two events $e_1,e_2\in E$ are called {\em simultaneous\/} if the elapsed time from $e_1$ to $e_2$ is zero,
i.e., if the vector $e_2-e_1$ is in the kernel of $\mathfrak t$. When
$e_1, e_2\in E$ are simultaneous, we can define the {\em distance\/} between $e_1$ and $e_2$ to be $\Vert e_1-e_2\Vert$,
where $\Vert\cdot\Vert$ is the norm on $\Ker(\mathfrak t)$ associated to the given inner product. The simultaneity
relation is an equivalence relation, which defines a partition of $E$ into equivalence classes. Such equivalence classes
are precisely the orbits of the action of $\Ker(\mathfrak t)$ on $E$. Those are three-dimensional real affine spaces with underlying vector space
$\Ker(\mathfrak t)$ (see Exercise~\ref{exe:affinesubspace}); we call them the {\em simultaneity hyperplanes}.

Notice that, unlike vector spaces in which the origin is a ``special'' point, affine spaces have no ``special'' points.
That is one of the reasons why spacetime is taken to be an affine space, rather than a vector space.
Also, notice that, given a Galilean
spacetime, one cannot define an object which represents space. Rather, one has a family of simultaneity hyperplanes
(one space for each instant of time). This happens because in Galilean spacetime it is meaningless to say
that two events $e_1$, $e_2$ on distinct simultaneity hyperplanes are ``at the same place
in space''. To say that two events happened at the same time is meaningful, but to say that they happened at the
same place is not!

\subsection{Units of measurement}
The definition of Galilean spacetime given above is not quite what it should be.
Notice that, according to our definitions, the elapsed time between two events $e_1, e_2\in E$ is a real number and this is
not quite right. The elapsed time between events should not be a number, but rather something that is expressed
using a unit of measurement of time, such as {\em seconds}. In order to obtain a more appropriate definition of
Galilean spacetime, one should fix a real one-dimensional vector space $S$ and demand that the time functional $\mathfrak t$ be a non zero
linear map $\mathfrak t:V\to S$ taking values in $S$. Thus, elapsed times would be elements of $S$, not real numbers!
The choice of a non zero element of $S$ amounts to the choice of both a
{\em time orientation\/} and of a {\em unit of measurement of time\/} and it allows one to identify $S$ with $\R$.
Notice that, by taking $\R$
as the counter-domain of $\mathfrak t$ in our definition of Galilean spacetime, we have automatically taken a time orientation to
be part of the structure of spacetime (i.e., we can say that an event $e_2$ is in the {\em future\/} of the event $e_1$ when
$\mathfrak t(e_2-e_1)>0$). One might argue that this is not appropriate\footnote{%
Orientation of time is not needed for specifying the dynamics of Classical Mechanics. One might argue that it would
be more appropriate to leave the orientation of time out of the structure of spacetime, and define it in terms of entropy.
We shall not discuss this issue here.}.

Just as the counter-domain of $\mathfrak t$ shouldn't be $\R$, also the counter-domain of the inner product on the kernel of $\mathfrak t$
shouldn't be $\R$. The outcome of an inner product should be expressed in units of {\em squared length}. So, just as in the
case of time, one should fix a real one-dimensional vector space $M$ whose elements are to be interpreted as
lengths; in this case,
a orientation should be given in $M$ (as there should be a canonical notion of positive length). Square lengths should
be elements of the tensor product $M\otimes M$, so instead of an inner product on $\Ker(\mathfrak t)$,
one should have a positive-definite symmetric bilinear form on $\Ker(\mathfrak t)$ taking values in $M\otimes M$.
A quotient of a length by an elapsed time (i.e., a {\em velocity}), would then be an element
of the one-dimensional vector space $M\otimes S^*$, where $S^*$ denotes the dual space of $S$
(Exercise~\ref{exe:units} clarifies the role of tensor products and dual spaces here).

We shall not pursue this programme further in this text. We just would like the reader to be aware that a better way to
formulate things in Physics is to have lots of given one-dimensional vector spaces, each one appropriate for the values
of a given type of physical dimension (such as elapsed time, length, mass, charge, etc) and lots of other one-dimensional
vector spaces constructed from the given ones using, for instance, tensor products and duals, which are spaces for the
values of things such as velocities, accelerations, forces, etc. Since working like this systematically would be too
annoying and distractive, we shall simply assume that units of measurement have been chosen and we will identify all such
would-be one-dimensional vector spaces with $\R$ once and for all.

\subsection{Inertial coordinate systems and the Galileo group}
We start by defining a notion of isomorphism for Galilean spacetimes.
Recall that, given affine spaces $E$, $E'$, with underlying vector spaces $V$, $V'$, respectively, then a map
$A:E\to E'$ is called {\em affine\/} if there exists a linear map $L:V\to V'$ such that:
\[A(v+e)=L(v)+A(e),\]
for all $v\in V$, $e\in E$. The linear map $L$ is unique when it exists and it is called the {\em underlying linear map\/}
of $A$. An affine map $A$ is called an {\em affine isomorphism\/} if it is bijective; this happens if and only if
the underlying linear map $L$ is bijective. When $A$ is an affine isomorphism, then its inverse $A^{-1}$ is also
affine, with underlying linear map $L^{-1}$. The affine automorphisms of $E$ (i.e., the affine isomorphisms from $E$ to $E$)
form a group $\Aff(E)$ under composition and the map $A\mapsto L$ that associates to each affine isomorphism $A$ its underlying
linear map $L$ is a group homomorphism from $\Aff(E)$ onto the general linear group $\GL(V)$ (i.e., the group
of linear isomorphisms of $V$). The kernel of such homomorphism is a normal subgroup of the group of affine
automorphisms of $E$; its elements are of the form $E\ni e\mapsto v+e\in E$,
with $v\in V$, and they are called {\em translations\/} of $E$. The group of all translations of $E$ is obviously isomorphic
to the additive group of $V$ (see Exercise~\ref{exe:shortexact}).

\begin{defin}\label{thm:isoGalileo}
Let $(E,V,\mathfrak t,\langle\cdot,\cdot\rangle)$, $(E',V',\mathfrak t',\langle\cdot,\cdot\rangle')$ be Galilean spacetimes.
An {\em isomorphism\/} $A:(E,V,t,\langle\cdot,\cdot\rangle)\to(E',V',t',\langle\cdot,\cdot\rangle')$ is an affine
isomorphism $A:E\to E'$ with underlying linear isomorphism $L:V\to V'$ such that:
\begin{itemize}
\item[(a)] $L$ {\em preserves time}, i.e., $\mathfrak t'\circ L=\mathfrak t$;
\item[(b)] $L$ carries the inner product $\langle\cdot,\cdot\rangle$ on $\Ker(\mathfrak t)$ to the inner product
$\langle\cdot,\cdot\rangle'$ on $\Ker(\mathfrak t')$, i.e., $\langle L(v_1),L(v_2)\rangle'=\langle v_1,v_2\rangle$,
for all $v_1,v_2\in\Ker(\mathfrak t)$.
\end{itemize}
\end{defin}
Notice that condition (a) above implies that $L$ sends $\Ker(\mathfrak t)$ onto $\Ker(\mathfrak t')$.
Galilean spacetimes and its isomorphisms form a category. Two Galilean spacetimes are always isomorphic, i.e.,
given two Galilean spacetimes there always exist an isomorphism from one to the other (see
Exercise~\ref{exe:isoGalileo}).

One might wonder why we care about defining isomorphisms of Galilean spacetimes or why should we ever talk about
``Galilean spacetimes'' in the plural when presenting a physical theory;
after all, in Physics we should have only one spacetime. We have two purposes in mind
for isomorphisms: one, is to consider the set of all automorphisms of a given Galilean spacetime. Such set
forms a group under composition and it is called the {\em Galileo group}. The other purpose is to give an elegant
definition of a privileged class of coordinate systems over Galilean spacetime, which is what we are going to do next.

Consider the affine space $\R^4$ canonically obtained from the vector space $\R^4$ (see Exercise~\ref{exe:canaffine}),
the time functional on the vector space $\R^4$ given by the projection onto the first coordinate and the canonical
Euclidean inner product on the kernel $\{0\}\times\R^3\cong\R^3$ of such time functional.
We have just defined a Galilean spacetime, which we call the
{\em Galilean spacetime of coordinates}. In what follows, we denote by
$(E,V,\mathfrak t,\langle\cdot,\cdot\rangle)$ the Galilean spacetime over which Classical Mechanics is going to be
formulated.

\begin{defin}
An isomorphism $\phi:E\to\R^4$ from the Galilean space time $E$ to the Galilean spacetime
of coordinates $\R^4$ is called an {\em inertial coordinate system}.
\end{defin}
Notice that we have avoided the (very common) terminology ``inertial observer''. This is no accident.
There is no need to talk about ``observers'' here. We will discuss this point in more detail later.

\begin{defin}
The group of automorphisms of the Galilean spacetime $E$ is called the {\em active Galileo group\/} and the
group of automorphisms of the Galilean spacetime of coordinates $\R^4$ is called the {\em passive Galileo group}.
\end{defin}
Obviously, the active and the passive Galileo groups are isomorphic, as any choice of inertial coordinate system
induces an isomorphism between them (see Exercise~\ref{exe:isoAutAut}); but such isomorphism is not canonical, in the
sense that it depends on the choice of the inertial coordinate system.
This two Galileo groups have distinct physical interpretations: elements of the active Galileo group are used to
transform stuff inside spacetime (say, move particle worldlines around), while elements of the passive Galileo
group are used to relate two inertial coordinate systems. More explicitly, if $\phi_1:E\to\R^4$, $\phi_2:E\to\R^4$
are inertial coordinate systems then the only map $A:\R^4\to\R^4$ such that $\phi_2=A\circ\phi_1$, i.e.,
such that the diagram:
\begin{equation}\label{eq:phi1phi2A}
\vcenter{\xymatrix{%
&E\ar[ld]_{\phi_1}\ar[rd]^{\phi_2}\\
\R^4\ar[rr]_A&&\R^4}}
\end{equation}
commutes is an element of the passive Galileo group.

We finish the section by taking a closer look at the Galileo group. Let $A:\R^4\to\R^4$ be an element of the passive
Galileo group. Then $A$ can be written as the composition of a linear isomorphism $L:\R^4\to\R^4$ with a translation
of $\R^4$:
\begin{equation}\label{eq:defAtx}
A(t,x)=L(t,x)+(t_0,x_0),\quad(t,x)\in\R\times\R^3=\R^4,
\end{equation}
where $(t_0,x_0)\in\R\times\R^3=\R^4$ is fixed. The linear isomorphism $L$ is required to satisfy conditions (a) and (b) in
Definition~\ref{thm:isoGalileo}. Condition (a) says that $L$ is of the form:
\begin{equation}\label{eq:defLtx}
L(t,x)=\big(t,L_0(x)-vt\big),\quad(t,x)\in\R\times\R^3=\R^4,
\end{equation}
for some fixed $v\in\R^3$ and some linear isomorphism $L_0:\R^3\to\R^3$. The expression $L_0(x)-vt$
just expresses an arbitrary $\R^3$-valued linear function of the pair $(t,x)$; the reason for us choosing
to write a minus sign in front of $v$ will be apparent in a moment. Condition
(b) now requires that the linear isomorphism $L_0$ be an isometry of $\R^3$, i.e., an element of the orthogonal group
$\Or(3)$. Let us consider the following subgroups of the passive Galileo group:
\begin{itemize}
\item the group of translations of $\R^4$, which is isomorphic to the additive group $\R^4$;
\item the group of isometries (i.e., rotations and reflections) of $\R^3$ (or, more precisely,
of $\{0\}\times\R^3$), which is the orthogonal group $\Or(3)$;
\item the group of linear isomorphisms of $\R^4$ of the form:
\begin{equation}\label{eq:boostv}
\R\times\R^3\ni(t,x)\longmapsto(t,x-vt)\in\R\times\R^3,
\end{equation}
with $v\in\R^3$. This group is isomorphic to the additive group $\R^3$ and its elements are called {\em Galilean boosts\/}
(see Exercise~\ref{exe:boosts} for the physical interpretation of Galilean boosts and for an explanation of why we prefer to use
a minus sign in front of $v$).
\end{itemize}
One immediately sees that every element of the passive Galileo group can be written in a unique way as a composition
of a translation of $\R^4$, an isometry of $\R^3$ and a Galilean boost. We can identify the passive Galileo group
with the cartesian product:
\begin{equation}\label{eq:identifyGalileo}
\R^4\times\Or(3)\times\R^3,
\end{equation}
by identifying the element $A$ defined by \eqref{eq:defAtx} and \eqref{eq:defLtx} with the triple
$\big((t_0,x_0),L_0,v\big)$. It is easy to check that the passive Galileo group is a closed Lie subgroup of the Lie
group\footnote{%
The group $\Aff(\R^4)$ has a differential structure because it is an open subset of the real finite-dimensional vector
space of all affine maps of $\R^4$. It is diffeomorphic to the cartesian product $\GL(\R^4)\times\R^4$, where the first
coordinate represents the linear part and the second the translation part.}
$\Aff(\R^4)$ of affine isomorphisms of $\R^4$ and that the map that identifies it with the cartesian product
\eqref{eq:identifyGalileo} is a smooth diffeomorphism (in particular, the Galileo group is a {\em ten dimensional\/} Lie group).
We emphasize, however, that the group structure on the cartesian
product \eqref{eq:identifyGalileo} that turns such map into a group isomorphism is {\em not\/} the one of the direct product
of the groups $\R^4$, $\Or(3)$ and $\R^3$. We left to the reader as an exercise to write down explicitly the appropriate
group structure on \eqref{eq:identifyGalileo} and to check that it coincides with the one of a {\em semi-direct product\/}
of the form:
\[\R^4\rtimes\big(\!\Or(3)\ltimes\R^3\big),\]
where $\Or(3)$ acts on $\R^3$ and $\Or(3)\ltimes\R^3$ acts on $\R^4$.

As for the active Galileo group, we have already observed that it is isomorphic to the passive Galileo group (by means
of a choice of an inertial coordinate system) and any
isomorphism between the two Galileo groups carries the three subgroups of the passive Galileo group defined above to subgroups of the
active Galileo group. However, except for the group of translations, the corresponding subgroups of the active
Galileo group {\em depend\/} on the choice of inertial coordinate system. Namely, it makes sense to say that
an affine isomorphism of the affine space $E$ is a (pure) translation, but it doesn't make sense to say that
it is purely linear (it only makes sense to say that an affine map is linear when its domain and counter-domain
are affine spaces which have been canonically obtained from vector spaces).

\end{section}

\begin{section}{Ontology and dynamics}
\label{sec:ontologydynamics}

Let us now present the ontology and the dynamics of Classical Mechanics. We will do this using an inertial coordinate
system, which will be fixed throughout the section. One must then check that the dynamics defined (i.e.,
the particle worldlines which are admissible by the theory) do not depend on the choice of inertial coordinate system.
This task is very simple and will be left to the reader (Exercises~\ref{exe:qtildeq}, \ref{exe:condForce}
and \ref{exe:condForcesat}). It is also
possible to give an {\em intrinsic\/} formulation of Classical Mechanics, i.e., to formulate it directly over Galilean
spacetime. To do so isn't terribly difficult, but we do not want to distract the reader with the technical complications
that arise during such task, so we relegate such formulation to an optional section (Section~\ref{sec:intrinsicClMech}).

Classical Mechanics is a theory about {\em particles}. Particles have {\em trajectories\/} that are
represented within the theory by smooth curves $q:\R\to\R^3$ (what we mean when we use the word ``particle'' is precisely
that such trajectories exist). The {\em graph\/} of the map $q$, i.e., the set:
\[\Gr(q)=\big\{\big(t,q(t)\big):t\in\R\big\}\subset\R^4\]
is called the {\em worldline\/} of the particle. More precisely, the worldline of the particle is the subset of the Galilean
spacetime $E$ that is mapped onto $\Gr(q)$ by our fixed inertial coordinate system; but we will keep referring
to $\Gr(q)$ as the worldline of the particle, anyway. Also, it should be mentioned that the map $q:\R\to\R^3$ becomes well-defined
only after a choice of an inertial coordinate system (as have been mentioned before, in the absence of a choice
of an inertial coordinate system, we do not even have an object to use as the counter-domain of the map $q$, i.e.,
we do not have an object representing space).

A universe described by Classical Mechanics contains a certain
number $n$ of particles, with trajectories\footnote{%
A particular correspondence between the particles and the numbers $1$, $2$, \dots, $n$ is obviously not meant to have
physical meaning. It would be more appropriate to use a general $n$-element set for labeling the particles, rather
than the set $\{1,2,\ldots,n\}$.}:
\[q_j:\R\longrightarrow\R^3,\quad j=1,2,\ldots,n.\]
To each particle it is associated a positive real number $m_j$ called the {\em mass\/} of the particle. The natural
number $n$ and the positive real numbers $m_j$ are parameters of the theory. This completes the description of the
ontology. Now, let us describe the dynamics. For each $j=1,\ldots,n$, the trajectory $q_j$ of the $j$-th particle
must satisfy the differential equation:
\begin{equation}\label{eq:dynamClMec}
\frac{\dd^2q_j}{\dd t^2}(t)=F_j\big(t,q_1(t),\ldots,q_n(t),\tfrac{\dd q_1}{\dd t}(t),\ldots,
\tfrac{\dd q_n}{\dd t}(t)\big),\quad t\in\R,
\end{equation}
where $F_j:\Dom(F_j)\subset\R\times(\R^3)^n\times(\R^3)^n\to\R^3$ is a smooth map defined over some open
subset $\Dom(F_j)$ of $\R\times(\R^3)^n\times(\R^3)^n$ and
it is called the {\em total force acting upon the $j$-th particle\/}
(actually, it is more appropriate to call the map $F_j$ the {\em force law\/} and the righthand side
of \eqref{eq:dynamClMec} the {\em force}).

What we have just presented is merely a prototype of a theory. In order to complete the formulation of the
theory, we have to present also a table of force laws, i.e., we have to say what the maps $F_j$ are.
We will now present just a very small table of force laws. One should keep in mind that by expanding such table
one can obtain new theories compatible with the prototype described above.

The force laws described below will only depend upon the positions $q_j(t)$ and thus we will omit the other variables.
\begin{itemize}
\item The {\em gravitational force}.\enspace For $i\ne j$, we set:
\[F_{ij}^{\mathrm{gr}}(q_1,\ldots,q_n)=\frac{Gm_im_j}{\Vert q_i-q_j\Vert^3}(q_j-q_i),\]
and we call it the {\em gravitational force of particle $j$ acting upon particle $i$\/} (or, alternatively,
the gravitational force that particle $j$ {\em exerts\/} upon particle $i$). Here $G$ denotes
a constant called {\em Newton's gravitational constant}.
\item The {\em electrical force}.\enspace For $i\ne j$, we set:
\[F_{ij}^{\mathrm{el}}(q_1,\ldots,q_n)=-\frac{Ce_ie_j}{\Vert q_i-q_j\Vert^3}(q_j-q_i),\]
and we call it the {\em electrical force of particle $j$ acting upon particle $i$\/} (or, alternatively,
the electrical force that particle $j$ {\em exerts\/} upon particle $i$). Here $C$ denotes
a constant called {\em Coulomb's constant\/} and $e_j\in\R$, $j=1,\ldots,n$, is a new parameter of the theory
called the {\em charge\/} of the $j$-th particle.
\end{itemize}
We define the {\em total gravitational force acting upon particle $j$\/} by:
\[F_j^{\mathrm{gr}}=\sum_{i\ne j}F_{ji}^{\mathrm{gr}},\]
and, similarly, the {\em total electrical force acting upon particle $j$\/} by:
\[F_j^{\mathrm{el}}=\sum_{i\ne j}F_{ji}^{\mathrm{el}}.\]
Now we state a rule that says that {\em the forces in our table should be added in order that the total
force $F_j$ be obtained}. Thus, the total force $F_j$ acting upon particle $j$ is given by:
\[F_j=F_j^{\mathrm{gr}}+F_j^{\mathrm{el}}.\]

Notice that both for the gravitational and for the electrical force we have:
\begin{equation}\label{eq:actionreaction}
F_{ij}=-F_{ji}.
\end{equation}
This is sometimes called {\em Newton's law of reciprocal actions}. It implies that:
\begin{equation}\label{eq:totalforcezero}
\sum_{j=1}^nF_j=\sum_{j=1}^n\sum_{i\ne j}F_{ji}=0.
\end{equation}

\medskip

The gravitational and the electrical forces are the {\em fundamental\/} forces of Classical Mechanics. When dealing with
practical physics problems, other forces do arise, such as friction, viscosity, contact forces, etc. Those other forces
are supposed to be {\em emergent\/} forces; one would not need them if all the microscopic details of the interactions
were to be taken into account\footnote{%
Of course, this cannot be taken too seriously, as we know that Classical Mechanics is not a fundamental theory.
Classical Mechanics does not work properly at the microscopic scale.}.

Also, we should mention {\em external forces}. A physical theory, in principle, is supposed to be about the entire universe.
So, in the universe described by Classical Mechanics as formulated above, there is nothing but those $n$ particles.
Evidently, one would like also to apply the theory to subsystems of the universe. For Classical Mechanics, a subsystem
of the universe is obtained by choosing a subset of the set of all particles, i.e., a subset $I$ of the set of labels
$\{1,2,\ldots,n\}$. Set $I^\compl=\{1,2,\ldots,n\}\setminus I$. One could then compare the dynamics of the particles with labels in $I$
when the presence of the particles with labels in $I^\compl$ is taken into account with the dynamics of the particles
with labels in $I$ when the presence
of the particles with labels in $I^\compl$ is {\em not\/} taken into account. One should not expect
the two dynamics to be identical, but in many cases it might happen that those two dynamics are very similar (observe,
for instance, that $F_{ij}$ becomes very small when $\Vert q_i-q_j\Vert$ becomes very large).
In those cases, we say that the subsystem defined by $I$ is {\em almost isolated}.
There is another possibility: maybe we don't
get a good approximation of the dynamics of the particles with labels in $I$ by ignoring the existence of the particles
with labels in $I^\compl$, but we do get a good approximation of the dynamics of the particles with labels in $I$ by
taking into account only the forces exerted upon particles with labels in $I$ by particles with labels in $I^\compl$ and
by ignoring the forces exerted upon particles with labels in $I^\compl$ by particles with labels in $I$ (for example,
we can study the dynamics of a tennis ball near the Earth by taking into account the gravitational force exerted by the
Earth upon the tennis ball, and by ignoring the gravitational force exerted by the tennis ball upon the Earth).
Thus, we could write down the differential equations for the trajectories of the particles with labels in $I$ using
the {\em internal forces\/} of the subsystem defined by $I$ (i.e., the forces between particles with labels in $I$)
and also the {\em external forces\/} (i.e., the forces exerted upon the particles with labels in $I$ by particles
with labels in $I^\compl$). There is also another type of situation in which we would like to talk about external forces.
Sometimes, in Physics, we need to make ``Frankenstein theories'', mixing up pieces of (not
necessarily fully compatible) distinct theories. For instance, one might like to consider the particles of Classical Mechanics
moving inside a magnetic field. Magnetic fields are not part of Classical Mechanics: they are part of Maxwell's electromagnetism,
a theory which is not even formulated within Galilean spacetime (it is formulated within {\em Minkowski spacetime}).
Nevertheless, for practical applications, one might well be willing to consider the particles of Classical Mechanics
inside a magnetic field and thus one would have to consider the force exerted by the magnetic field upon the particles
(the {\em Lorentz force}).

\subsection{Forces with a potential}
\label{sub:potential}

Taking together all the force maps $F_j$ we obtain a map:
\[F=(F_1,\ldots,F_n):\Dom(F)\subset\R\times(\R^3)^n\times(\R^3)^n\longrightarrow(\R^3)^n.\]
If the righthand side of \eqref{eq:dynamClMec} does not depend on the velocities $\frac{\dd q_j}{\dd t}(t)$ then
we can think of $F$ as a map of the form:
\[F:\Dom(F)\subset\R\times(\R^3)^n\longrightarrow(\R^3)^n.\]
That is a {\em time-dependent vector field\/} over $(\R^3)^n$, i.e., for each $t\in\R$, we have a vector field
$F(t,\cdot)$ over (an open subset of) $(\R^3)^n$. When the forces $F_j$ do not depend on the velocities and
there exists a smooth map:
\[V:\Dom(V)=\Dom(F)\subset\R\times(\R^3)^n\longrightarrow\R\]
such that\footnote{%
Why do we use a minus sign in the righthand side of \eqref{eq:FgradV}? It is a good convention. For instance,
it allows us to build some intuition about the behavior of the particles by thinking about the graph of $V$ as some
sort of roller coaster.}:
\begin{equation}\label{eq:FgradV}
F(t,q)=-\nabla_qV(t,q),\quad(t,q)\in\Dom(F)\subset\R\times(\R^3)^n,
\end{equation}
then we call $V$ a {\em potential\/} for the force $F$. In \eqref{eq:FgradV} we have denoted by
$\nabla_qV(t,q)$ the gradient of the map $V(t,\cdot)$ evaluated at the point $q$. Forces that admit a potential
will be essential for the variational formulation of Classical Mechanics (Subsection~\ref{sub:varClassMech}).
When $F(t,q)$ does not depend on $t$, then, as is well-known from elementary calculus, the existence of the potential
$V$ is equivalent to the condition that the line integral $\int_qF$ of the vector field $F$ over a piecewise
smooth curve $q:[a,b]\to(\R^3)^n$ depend only on the endpoints $q(a)$, $q(b)$ of the curve\footnote{%
If the curve is $q=(q_1,\ldots,q_n)$, with $q_j$ denoting the trajectory of the $j$-th particle, then such
line integral is called the {\em work\/} done by the force $F$.}. In that case the force $F$ is called
{\em conservative}. We leave as a simple exercise to the reader to check that the following:
\[V^{\mathrm{gr}}(q_1,\ldots,q_n)=-\sum_{i<j}\frac{Gm_im_j}{\Vert q_i-q_j\Vert},\quad
V^{\mathrm{el}}(q_1,\ldots,q_n)=\sum_{i<j}\frac{Ce_ie_j}{\Vert q_i-q_j\Vert},\]
are potentials for the gravitational and for the electrical forces, respectively.
Obviously, $V=V^{\mathrm{gr}}+V^{\mathrm{el}}$ is a potential for $F=F^{\mathrm{gr}}+F^{\mathrm{el}}$.

\end{section}

\begin{section}{Optional section: intrinsic formulation}
\label{sec:intrinsicClMech}

In this section we present an intrinsic (i.e., coordinate system free) formulation of the ontology and
dynamics of Classical Mechanics. Material from this section will not be used elsewhere, so uninterested readers
may safely skip it. While some might complain that this section contains too much ``abstract nonsense'', we think
that something is learned from this exercise. In what follows, a Galilean spacetime
$(E,V,\mathfrak t,\langle\cdot,\cdot\rangle)$ is fixed.

Consider the quotient:
\[\mathfrak T=E/\Ker(\mathfrak t),\]
which is a one-dimensional real affine space with underlying vector space $V/\Ker(\mathfrak t)$ (see Exercise~\ref{eq:quotientaffine}).
The linear functional $\mathfrak t$ induces an isomorphism between the quotient $V/\Ker(\mathfrak t)$ and the real
line $\R$; by means of such isomorphism, we may regard $\R$ as the underlying vector space of the
affine space $\mathfrak T$. Denote by:
\[\bar{\mathfrak t}:E\longrightarrow\mathfrak T\]
the quotient map, which is an affine map whose underlying linear map is $\mathfrak t$. A point $t$ of the affine space
$\mathfrak T$ is called an {\em instant of time}; the inverse image of $t$ by $\bar{\mathfrak t}$ is a simultaneity
hyperplane, which is to be interpreted as {\em space at the instant $t$}.

Recall that a real finite-dimensional affine space is, in a natural way, a differentiable manifold whose tangent space
at an arbitrary point is naturally identified with the underlying vector space of the affine space
(see Exercise~\ref{exe:affspacemanifold}). The affine map $\bar{\mathfrak t}$ is smooth\footnote{%
Actually, it is a smooth fibration.} (see Exercise~\ref{exe:affinesmooth}).
By a {\em section\/} of $\bar{\mathfrak t}$ we mean a map $q:\mathfrak T\to E$ such that
$\bar{\mathfrak t}\circ q$ is the identity map of $\mathfrak T$. We can define a {\em particle worldline\/} to be
the image of a smooth section $q$ of $\bar{\mathfrak t}$; clearly, the section $q$ is uniquely determined by the
corresponding worldline. Equivalently, one can define a particle worldline to be a smooth submanifold of $E$
that is mapped diffeomorphically onto $\mathfrak T$ by the map $\bar{\mathfrak t}$.

Let $q:\mathfrak T\to E$ be a smooth section of $\bar{\mathfrak t}$.
Given $t\in\mathfrak T$, then the tangent space $T_t\mathfrak T$ is canonically identified with $\R$ and the tangent
space $T_{q(t)}E$ is canonically identified with $V$; thus, the differential $\dd q(t)$ is a linear map from $\R$
to $V$. We set:
\[\dot q(t)=\dd q(t)1\in V,\]
and we call it the {\em velocity at the instant $t\in\mathfrak T$\/} of a particle whose worldline is the image of $q$.
Since $\bar{\mathfrak t}\circ q$ is the identity of $\mathfrak T$, it follows by differentiation that $\mathfrak t\circ\dd q(t)$
is the identity of $\R$, so that, in particular:
\[\mathfrak t\big(\dot q(t)\big)=1.\]
The velocity $\dot q(t)$ is a generator of the (one-dimensional) tangent space at the point $q(t)$
of the worldline $q(\mathfrak T)$; actually, it is the only vector in that tangent space that is mapped by $\mathfrak t$ to the number $1$.
The map $\dot q$ (which is essentially the differential of $q$) is smooth and it takes values in the affine subspace
$\mathfrak t^{-1}(1)$ of $V$:
\[\dot q:\mathfrak T\longrightarrow\mathfrak t^{-1}(1)\subset V.\]
The underlying vector space of the affine space $\mathfrak t^{-1}(1)$ is $\Ker(\mathfrak t)$.
We can differentiate the map $\dot q$ at some $t\in\mathfrak T$ to obtain a linear map:
\[\dd\dot q(t):\R\longrightarrow\Ker(\mathfrak t).\]
We set:
\[\ddot q(t)=\dd\dot q(t)1\in\Ker(\mathfrak t).\]
We call $\ddot q(t)$ the {\em acceleration at the instant $t\in\mathfrak T$\/} of a particle whose
worldline is the image of $q$.

We have learned some interesting things: velocities $\dot q(t)$ are elements of the affine space $\mathfrak t^{-1}(1)$.
So we cannot add two velocities and get a new velocity! On the other hand, we can subtract two velocities and obtain
an element of the three-dimensional vector space $\Ker(\mathfrak t)$. A difference of velocities is a {\em relative velocity}; since
$\Ker(\mathfrak t)$ is endowed with an inner product, we can talk about the norm of a relative velocity. That's
a {\em relative speed}. Notice that, unlike velocities, accelerations $\ddot q(t)$ are elements of the vector space
$\Ker(\mathfrak t)$, so we can add them and multiply them by real numbers obtaining new elements of $\Ker(\mathfrak t)$.
It is also meaningful to take the norm of an acceleration.

The intrinsic formulation of Classical Mechanics is almost done. The trajectories $q_j:\R\to\R^3$ used
in the formulation of Section~\ref{sec:ontologydynamics} must be replaced
by sections $q_j:\mathfrak T\to E$ of $\bar{\mathfrak t}$. The lefthand side of equality \eqref{eq:dynamClMec}
is replaced with $m_j\ddot q_j(t)$. We just have to explain what type of objects should the maps $F_j$ be
replaced with.

Consider the cartesian product $E^n$ of $n$ copies of $E$. This is an affine space with underlying vector
space $V^n$. Let $Q$ denote the subset of $E^n$ defined by:
\[Q=\big\{(q_1,\ldots,q_n)\in E^n:\mathfrak t(q_i-q_j)=0,\ i,j=1,\ldots,n\big\}.\]
The set $Q$ is an affine subspace of $E^n$ with underlying vector space:
\[\big\{(v_1,\ldots,v_n)\in V^n:\mathfrak t(v_i-v_j)=0,\ i,j=1,\ldots,n\big\}.\]
The affine space $Q$ is called {\em configuration spacetime}. Notice that, for each $t$ in $\mathfrak T$,
the $n$-tuple $\big(q_1(t),\ldots,q_n(t)\big)$ belongs to $Q$. In the intrinsic formulation of Classical Mechanics,
the force laws $F_j$ are maps of the form:
\begin{equation}\label{eq:intrinsicforce}
F_j:\Dom(F_j)\subset Q\times\mathfrak t^{-1}(1)^n\longrightarrow\Ker(\mathfrak t).
\end{equation}
It is readily checked that both the gravitational and the electrical forces are well-defined maps
of the form \eqref{eq:intrinsicforce}.

\end{section}

\begin{section}{An introduction to variational calculus}
\label{sec:introvariations}

A {\em variational problem\/} is a particular case of the problem of finding a critical point of a map
whose domain is typically infinite-dimensional. We proceed to describe an important class of variational problems
for curves.

Let $[a,b]$ be an interval and consider the vector space $C^\infty\big([a,b],\R^n\big)$ of smooth maps
$q:[a,b]\to\R^n$. Given points $q_a,q_b\in\R^n$, then the set:
\[C^\infty_{q_aq_b}\big([a,b],\R^n\big)=\big\{q\in C^\infty\big([a,b],\R^n\big):q(a)=q_a,\ q(b)=q_b\big\}\]
is an affine subspace of $C^\infty\big([a,b],\R^n\big)$ whose underlying vector space is the space
$C^\infty_{00}\big([a,b],\R^n\big)$ of smooth maps from $[a,b]$ to $\R^n$ vanishing at the endpoints of the
interval $[a,b]$. Consider a smooth map:
\[L:\R\times\R^n\times\R^n\longrightarrow\R\]
and define:
\[S_L:C^\infty_{q_aq_b}\big([a,b],\R^n)\longrightarrow\R\]
by setting:
\[S_L(q)=\int_a^bL\big(t,q(t),\dot q(t)\big)\,\dd t,\]
where $\dot q(t)=\frac{\dd q}{\dd t}(t)$. The map $L$ is called a {\em Lagrangian\/} and $S_L$ is called the
corresponding {\em action functional}. More precisely, we have a {\em family\/} of action functionals associated to the Lagrangian $L$
(one action functional for each interval $[a,b]$ and for each choice of points $q_a,q_b\in\R^n$); but we will be
a little sloppy and refer to any of them as ``the action functional''.
The variational problem that we are going to consider is the problem
of finding the critical points of $S_L$. But we do not intend to discuss infinite-dimensional calculus seriously,
because we don't have to. We will just present a definition of critical point for this specific context.
If we were to discuss infinite-dimensional calculus seriously, then it would be preferable to replace
$C^\infty\big([a,b],\R^n\big)$ with the space $C^k\big([a,b],\R^n\big)$ of maps of class $C^k$, for some fixed
finite $k$. The space $C^k\big([a,b],\R^n\big)$, endowed with the appropriate norm (for instance, the
sum of the supremum norm of the function with the supremum norms of its first $k$ derivatives) is a Banach space\footnote{%
The space $C^k_{q_aq_b}\big([a,b],\R^n\big)$ is then a closed affine subspace of a Banach space and therefore
it is a Banach manifold.}, while $C^\infty\big([a,b],\R^n\big)$ can only handle the structure of a {\em Fr\'echet space}.
Calculus on Banach spaces is simpler to handle (it is more similar to finite-dimensional calculus)
than calculus on Fr\'echet spaces. But we are not going to need any theorems from infinite-dimensional calculus,
so we don't have to worry about any of that.

\begin{defin}\label{thm:defvariation}
Let $q:[a,b]\to\R^n$ be a smooth curve. By a {\em variation\/} of $q$ we mean a family $(q_s)_{s\in I}$
of smooth curves $q_s:[a,b]\to\R^n$, where $I\subset\R$ is an open interval with $0\in I$,
$q_0$ equals $q$ and the map:
\[I\times[a,b]\ni(s,t)\longmapsto q_s(t)\in\R^n\]
is smooth. We say that the variation $(q_s)_{s\in I}$ has {\em fixed endpoints\/} if the maps:
\[I\ni s\longmapsto q_s(a)\in\R^n,\quad I\ni s\longmapsto q_s(b)\in\R^n\]
are constant. The {\em variational vector field\/} of a variation $(q_s)_{s\in I}$ is the smooth map
$v:[a,b]\to\R^n$ defined by:
\begin{equation}\label{eq:defvarvecfield}
v(t)=\left.\frac{\dd}{\dd s}\,q_s(t)\right\vert_{s=0},\quad t\in[a,b].
\end{equation}
\end{defin}
Clearly, if $v$ is the variational vector field of a variation with fixed endpoints then $v(a)=v(b)=0$.

Notice that a smooth curve $q:[a,b]\to\R^n$ is a point of the space $C^\infty_{q_aq_b}\big([a,b],\R^n\big)$,
where $q_a=q(a)$, $q_b=q(b)$; a variation with fixed endpoints of $q$ is a curve $s\mapsto q_s$ in
$C^\infty_{q_aq_b}\big([a,b],\R^n\big)$ passing through the point $q$ at $s=0$. The corresponding variational
vector field $v$ is like the vector tangent to that curve at $s=0$.
\begin{defin}\label{thm:defcriticalSL}
We say that a smooth curve $q:[a,b]\to\R^n$ is a {\em critical point\/} of the action functional $S_L$ if
\begin{equation}\label{eq:diffSL}
\left.\frac{\dd}{\dd s}\,S_L(q_s)\right\vert_{s=0}=0,
\end{equation}
for every variation with fixed endpoints $(q_s)_{s\in I}$ of $q$.
\end{defin}
In the context of infinite-dimensional calculus, the map $S_L$ is smooth and the lefthand side of
\eqref{eq:diffSL} is precisely the differential of $S_L$ at the point $q$ in the direction $v$, where $v$ is the
variational vector field of $(q_s)_{s\in I}$. With our approach, however, it is not clear in principle that the
lefthand side of \eqref{eq:diffSL} depends only on $v$ (i.e., that two variations with the same variational vector
field would yield the same value for the lefthand side of \eqref{eq:diffSL}) and that it is linear in $v$.
Nevertheless, as we will see now, a very simple direct computation of the lefthand side of \eqref{eq:diffSL}
shows that both things are true. Using any standard result about differentiation under the integral sign and
the chain rule we obtain:
\begin{equation}\label{eq:computedSL}
\left.\frac{\dd}{\dd s}\,S_L(q_s)\right\vert_{s=0}=
\int_a^b\frac{\partial L}{\partial q}\big(t,q(t),\dot q(t)\big)v(t)+
\frac{\partial L}{\partial\dot q}\big(t,q(t),\dot q(t)\big)\dot v(t)\,\dd t,
\end{equation}
where $\dot v(t)=\frac{\dd v}{\dd t}(t)$. Above, we have denoted by $\frac{\partial L}{\partial q}$ and
by $\frac{\partial L}{\partial\dot q}$ the differential of $L$ with respect to its second and third variables;
thus, in $\frac{\partial L}{\partial\dot q}$ the symbol $\dot q$ is merely a label, not ``the derivative of $q$''.
Notice that the expressions:
\[\frac{\partial L}{\partial q}\big(t,q(t),\dot q(t)\big),\quad\frac{\partial L}{\partial\dot q}\big(t,q(t),\dot q(t)\big)\]
are differentials evaluated at a point of real valued functions over $\R^n$ and therefore they are elements of the dual space
${\R^n}^*$ (thus they can be applied to vectors $v(t)$, $\dot v(t)$ of $\R^n$, as we did in \eqref{eq:computedSL}).
We won't systematically take too seriously the difference between $\R^n$ and ${\R^n}^*$ and we will sometimes
identify the two spaces in the usual way.

Now we want to use integration by parts in \eqref{eq:computedSL} to get rid of $\dot v(t)$. In other words, we observe
that:
\begin{multline}\label{eq:SLintegparts}
\frac{\partial L}{\partial\dot q}\big(t,q(t),\dot q(t)\big)\dot v(t)=
\frac{\dd}{\dd t}\Big(\frac{\partial L}{\partial\dot q}\big(t,q(t),\dot q(t)\big)v(t)\Big)\\
-\Big(\frac{\dd}{\dd t}\frac{\partial L}{\partial\dot q}\big(t,q(t),\dot q(t)\big)\Big)v(t).
\end{multline}
Using the fundamental theorem of calculus and the fact that:
\[v(a)=v(b)=0\]
for variations with fixed endpoints, we see
that the first term on the righthand side of \eqref{eq:SLintegparts} vanishes once it goes inside the integral sign.
Thus:
\begin{multline}\label{eq:dSLcomputed}
\left.\frac{\dd}{\dd s}\,S_L(q_s)\right\vert_{s=0}=\\
\int_a^b\Big({-\frac{\dd}{\dd t}\frac{\partial L}{\partial\dot q}
\big(t,q(t),\dot q(t)\big)}+\frac{\partial L}{\partial q}\big(t,q(t),\dot q(t)\big)\Big)v(t)\,\dd t.
\end{multline}
We want to infer from \eqref{eq:dSLcomputed} that $q$ is a critical point of $S_L$ if and only if the big expression
inside the parenthesis in \eqref{eq:dSLcomputed} vanishes, i.e., if and only if:
\begin{equation}\label{eq:EulerLagrange}
\frac{\dd}{\dd t}\frac{\partial L}{\partial\dot q}\big(t,q(t),\dot q(t)\big)=
\frac{\partial L}{\partial q}\big(t,q(t),\dot q(t)\big),\quad t\in[a,b].
\end{equation}
The differential equation \eqref{eq:EulerLagrange} is called the {\em Euler--Lagrange equation}. Obviously,
if the Euler--Lagrange equation holds then, by \eqref{eq:dSLcomputed}, $q$ is a critical point of $S_L$. In order
to prove the converse, we need two ingredients. The first, is the observation that any smooth map $v:[a,b]\to\R^n$
with $v(a)=v(b)=0$ is the variational vector field of some variation of $q$ with fixed endpoints. For instance,
we can consider the variation\footnote{%
Some authors define critical points using exclusively variations of this type. This is not a good idea, since this
breaks the manifest invariance under general diffeomorphisms of the notion of critical point!}:
\[q_s(t)=q(t)+sv(t).\]
By this observation, the assumption that $q$ be a critical point of $S_L$ implies that the righthand side of
\eqref{eq:dSLcomputed} vanishes, for any smooth map $v$ such that $v(a)=v(b)=0$. The second ingredient
is this.

\begin{lem}[fundamental lemma of the calculus of variations]
Let $\alpha:[a,b]\to{\R^n}^*$ be a continuous map and assume that:
\begin{equation}\label{eq:intalphavdt}
\int_a^b\alpha(t)v(t)\,\dd t=0,
\end{equation}
for any smooth map $v:[a,b]\to\R^n$ having support contained in the open interval $\left]a,b\right[$.
Then $\alpha=0$.
\end{lem}
\begin{proof}
Assuming by contradiction that $\alpha$ is not zero, then the $j$-th coordinate
$\alpha_j:[a,b]\to\R$ of $\alpha$ is not zero for some $j=1,\ldots,n$. By continuity, $\alpha_j$ never vanishes
(and has a fixed sign) over some interval $[c,d]$ contained in $\left]a,b\right[$. One can construct
a smooth $v$ whose only non vanishing coordinate is the $j$-th coordinate $v_j$ and such that $v_j$ is non negative
(but not identically zero) over $[c,d]$ and vanishes outside $[c,d]$. Then the integral in
\eqref{eq:intalphavdt} will not be zero.
\end{proof}
It is possible to prove a more general version of the fundamental lemma of the calculus of variations, by assuming
$\alpha$ to be merely Lebesgue integrable; in that case, the thesis says that $\alpha$ vanishes almost everywhere.
This version is a little harder to prove and we are not going to need it.

As we have seen, if $q$ is a critical point of $S_L$ then the integral in \eqref{eq:dSLcomputed} vanishes
for every smooth $v:[a,b]\to\R^n$ such that $v(a)=v(b)=0$ and thus, by the fundamental lemma of the calculus
of variations, it follows that the Euler--Lagrange equation is satisfied. We have just proven:

\begin{teo}\label{thm:teoEL}
A smooth curve $q:[a,b]\to\R^n$ is a critical point of $S_L$ if and only if it satisfies the Euler--Lagrange
equation \eqref{eq:EulerLagrange}.\qed
\end{teo}

The lefthand side of the Euler--Lagrange equation \eqref{eq:EulerLagrange} should {\em not\/} be confused with
$\frac{\partial^2L}{\partial t\partial\dot q}\big(t,q(t),\dot q(t)\big)$ (for instance, such expression is automatically
zero if $L$ does not depend on $t$). The lefthand side of the Euler--Lagrange equation is the derivative
of the map $t\mapsto\frac{\partial L}{\partial\dot q}\big(t,q(t),\dot q(t)\big)$.

\begin{rem}\label{thm:remdomLopen}
We have worked so far with a Lagrangian whose domain is $\R\times\R^n\times\R^n$. We could have used a Lagrangian
whose domain is an open subset of $\R\times\R^n\times\R^n$ instead. Some obvious adaptations would have to be done in our
definitions. For instance, the domain of the action functional $S_L$ would consist only of curves $q$ such that
$\big(t,q(t),\dot q(t)\big)$ is in the domain of $L$, for all $t\in[a,b]$. Also, in the definition of critical point,
one should only consider curves $q:[a,b]\to\R^n$ that are in the domain of $S_L$ and only variations
$(q_s)_{s\in I}$ of $q$ such that $q_s$ is in the domain of $S_L$, for all $s\in I$. Observe that if $(q_s)_{s\in I}$
is an arbitrary variation of a curve $q$ that is in the domain of $S_L$ then there exists an open interval $I'\subset I$
containing the origin such that $q_s$ is in the domain of $S_L$, for all $s\in I'$. Namely, the set of pairs
$(s,t)\in I\times[a,b]$ such that $\big(t,q_s(t),\frac{\dd}{\dd t}\,q_s(t)\big)$ is in the domain of $L$ is open
in $I\times[a,b]$ and contains $\{0\}\times[a,b]$; since $[a,b]$ is compact, it follows that such set contains
$I'\times[a,b]$ for some open interval $I'\subset I$ containing the origin.
Obviously, Theorem~\ref{thm:teoEL} also holds for a Lagrangian defined in an open subset
of $\R\times\R^n\times\R^n$.
\end{rem}

\begin{rem}\label{thm:remcompsupport}
We have actually proved a little more than what is stated in Theorem~\ref{thm:teoEL}.
A variation $(q_s)_{s\in I}$ is said to have {\em compact support\/} if the map:
\[I\ni s\longmapsto q_s(t)\in\R^n\]
is constant for $t$ in a neighborhood of $a$ in $[a,b]$ and for $t$ in a neighborhood of $b$ in $[a,b]$.
Our proof of Theorem~\ref{thm:teoEL} has actually shown that if \eqref{eq:diffSL} holds for all variations
of $q$ having compact support then $q$ satisfies the Euler--Lagrange equation.
\end{rem}

\subsection{Variational formulation of Classical Mechanics}
\label{sub:varClassMech}

Now it is easy to see that when the force $F=(F_1,\ldots,F_n)$ admits a potential $V$
(see Subsection~\ref{sub:potential}) then the differential equation \eqref{eq:dynamClMec} defining the dynamics of
Classical Mechanics is precisely the Euler--Lagrange equation of the Lagrangian
$L:\Dom(V)\times(\R^3)^n\subset\R\times(\R^3)^n\times(\R^3)^n\to\R$ defined by:
\begin{equation}\label{eq:Lmechanics}
L(t,q_1,\ldots,q_n,\dot q_1,\ldots,\dot q_n)=\sum_{j=1}^n\frac12\,m_j\Vert\dot q_j\Vert^2-V(t,q_1,\ldots,q_n).
\end{equation}
Namely, the Euler--Lagrange equation for $L$ can be written as:
\[\frac{\dd}{\dd t}\frac{\partial L}{\partial\dot q_j}\big(t,q(t),\dot q(t)\big)=
\frac{\partial L}{\partial q_j}\big(t,q(t),\dot q(t)\big),\quad j=1,\ldots,n,\]
and (identifying $\R^3$ with its dual space):
\begin{gather*}
\frac{\dd}{\dd t}\frac{\partial L}{\partial\dot q_j}\big(t,q(t),\dot q(t)\big)=\frac{\dd}{\dd t}\big(m_j\dot q_j(t)\big)
=m_j\frac{\dd^2q_j}{\dd t^2}(t),\\
\frac{\partial L}{\partial q_j}\big(t,q(t),\dot q(t)\big)=-\nabla_{q_j}V\big(t,q(t)\big)=F_j\big(t,q(t)\big).
\end{gather*}

The Lagrangian \eqref{eq:Lmechanics} motivates the following definition.
\begin{defin}\label{thm:defenergies}
We call $\frac12\,m_j\Vert\dot q_j(t)\Vert^2$ the {\em kinetic energy\/} of the $j$-th particle at the instant $t$
and the sum $\sum_{j=1}^n\frac12\,m_j\Vert\dot q_j(t)\Vert^2$ appearing in \eqref{eq:Lmechanics} the {\em total kinetic
energy\/} at the instant $t$. We also call $V\big(t,q(t)\big)$ the {\em potential energy\/} at the instant $t$.
\end{defin}
It is common to denote the total kinetic energy by $T$, so that the Lagrangian becomes $L=T-V$.
Notice that we can define a kinetic energy for just one particle, but not a potential energy for just one particle.

What is so good about rewriting the dynamics of Classical Mechanics in terms of an Euler--Lagrange equation?
One motivation is that the Euler--Lagrange equation is good for doing arbitrary transformations of coordinates
(even time-dependent transformations of coordinates\footnote{%
One could be willing to use, say, spherical coordinates based on an orthonormal basis that
rotates!}), i.e., after a transformation of coordinates the Euler--Lagrange equation is transformed into
the Euler--Lagrange equation of a transformed Lagrangian (what we mean exactly by this will become clearer when we do
variational calculus on manifolds). On the other hand, equation \eqref{eq:dynamClMec} isn't good for
arbitrary transformations of coordinates (for instance, under a general transformation of coordinates
the lefthand side of \eqref{eq:dynamClMec} gets a term with a first order derivative of $q$).
Also, essentially all equations in Physics (the field equations of Electromagnetism, of General Relativity and
of the Gauge theories of the Standard Model, for instance) can be obtained from variational problems. So, one usually talks
about ``the Lagrangian'' of the theory. We will also see that Lagrangians (and {\em Hamiltonians}) are essential for
doing Quantum Theory.

\begin{rem}
The Lorentz force (exerted by a magnetic field upon a charged particle) is an important example of a force
which depends on the velocity of the particle and therefore it is not a force with a potential (in the
sense defined in Subsection~\ref{sub:potential}). Nevertheless, as we will see later, it is possible to handle it using
a Lagrangian.
\end{rem}

\end{section}

\begin{section}{Lagrangians on manifolds}
\label{sec:Lagrmanifolds}

The variational problem discussed in Section~\ref{sec:introvariations} can be straightforwardly reformulated for curves
on manifolds. One can then prove that the critical points of the action functional are again the curves which satisfy
the Euler--Lagrange equation. But, in a manifold, in order to give meaning to the statement that a curve satisfies
the Euler--Lagrange equation, one has to use a coordinate chart; it turns out that such statement does not
depend on the choice of the coordinate chart. There are some minor technical complications that arise in the context
of manifolds and for the reader's convenience we give all details. We won't try to give a manifestly coordinate independent
formulation of the Euler--Lagrange equation; attempts at doing so tend to get nasty and it happens that for an important
class of Lagrangians ({\em hyper-regular\/} Lagrangians), the Euler--Lagrange equation is equivalent to {\em Hamilton's equations\/}
(which we will present later on) and those are easily formulated in a manifestly coordinate independent manner.

Let $Q$ be a differentiable manifold. Given an interval $[a,b]$, we denote by $C^\infty\big([a,b],Q\big)$ the set
of all smooth maps $q:[a,b]\to Q$ and, given points $q_a,q_b\in Q$, we denote by $C^\infty_{q_aq_b}\big([a,b],Q\big)$
the subset of $C^\infty\big([a,b],Q\big)$ consisting of maps $q$ with $q(a)=q_a$, $q(b)=q_b$. In order to define an action
functional $S_L:C^\infty_{q_aq_b}\big([a,b],Q\big)\to\R$ of the form:
\begin{equation}\label{eq:defSLmanifold}
S_L(q)=\int_a^bL\big(t,q(t),\dot q(t)\big)\,\dd t,
\end{equation}
the appropriate domain for the map $L$ is the product $\R\times TQ$, where $TQ$ denotes the tangent bundle of $Q$.
So, we define a {\em Lagrangian\/} on the manifold $Q$ to be a smooth map:
\[L:\R\times TQ\longrightarrow\R\]
and the map $S_L$ defined above is called the corresponding {\em action functional}. As before, we should point
out that in fact there is a family of action functionals for a given Lagrangian $L$ (one for each interval $[a,b]$
and for each choice of points $q_a,q_b\in Q$), but as before we will be a little sloppy and use the same name and notation
for all of them. Also, as before (see Remark~\ref{thm:remdomLopen}), it is possible to work with a map $L$ whose domain
is some open subset of $\R\times TQ$ and in order to do that there are obvious adaptations in the definitions and proofs
that follow; we don't want to distract the reader with things like that. In order to avoid awkward moments in the future,
let us make a warning about notation: if $q:[a,b]\to Q$ is a differentiable curve in a manifold then for each $t\in[a,b]$
the derivative $\dot q(t)$ of $q$ at $t$ is an element of the tangent space $T_{q(t)}Q$ which is a subset
of the tangent bundle $TQ$. So, strictly speaking, the notation in \eqref{eq:defSLmanifold} is wrong; it is
$\big(t,\dot q(t)\big)$, and not $\big(t,q(t),\dot q(t)\big)$, which is a point of the domain $\R\times TQ$ of the Lagrangian
$L$. Nevertheless, we find it convenient in many cases to write elements of the tangent bundle $TQ$ as ordered pairs
consisting of a point of $Q$ and a tangent vector at that point. It happens that typical constructions of the
tangent space of a manifold at a point have the property that tangent spaces at distinct points are disjoint, so that the set
$TQ$ can be taken to be literally the union of all tangent spaces. But when $Q$ is a submanifold of $\R^n$ one identifies
the tangent space $T_qQ$ at a point $q\in Q$ with a subspace of $\R^n$, and such subspaces of $\R^n$ are not disjoint
and thus one is forced to take $TQ$ to be the {\em disjoint\/} union $TQ=\bigcup_{q\in Q}\big(\{q\}\times T_qQ\big)$; in
that case, one {\em has to\/} write elements of the tangent bundle as ordered pairs (a point and a vector). So,
for reasons of uniformity, we find it convenient to do so also when $Q$ is a general manifold.

Definitions~\ref{thm:defvariation} and \ref{thm:defcriticalSL} can be readily adapted to the present context.
Given a smooth curve $q:[a,b]\to Q$, we define a {\em variation\/} of $q$ to be a family $(q_s)_{s\in I}$ of
smooth curves $q_s:[a,b]\to Q$, where $I\subset\R$ is an open interval with $0\in I$, $q_0$ equals $q$ and the map:
\[I\times[a,b]\ni(s,t)\longmapsto q_s(t)\in Q\]
is smooth. If the maps $s\mapsto q_s(a)$, $s\mapsto q_s(b)$ are constant, we say that the variation $(q_s)_{s\in I}$
has {\em fixed endpoints}. The {\em variational vector field\/} of a variation $(q_s)_{s\in I}$ is defined
again by formula \eqref{eq:defvarvecfield}, but now $v$ is a smooth map from $[a,b]$ to the tangent bundle $TQ$ such
that $v(t)\in T_{q(t)}Q$ for all $t\in[a,b]$, i.e., $v$ is a vector field along the curve $q$. Again, for a variation
with fixed endpoints, the variational vector field $v$ satisfies $v(a)=v(b)=0$. As before, we say that
$q$ is a {\em critical point\/} of $S_L$ if \eqref{eq:defvarvecfield} holds for every variation with fixed
endpoints $(q_s)_{s\in I}$ of $q$.

If we were to do serious infinite-dimensional calculus here, we would have to show how to turn $C^\infty\big([a,b],Q\big)$
into an infinite-dimensional manifold (it is a Fr\'echet manifold and, by replacing $C^\infty$ with $C^k$, for
some fixed finite $k$, it would be a Banach manifold), how to identify the space of smooth vector fields along
$q\in C^\infty\big([a,b],Q\big)$ with the tangent space $T_q C^\infty\big([a,b],Q\big)$, we would have to show
that $C^\infty_{q_aq_b}\big([a,b],Q\big)$ is a submanifold of $C^\infty\big([a,b],Q\big)$ whose tangent space at
a point $q$ consists of the smooth vector fields $v$ along $q$ with $v(a)=v(b)=0$ and we would have to show that the action
functional $S_L$ is smooth. All of that would require a considerable amount of work. The construction of the manifold
structure of $C^\infty\big([a,b],Q\big)$ is completely standard, but not completely straightforward. Luckily, we don't
have to worry about any of that here, as we won't be using any theorems from infinite-dimensional calculus.

If $L:\R\times TQ\to\R$ is a Lagrangian, $\widetilde Q$ is another differentiable manifold and $\varphi:Q\to\widetilde Q$
is a smooth diffeomorphism, then there is an obvious way to push $L$ to the manifold $\widetilde Q$ using $\varphi$.
Namely, we define a Lagrangian $L_\varphi:\R\times T\widetilde Q\to\R$ by requiring
that the diagram:
\[\xymatrix{%
\R\times TQ\ar[rr]^{\Id\times\dd\varphi}\ar[dr]_L&&\R\times T\widetilde Q\ar[dl]^{L_\varphi}\\
&\R}\]
be commutative, where $\Id$ denotes the identity map of $\R$ and $\dd\varphi:TQ\to T\widetilde Q$ denotes the differential
of the map $\varphi$ (the map whose restriction to the tangent space $T_qQ$ is the differential\footnote{%
Some authors prefer to use $T\varphi$ instead of $\dd\varphi$; that is indeed the natural notation if one thinks of $T$
as a functor that carries $Q$ to $TQ$. Also, if $\varphi$ is a map from $\R^m$ to $\R^n$
then there is some conflict of notation: in that context, $\dd\varphi$ usually refers to the map that associates to each
$x\in\R^m$ a linear map $\dd\varphi_x$ from $\R^m$ to $\R^n$, while the map $d\varphi$ from the tangent bundle $T\R^m\cong
\R^m\times\R^m$ to the tangent bundle $T\R^n=\R^n\times\R^n$ is not exactly that. Having said that, we will continue to
use $\dd\varphi$ instead of $T\varphi$.}
$\dd\varphi_q:T_qQ\to T_{\varphi(q)}\widetilde Q$, for all $q\in Q$). Obviously, if $q:[a,b]\to Q$ is a smooth
map and if $\tilde q=\varphi\circ q:[a,b]\to\widetilde Q$ is the curve obtained by pushing $q$ using $\varphi$, then:
\[L_\varphi\big(t,\tilde q(t),\dot{\tilde q}(t)\big)=L\big(t,q(t),\dot q(t)\big),\]
for all $t\in[a,b]$, so that:
\[S_L(q)=S_{L_\varphi}(\tilde q).\]
The equality $\tilde q_s=\varphi\circ q_s$ defines a bijection between variations with fixed endpoints
$(q_s)_{s\in I}$ of $q$ and variations with fixed endpoints $(\tilde q_s)_{s\in I}$ of $\tilde q$ and then,
since $S_L(q_s)=S_{L_\varphi}(\tilde q_s)$ for all $s\in I$, it follows that $q$ is a critical point
of $S_L$ if and only if $\tilde q$ is a critical point of $S_{L_\varphi}$. What we have just observed is
pretty obvious: any definition that makes sense for manifolds must give rise to a concept that is invariant
under diffeomorphisms (just like, say, any definition that makes sense for groups must give rise to a concept
that is invariant under group isomorphisms, and so on).

An important particular case of the construction of pushing a Lagrangian using a smooth diffeomorphism is this:
consider a local chart $\varphi:U\to\widetilde U$, where $U$ is an open subset of $Q$ and $\widetilde U$ is an open
subset of $\R^n$. We can push the Lagrangian $L$ (more precisely, the restriction of $L$ to the open subset
$\R\times TU$ of $\R\times TQ$) using $\varphi$ obtaining a Lagrangian:
\[L_\varphi:\R\times T\widetilde U=\R\times\widetilde U\times\R^n\longrightarrow\R\]
on the manifold $\widetilde U$. We call $L_\varphi$ the {\em representation of $L$ with respect to the chart
$\varphi$}. A smooth curve $q:[a,b]\to Q$ with $q\big([a,b]\big)\subset U$ is a critical point
of $S_L$ if and only if the curve $\tilde q=\varphi\circ q$ is a critical point of $S_{L_\varphi}$. Actually,
in order to get to that conclusion, there is a minor detail one should pay attention to:
$\varphi$ only induces a bijection between variations of $q$
{\em that stay inside of $U$\/} and variations of $\tilde q$ that stay inside of $\widetilde U$. But if the image
of $q$ is contained in $U$ then, for any variation $(q_s)_{s\in I}$ of $q$, we have that the image of $q_s$ is
contained in $U$ for $s$ in some open interval $I'\subset I$ containing the origin (see the argument that
appears in Remark~\ref{thm:remdomLopen}). Thus, for the definition of critical point, it doesn't make any difference
to consider only variations of the curve that stay inside some given open set containing the image of the curve.
By Theorem~\ref{thm:teoEL}, the curve $\tilde q$ is a critical point of $S_{L_\varphi}$ if and only
if\footnote{%
We denote by $\frac{\partial L_\varphi}{\partial q}$ and
$\frac{\partial L_\varphi}{\partial\dot q}$ the derivatives of
$L_\varphi$ with respect to its second and third variables,
respectively, even though $(t,q,\dot q)$ is the typical name of a
point of $\R\times TQ$, not of a point of the domain of
$L_\varphi$.}:
\begin{equation}\label{eq:ELchart}
\frac{\dd}{\dd t}\frac{\partial L_\varphi}{\partial\dot q}\big(t,\tilde q(t),\dot{\tilde q}(t)\big)=
\frac{\partial L_\varphi}{\partial q}\big(t,\tilde q(t),\dot{\tilde q}(t)\big),
\end{equation}
for all $t\in[a,b]$. When \eqref{eq:ELchart} holds for all $t\in[a,b]$ (where $\tilde q=\varphi\circ q$),
we say that the curve $q$ {\em satisfies the Euler--Lagrange
equation with respect to the chart $\varphi$}. We have shown that a smooth curve with image contained in the domain
of a chart is a critical point of $S_L$ if and only if it satisfies the Euler--Lagrange equation with respect to that chart.
It follows that, if $\varphi_1$, $\varphi_2$ are local
charts whose domains contain the image of a smooth curve $q$ then $q$ satisfies the Euler--Lagrange equation with respect to
$\varphi_1$ if and only if $q$ satisfies the Euler--Lagrange equation with respect to $\varphi_2$.
That could also be checked by a direct computation using the transition function $\varphi_2\circ\varphi_1^{-1}$
(see Exercises~\ref{exe:transfLag1} and \ref{exe:transfLag2}). But we
have just proven that fact and no such computation was necessary\footnote{%
Suppose that a curve $q$ satisfies the Euler--Lagrange equation with respect to a chart $\varphi_1$ just
at {\em one given instant\/} $t\in[a,b]$. Is it true that it satisfies the Euler--Lagrange equation with respect to a different
chart $\varphi_2$ at that same instant? The argument that we have just presented does not allow us to conclude that.
But that is indeed true and it follows from the result of Exercise~\ref{exe:transfLag2}.};
that is because being a solution of the Euler--Lagrange equation is equivalent to being a critical
point of the action functional and the latter notion is manifestly invariant under diffeomorphisms!

What we have done above is to prove a version of Theorem~\ref{thm:teoEL} for curves on manifolds, but only for
curves whose image is contained in the domain of a local chart! Now we have to handle arbitrary curves. Consider
a smooth curve $q:[a,b]\to Q$ and a local chart $\varphi:U\to\widetilde U$ (whose domain does not necessarily
contain the image of $q$). We will say that $q$ {\em satisfies the Euler--Lagrange equation with respect to $\varphi$\/}
if \eqref{eq:ELchart} holds for all $t\in q^{-1}(U)$, where:
\begin{equation}\label{eq:defqchart}
\tilde q=\varphi\circ q\vert_{q^{-1}(U)}.
\end{equation}
Notice that $q^{-1}(U)$ is an open subset of $[a,b]$ (open with respect to $[a,b]$, of course). When the image
of $q$ does not intercept $U$ then $q^{-1}(U)$ is the empty set and the condition that $q$ satisfy the Euler--Lagrange
equation with respect to $\varphi$ is vacuously satisfied. The right generalization of Theorem~\ref{thm:teoEL} is:
\begin{teo}\label{thm:ELmanifold}
Let $L:\R\times TQ\to\R$ be a Lagrangian and $q:[a,b]\to Q$ be a smooth curve. Then the following statements
are equivalent:
\begin{itemize}
\item[(a)] $q$ is a critical point of $S_L$;
\item[(b)] for every local chart $\varphi$ on $Q$, $q$ satisfies the Euler--Lagrange equation with respect to $\varphi$;
\item[(c)] there exists a family of local charts on $Q$, whose domains cover the image of $q$, such that $q$ satisfies
the Euler--Lagrange equation with respect to any chart belonging to that family.
\end{itemize}
\end{teo}
\begin{proof}
The implication (b)$\Rightarrow$(c) is obvious. The implications (a)$\Rightarrow$(b) and (c)$\Rightarrow$(a)
will be proven in Lemmas~\ref{thm:ELmanifoldab} and \ref{thm:ELmanifoldca} below.
\end{proof}

The fact that (a)$\Rightarrow$(b) is useful when one knows that $q$ is a critical point of $S_L$; then you can
choose whatever local chart you like and you know that $q$ satisfies the Euler--Lagrange equation with respect to that
chart. The fact that (c)$\Rightarrow$(a), on the other hand, is useful when you want to check that $q$ is a critical
point of $S_L$; in that case, you can choose your favorite set of charts (as long as their domains are able to cover
the image of $q$) for checking that $q$ satisfies the Euler--Lagrange equation with respect to them.

Let's get to the proof of (a)$\Rightarrow$(b).
\begin{lem}\label{thm:ELmanifoldab}
Condition (a) in the statement of Theorem~\ref{thm:ELmanifold} implies condition (b).
\end{lem}
\begin{proof}
Let $\varphi:U\to\widetilde U\subset\R^n$ be a local chart on $Q$. Defining $\tilde q$ as in \eqref{eq:defqchart},
it suffices to check that \eqref{eq:ELchart} holds for all $t\in[c,d]$, where $[c,d]$ is an arbitrary interval contained in
$q^{-1}(U)$. Because of Remark~\ref{thm:remcompsupport}, it suffices to prove that:
\begin{equation}\label{eq:ddstildeqs}
\left.\frac{\dd}{\dd s}\int_c^dL_\varphi\big(t,\tilde q_s(t),\dot{\tilde q}_s(t)\big)\,\dd t\,\right\vert_{s=0}=0,
\end{equation}
for any variation with {\em compact support}:
\begin{equation}\label{eq:vartildeq}
I\times[c,d]\ni(s,t)\longmapsto\tilde q_s(t)\in\R^n
\end{equation}
of the curve $\tilde q\vert_{[c,d]}$. By replacing $I$ with a smaller interval, we can assume that $\tilde q_s(t)$
belongs to $\widetilde U$, for all $s\in I$, $t\in[c,d]$. Set:
\[q_s(t)=\varphi^{-1}\big(\tilde q_s(t)\big),\]
for $s\in I$, $t\in[c,d]$ and $q_s(t)=q(t)$, for $s\in I$, $t\in[a,b]\setminus[c,d]$. The fact that
the variation \eqref{eq:vartildeq} has compact support implies easily that the map:
\[I\times[a,b]\ni(s,t)\longmapsto q_s(t)\in Q\]
is smooth and therefore it is a variation with fixed endpoints of $q$. Since $q$ is a critical point of $S_L$,
we have:
\begin{equation}\label{eq:ddsSLagain}
\left.\frac{\dd}{\dd s}\int_a^bL\big(t,q_s(t),\dot q_s(t)\big)\,\dd t\,\right\vert_{s=0}=0.
\end{equation}
The difference:
\[\int_a^bL\big(t,q_s(t),\dot q_s(t)\big)\,\dd t-\int_c^dL\big(t,q_s(t),\dot q_s(t)\big)\,\dd t\]
is independent of $s\in I$ and since:
\[\int_c^dL\big(t,q_s(t),\dot q_s(t)\big)\,\dd t=\int_c^dL_\varphi\big(t,\tilde q_s(t),\dot{\tilde q}_s(t)\big)\,\dd t\]
for all $s\in I$, it follows from \eqref{eq:ddsSLagain} that \eqref{eq:ddstildeqs} holds.
\end{proof}

The implication (c)$\Rightarrow$(a) is a bit trickier. The plan is to show that the derivative of $s\mapsto S_L(q_s)$
at $s=0$ is zero for variations $(q_s)_{s\in I}$ whose variational vector field $v$ has small support. Then we have
to show that this suffices for establishing that $q$ is a critical point of $S_L$.
That happens because the derivative of $s\mapsto S_L(q_s)$ defines a linear function
of the variational vector field $v$ (our next lemma) and because variational vector fields with small support
span all smooth vector fields $v$ along $q$ with $v(a)=v(b)=0$.
Curiously, we won't need to show that every smooth vector field $v$ along $q$
with $v(a)=v(b)=0$ is the variational vector field of some variation with fixed endpoints. That wouldn't be hard
to do (but it isn't as straightforward as when the manifold is $\R^n$); one could, for instance, use the exponential
map of some arbitrary Riemannian metric of $Q$ to define a variation of $q$ by $q_s(t)=\exp_{q(t)}\!\big(sv(t)\big)$
(or one could embed $Q$ in $\R^N$, using Whitney's theorem, then construct
the desired variation in $\R^N$, and then retract it back to $Q$ using a tubular neighborhood).
But we simply don't need to prove that.

\begin{lem}\label{thm:dSLlinear}
Let $q:[a,b]\to Q$ be a smooth curve. There exists a real valued linear map $D$ defined in the space
of all smooth vector fields along $q$ such that:
\[D(v)=\left.\frac{\dd}{\dd s}S_L(q_s)\right\vert_{s=0},\]
for any variation $(q_s)_{s\in I}$ of $q$, where $v$ denotes the variational vector field.
\end{lem}
\begin{proof}
If the image of $q$ is contained in the domain of a local chart $\varphi:U\to\widetilde U\subset\R^n$ then
one can simply define $D$ by:
\[D(v)=\int_a^b\frac{\partial L_\varphi}{\partial q}\big(t,\tilde q(t),\dot{\tilde q}(t)\big)\tilde v(t)+
\frac{\partial L_\varphi}{\partial\dot q}\big(t,\tilde q(t),\dot{\tilde q}(t)\big)\dot{\tilde v}(t)\,\dd t,\]
where $\tilde q=\varphi\circ q$, $\tilde v(t)=\dd\varphi_{q(t)}\big(v(t)\big)$ and $t\in[a,b]$.
For the general case, choose a partition
$a=t_0<t_1<\cdots<t_k=b$ of $[a,b]$ such that $q\big([t_i,t_{i+1}]\big)$ is contained in the domain of some chart,
for all $i$. Then (using the same symbol $S_L$ to denote the action functional for curves defined on the smaller intervals):
\[S_L(q_s)=\sum_{i=0}^{k-1}S_L(q_s\vert_{[t_i,t_{i+1}]}),\quad s\in I,\]
and if $D_i$ is a linear map that satisfies the thesis of the lemma for the restricted curve $q\vert_{[t_i,t_{i+1}]}$,
we simply define $D$ by setting:
\[D(v)=\sum_{i=0}^{k-1}D_i(v\vert_{[t_i,t_{i+1}]}).\qedhere\]
\end{proof}

Now we show that smooth vector fields with small support span all smooth vector fields that vanish at the endpoints.
\begin{lem}\label{thm:Usmallspan}
Let $q:[a,b]\to Q$ be a smooth curve and let $\mathcal U$ be an open cover of the interval $[a,b]$.
Every smooth vector field $v$ along $q$ with $v(a)=v(b)=0$ can be written as a finite sum $\sum_{i=1}^kv_i$
of smooth vector fields $v_i$ along $q$ with $v_i(a)=v_i(b)=0$ and such that the support of $v_i$ is contained
in some element of $\mathcal U$.
\end{lem}
\begin{proof}
Replace $\mathcal U$ with a finite subcover $\{U_1,\ldots,U_k\}$ and consider a smooth partition of unit subordinated to it, i.e.,
smooth maps $\xi_i:[a,b]\to\R$, $i=1,\ldots,k$, with $\sum_{i=1}^k\xi_i=1$, such that the support of $\xi_i$ is
contained in $U_i$, for all $i$. Set $v_i=v\xi_i$, $i=1,\ldots,k$.
\end{proof}

Now we check that, under condition (c) in the statement of Theorem~\ref{thm:ELmanifold}, the derivative
of $s\mapsto S_L(q_s)$ at $s=0$ vanishes when the variational vector field has small support.
\begin{lem}\label{thm:dSsmallsupport}
Let $q:[a,b]\to Q$ be a smooth curve and let $\varphi:U\to\widetilde U$ be a local chart such that
$q$ satisfies the Euler--Lagrange equation with respect to $\varphi$. If $v$ is a smooth vector field along
$q$ such that $v(a)=v(b)=0$ and such that the support of $v$ is contained in an interval $[c,d]$ contained in $q^{-1}(U)$
then $D(v)=0$, where $D$ is a linear map satisfying the condition in the statement of Lemma~\ref{thm:dSLlinear}.
\end{lem}
\begin{proof}
We define a variation $(q_s)_{s\in I}$ of $q$ as follows: if $t$ is in $[a,b]\setminus[c,d]$, we set
$q_s(t)=q(t)$, for all $s\in I$. For $t\in[c,d]$, we set:
\begin{equation}\label{eq:defqsvarphilinear}
q_s(t)=\varphi^{-1}\big(\tilde q(t)+s\tilde v(t)\big),
\end{equation}
for all $s\in I$, where $\tilde q(t)=\varphi\big(q(t)\big)$, $\tilde v(t)=\dd\varphi_{q(t)}\big(v(t)\big)$ and $I$ is
chosen small enough so that $\tilde q(t)+s\tilde v(t)$ is in $\widetilde U$, for all $s\in I$, $t\in[c,d]$.
The map $(s,t)\mapsto q_s(t)$ is smooth in $I\times[a,b]$ because equality \eqref{eq:defqsvarphilinear} actually
holds for $t$ in the neighborhood $q^{-1}(U)$ of $[c,d]$. Thus $(q_s)_{s\in I}$ is a variation
of $q$ with variational vector field $v$, so that:
\[D(v)=\left.\frac{\dd}{\dd s}S_L(q_s)\right\vert_{s=0}.\]
Arguing as in the proof of Lemma~\ref{thm:ELmanifoldab}, we see that:
\begin{equation}\label{eq:ddsabcd}
\left.\frac{\dd}{\dd s}S_L(q_s)\right\vert_{s=0}=\left.\frac{\dd}{\dd s}
\int_c^dL_\varphi\big(t,\tilde q_s(t),\dot{\tilde q}_s(t)\big)\,\dd t\,\right\vert_{s=0},
\end{equation}
where $\tilde q_s(t)=\varphi(q_s(t)\big)=\tilde q(t)+s\tilde v(t)$, for $t\in[c,d]$, $s\in I$. Since:
\[I\times[c,d]\ni(s,t)\longmapsto\tilde q_s(t)\]
is a variation with fixed endpoints of $\tilde q\vert_{[c,d]}$ and since $\tilde q\vert_{[c,d]}$ is a solution
of the Euler--Lagrange equation \eqref{eq:ELchart}, it follows that the righthand side of \eqref{eq:ddsabcd} vanishes.
This concludes the proof.
\end{proof}

\begin{lem}\label{thm:ELmanifoldca}
Condition (c) in the statement of Theorem~\ref{thm:ELmanifold} implies condition (a).
\end{lem}
\begin{proof}
Let $\mathcal U$ be the set of all intervals $J$ that are open in $[a,b]$ and such that $q(J)$ is contained
in the domain of one of the charts belonging to the family whose existence is assumed in condition (c). By Lemma~\ref{thm:dSsmallsupport},
$D(v)=0$ if $v(a)=v(b)=0$ and the support of $v$ is contained in some $J\in\mathcal U$ (obviously, if the support of $v$ is contained
in $J$ then it is contained in some closed interval $[c,d]$ contained in $J$). By Lemma~\ref{thm:Usmallspan}
and by the linearity of $D$, it follows that $D(v)=0$ for any $v$ with $v(a)=v(b)=0$,
proving that $q$ is a critical point of $S_L$.
\end{proof}

Assume now that the manifold $Q$ is a submanifold of $\R^n$ (we have been using $n$ to denote the dimension
of $Q$, but, obviously, now we are not) and assume that we have a Lagrangian $L:\R\times\R^n\times\R^n\to\R$ on $\R^n$.
Since $Q$ is a submanifold of $\R^n$, $TQ$ is naturally identified with a submanifold of $T\R^n=\R^n\times\R^n$ and
thus we can restrict $L$ to $\R\times TQ$. What are the critical points of the action functional of the restriction
of $L$? The following proposition answers that.

\begin{prop}\label{thm:propLagrestr}
Let $L$ be a Lagrangian on $\R^n$, $Q$ be a submanifold of $\R^n$ and $L^{\mathrm{restr}}$ denote the Lagrangian
on $Q$ obtained by restricting $L$ to $\R\times TQ$. A smooth curve $q:[a,b]\to Q$ is a critical point of
the action functional $S_{L^{\mathrm{restr}}}$ if and only if:
\[\frac{\dd}{\dd t}\frac{\partial L}{\partial\dot q}\big(t,q(t),\dot q(t)\big)-
\frac{\partial L}{\partial q}\big(t,q(t),\dot q(t)\big)\in(T_{q(t)}Q)^\anul,\]
for all $t\in[a,b]$, where $(T_{q(t)}Q)^\anul\subset{\R^n}^*$ denotes the annihilator of the tangent
space $T_{q(t)}Q\subset\R^n$ (if one identifies the dual space ${\R^n}^*$ with $\R^n$ in the usual way then
then annihilator $(T_{q(t)}Q)^\anul$ is identified with the orthogonal complement $(T_{q(t)}Q)^\perp$).
\end{prop}
\begin{proof}
We leave details to the reader as an exercise. The hint is {\em don't\/} (really, don't!) use
Theorem~\ref{thm:ELmanifold} and the representation of $L^{\mathrm{restr}}$ with respect to a local chart on $Q$.
It is much easier to use directly the definition of critical point, observing that
$S_{L^{\mathrm{restr}}}(q_s)=S_L(q_s)$, for any variation $(q_s)_{s\in I}$ of $q$ in $Q$. The derivative
$\frac{\dd}{\dd s}S_L(q_s)\vert_{s=0}$ can be computed exactly as in Section~\ref{sec:introvariations}. The only
difference from what was done there is that now we are going obtain that $q$ is a critical point of $S_{L^{\mathrm{restr}}}$
if and only if the righthand side of \eqref{eq:dSLcomputed} vanishes for smooth maps $v:[a,b]\to\R^n$ such that
$v(a)=v(b)=0$ {\em and\/} $v(t)\in T_{q(t)}Q$, for all $t\in[a,b]$. You will then need a generalization of
the Fundamental Lemma of the Calculus of Variations (which is stated as Exercise~\ref{exe:genfundlemcalcvar})
to conclude the proof. If you prefer not to use the fact that any $v$ is the variational vector field of a variation
of $q$ in $Q$, notice that
in order to conclude the proof it is enough to consider vector fields $v$ along $q$ having small support
(contained in $q^{-1}(U)$, where $U\subset Q$ is the domain of a local chart) and
for those one straightforwardly constructs a variation of $q$ using a local chart on $Q$.
\end{proof}

\subsection{Back to Classical Mechanics. Constraints.}\label{sub:constraints}
Are the variational problems on manifolds discussed in
this section of any use to Classical Mechanics? They are. This stuff is useful when we work on a problem in which
there is something which constrains the motion of the particles: a track from which a particle cannot get away,
a string or a rigid bar connecting two particles, etc. Of course, at the fundamental level, there are no such constraints;
tracks, strings and bars are themselves made out of particles. The problem of $n$ constrained particles which we are
going to discuss in this subsection would, at the fundamental level, be a problem of $N>n$ free particles,
interacting through the fundamental forces of Classical Mechanics (the electrical and the gravitational forces\footnote{%
As noted earlier, this cannot be taken too seriously, as we know that Classical Mechanics is not a fundamental theory.}).

Let us consider a system of $n$ particles subject to forces $F=(F_1,\ldots,F_n)$ that admit a potential $V$.
We then know that their
trajectories satisfy the Euler--Lagrange equation of the Lagrangian $L$ defined in \eqref{eq:Lmechanics}. Now,
assume that apart from the forces encoded in the potential $V$, there are also additional forces acting on the
particles, in such a way that the configuration $\big(q_1(t),\ldots,q_n(t)\big)\in(\R^3)^n$ must stay
inside a subset $Q$ of $(\R^3)^n$, for all $t$. Such additional forces will be called the {\em forces from the constraint}.
We will assume that the subset $Q$ is a smooth submanifold of $(\R^3)^n$.

The type of constraint that we are considering
here are constraints on the {\em positions\/} of the particles (known as {\em holonomic\/} constraints).
One could also consider constraints formulated in terms of the {\em velocities\/} of the particles that cannot
be reduced to constraints on the positions alone ({\em non holonomic\/} constraints); we are not going to consider
those in this course. The constraints under consideration in this subsection are also being assumed to be independent of time
(otherwise, we should consider a submanifold $Q_t$ of $(\R^3)^n$, for each $t$).

Let us try to take the Lagrangian $L$ defined in \eqref{eq:Lmechanics} and consider its restriction to
a Lagrangian $L^{\mathrm{cons}}$ on the submanifold $Q$. What are the critical points of the action functional?
The answer is obtained from Proposition~\ref{thm:propLagrestr}. A curve $q=(q_1,\ldots,q_n):[a,b]\to Q\subset(\R^3)^n$
is a critical point of the action functional $S_{L^{\mathrm{cons}}}$ if and only if the vector:
\begin{multline}\label{eq:vecR}
\Big(m_j\frac{\dd^2q_j}{\dd t^2}(t)+\nabla_{q_j}V\big(t,q(t)\big)\!\Big)_{\!j=1,\ldots,n}\\[5pt]
=\Big(m_j\frac{\dd^2q_j}{\dd t^2}(t)-F_j\big(t,q(t)\big)\!\Big)_{\!j=1,\ldots,n}\in(\R^3)^n
\end{multline}
is orthogonal\footnote{%
Orthogonality here means orthogonality with respect to the standard inner product of $(\R^3)^n$. We observe that sometimes
it is useful to consider an inner product on $(\R^3)^n$ that is scaled by the masses of the particles. See
Exercise~\ref{exe:massinnprod} for details.} to the tangent space $T_{q(t)}Q\subset(\R^3)^n$, for all $t\in[a,b]$.
We can reformulate this by saying that:
\begin{multline}\label{eq:vecR2}
m_j\frac{\dd^2q_j}{\dd t^2}(t)=-\nabla_{q_j}V\big(t,q(t)\big)+R_j(t)\\
=F_j\big(t,q(t)\big)+R_j(t),\quad j=1,\ldots,n,
\end{multline}
for all $t\in[a,b]$, for some map $R=(R_1,\ldots,R_n):[a,b]\to(\R^3)^n$ such that
$R(t)\in(\R^3)^n$ is orthogonal to the tangent space $T_{q(t)}Q$, for all $t\in[a,b]$.
If $q=(q_1,\ldots,q_n)$ are the actual trajectories of the particles, then \eqref{eq:vecR2} holds if and only if
the vector $R_j(t)\in\R^3$ is the force exerted upon the $j$-th particle by the constraint at the instant $t$.

Of course, there is no {\em logical\/} reason why the critical points of the action functional $S_{L^{\mathrm{cons}}}$
should correspond to the trajectories of $n$ particles subject to the forces encoded in the potential $V$ and to the forces from
the constraint. That isn't a logical consequence of the dynamics of Classical Mechanics, as presented in Section~\ref{sec:ontologydynamics}.
What we have shown is that it is true that the critical points of the action functional $S_{L^{\mathrm{cons}}}$ correspond to the trajectories
of the $n$ constrained particles if and only if the forces $R_j(t)$ exerted by the constraint upon the particles
constitute a vector $R(t)=\big(R_1(t),\ldots,R_n(t)\big)$ that is orthogonal to the tangent space $T_{q(t)}Q$, for all $t$.
Is the orthogonality between $R(t)$ and $T_{q(t)}Q$ a reasonable assumption? That can only be judged within the context
of a specific example. When $n=1$, so that the manifold $Q$ corresponds to a curve or a surface inside physical space
from which the particle cannot get away, then the assumption of orthogonality between the vector $R(t)$ (in $\R^3$)
and the tangent space $T_{q(t)}Q$ is easy to be physically interpreted. It means that, whatever keeps the particle
inside $Q$, does that by exerting upon the particle a force that is orthogonal to $Q$; such condition can usually be
understood as the absence of {\em friction\/} between the particle and the surface. For the general case, the orthogonality
between $R(t)$ and the vectors in $T_{q(t)}Q$ is an orthogonality between vectors in {\em configuration space\/}
$(\R^3)^n$, not an orthogonality between vectors in physical space, so it is hard to understand the physical meaning
of such orthogonality condition. In Exercises~\ref{exe:doublependulum} and \ref{exe:doublespherependulum}
we ask the reader to work with concrete examples (the {\em double pendulum\/} and the {\em double spherical pendulum})
and to discover what the orthogonality condition means in those cases.

\end{section}

\begin{section}{Hamiltonian formalism}
\label{sec:Hamformalism}

The ordinary differential equation \eqref{eq:dynamClMec} defining the dynamics of Classical Mechanics is of second order. Everyone knows
that a second order ordinary differential equation over a certain space can be reformulated in terms of a first order
ordinary differential equation over a new space whose dimension is the double of the dimension of the original space.
Such reformulation has some advantages. For an ordinary differential equation of first order the initial condition
that determines the solution is just the value of the solution at one given initial instant and thus one can see
the equation as defining a family of maps (the flow) from a space to itself.
One can then ask questions such as ``is there any interesting structure that is invariant by that flow?''. One
obvious way of reformulating the differential equation that defines the dynamics of Classical Mechanics as a first order
equation is to introduce a new independent variable for $\dot q(t)$. It turns out that this is not
the best idea; a small modification of that idea yields much better results.

\begin{defin}\label{thm:defmomentum}
The {\em momentum\/} of the $j$-th particle at time $t\in\R$ is defined by:
\begin{equation}\label{eq:defmomentum}
p_j(t)=m_j\dot q_j(t).
\end{equation}
\end{defin}
Notice that (by \eqref{eq:dynamClMec}) the derivative of $p_j$ at an instant $t$ is equal to the total force
$F_j$ acting upon the $j$-th particle at that instant. Assuming \eqref{eq:totalforcezero} it follows immediately that the
{\em total momentum}:
\[\sum_{j=1}^np_j(t)\]
is independent of $t$, i.e., it is {\em conserved}. Condition \eqref{eq:totalforcezero} holds if we don't have
external forces (either we are dealing with the entire universe or with a system which is almost isolated, so that
external forces are neglected) and if the force laws satisfy Newton's law of reciprocal actions \eqref{eq:actionreaction}
(which holds for the fundamental forces of Classical Mechanics, i.e., the gravitational and the electrical forces).
In the presence of external forces (and assuming Newton's law of reciprocal actions) the derivative
of the total momentum is equal to the sum of the external forces; therefore, even when the total momentum
of a system is not conserved due to external forces, it is not hard to keep track of its evolution, since we can
ignore the internal forces which are often the ones which are complicated to handle (imagine keeping track of the
internal forces among all particles of some macroscopic system!). Later on we will take a closer look at the subject of
conservation laws and we will understand the relationship between those and {\em symmetry};
for now, that is all that must be said about the conservation of the total momentum. We are just trying to give
some motivation for Definition~\ref{thm:defmomentum}, so that it doesn't look like we have given a name for some
arbitrary formula involving the particle trajectories $q_j$.

\medskip

The maps $q_j:\R\to\R^3$, $p_j:\R\to\R^3$, $j=1,\ldots,n$, define a curve:
\[(q,p)=(q_1,\ldots,q_n,p_1,\ldots,p_n):\R\longrightarrow(\R^3)^n\times(\R^3)^n.\]
If the maps $q_j$ satisfy \eqref{eq:dynamClMec} and the maps $p_j$ are defined by \eqref{eq:defmomentum}
then the curve $(q,p)$ satisfies the first order differential equation:
\begin{equation}\label{eq:Hamilton}
\frac{\dd q_j}{\dd t}(t)=\frac{p_j(t)}{m_j},\quad\frac{\dd p_j}{\dd t}(t)=F_j,\qquad j=1,\ldots,n,
\end{equation}
where, in the second equation, $F_j$ is understood to be evaluated at the point:
\[\big(t,q_1(t),\ldots,q_n(t),\tfrac1{m_1}\,p_1(t),\ldots,\tfrac1{m_n}\,p_n(t)\big).\]
Conversely, if the curve $(q,p)$ satisfies \eqref{eq:Hamilton} then the maps $q_j$ satisfy \eqref{eq:dynamClMec}
and the maps $p_j$ are given by \eqref{eq:defmomentum}. Let us now assume that the force $F$
admits a potential $V:\Dom(V)\subset\R\times(\R^3)^n\to\R$. Equation \eqref{eq:Hamilton} becomes:
\begin{equation}\label{eq:Hamilton2}
\frac{\dd q_j}{\dd t}(t)=\frac{p_j(t)}{m_j},\quad\frac{\dd p_j}{\dd t}(t)=-\nabla_{q_j}V\big(t,q(t)\big),\qquad j=1,\ldots,n.
\end{equation}
Recall from Definition~\ref{thm:defenergies} the notions of kinetic and potential energy. The {\em total energy\/}
(also known as {\em mechanical energy\/} or just {\em energy}) at the instant $t$ is defined to be the sum of
the total kinetic energy with the potential energy:
\begin{equation}\label{eq:totalenergy}
\sum_{j=1}^n\frac12\,m_j\Vert\dot q_j(t)\Vert^2+V\big(t,q(t)\big).
\end{equation}
The kinetic energy of the $j$-th particle can be written in terms of its momentum $p_j$:
\[\frac12\,m_j\Vert\dot q_j(t)\Vert^2=\frac{\Vert p_j(t)\Vert^2}{2m_j}\]
and so the total energy is equal to:
\[\sum_{j=1}^n\frac{\Vert p_j(t)\Vert^2}{2m_j}+V\big(t,q(t)\big).\]
Define a map $H:\Dom(V)\times(\R^3)^n\subset\R\times(\R^3)^n\times(\R^3)^n\to\R$ by setting\footnote{%
We are using the symbols $q$, $p$ for names of curves $t\mapsto q(t)$, $t\mapsto p(t)$ and also for names of
points $q,p\in(\R^3)^n$. This type of notation abuse is very convenient and it hardly generates any misunderstandings.}:
\begin{equation}\label{eq:defH}
H(t,q,p)=H(t,q_1,\ldots,q_n,p_1,\ldots,p_n)=\sum_{j=1}^n\frac{\Vert p_j\Vert^2}{2m_j}+V(t,q),
\end{equation}
for all $(t,q)\in\Dom(V)\subset\R\times(\R^3)^n$ and all $p\in(\R^3)^n$. The total
energy at the instant $t$ is then equal to $H\big(t,q(t),p(t)\big)$.

\begin{defin}
The map $H$ defined in \eqref{eq:defH} is called the {\em Hamiltonian\/} of Classical Mechanics.
\end{defin}

Notice that:
\begin{equation}\label{eq:partialH}
\frac{\partial H}{\partial q_j}(t,q,p)=\nabla_{q_j}V(t,q),\quad
\frac{\partial H}{\partial p_j}(t,q,p)=\frac{p_j}{m_j},\quad j=1,\ldots,n,
\end{equation}
where we have used the standard identification between $\R^3$ and the dual space ${\R^3}^*$ (notice that
the lefthand sides of the equalities in \eqref{eq:partialH} are elements of ${\R^3}^*$, while the righthand
sides are elements of $\R^3$). It follows from
\eqref{eq:partialH} that equation \eqref{eq:Hamilton2} is equivalent to:
\begin{equation}\label{eq:Hamilton3}
\frac{\dd q}{\dd t}(t)=\frac{\partial H}{\partial p}\big(t,q(t),p(t)\big),\quad
\frac{\dd p}{\dd t}(t)=-\frac{\partial H}{\partial q}\big(t,q(t),p(t)\big),
\end{equation}
where now we have used the standard identification between $(\R^3)^n$ and the dual space
${(\R^3)^n}^*$ (the lefthand sides of the equalities in \eqref{eq:Hamilton3} are elements of $(\R^3)^n$,
while the righthand sides are elements of the dual space ${(\R^3)^n}^*$).
In this section we will continue to use such identification.
Later, when we work with manifolds, we will discover that
the more appropriate domain for $H$ is an open subset of $\R\times(\R^3)^n\times{(\R^3)^n}^*$ and that $p(t)$ should be
regarded as an element of ${(\R^3)^n}^*$. Under such conditions, both sides of the first equation in \eqref{eq:Hamilton3}
become elements of $(\R^3)^n$ (actually, the righthand side is an element of the {\em bidual\/} of $(\R^3)^n$, which
is naturally identified with $(\R^3)^n$) and both sides of the second equation in \eqref{eq:Hamilton3} become
elements of ${(\R^3)^n}^*$. For now, insisting on not identifying $(\R^3)^n$ with ${(\R^3)^n}^*$ would just be annoying.

\begin{defin}
Equations \eqref{eq:Hamilton3} are called {\em Hamilton's equations}.
The space $(\R^3)^n\times(\R^3)^n$ on which the curve $t\mapsto\big(q(t),p(t)\big)$ takes values is called the
{\em phase space}.
\end{defin}

Assume that $H(t,q,p)$ does not depend on $t$, so that we write $H(q,p)$ instead of $H(t,q,p)$
(if $H$ is of the form \eqref{eq:defH}, this happens if and only if the potential $V(t,q)$ does not depend on $t$).
If $t\mapsto\big(q(t),p(t)\big)$ is a solution of Hamilton's equations \eqref{eq:Hamilton3} then:
\begin{align}
\notag\frac{\dd}{\dd t}\,H\big(&q(t),p(t)\big)=\frac{\partial H}{\partial q}\big(q(t),p(t)\big)\frac{\dd q}{\dd t}(t)
+\frac{\partial H}{\partial p}\big(q(t),p(t)\big)\frac{\dd p}{\dd t}(t)\\[5pt]
\label{eq:difzero}&=\frac{\partial H}{\partial q}\big(q(t),p(t)\big)\frac{\partial H}{\partial p}\big(q(t),p(t)\big)
-\frac{\partial H}{\partial p}\big(q(t),p(t)\big)\frac{\partial H}{\partial q}\big(q(t),p(t)\big).
\end{align}
In the formula above, the expressions:
\[\frac{\partial H}{\partial q}\big(q(t),p(t)\big)\frac{\partial H}{\partial p}\big(q(t),p(t)\big),\quad
\frac{\partial H}{\partial p}\big(q(t),p(t)\big)\frac{\partial H}{\partial q}\big(q(t),p(t)\big)\]
denote the evaluation of an element of ${(\R^3)^n}^*$ at an element of $(\R^3)^n$. Under the identification of ${(\R^3)^n}^*$
with $(\R^3)^n$, both expressions become the (standard) inner product between the vectors
$\frac{\partial H}{\partial q}\big(q(t),p(t)\big)$, $\frac{\partial H}{\partial p}\big(q(t),p(t)\big)$ of
$(\R^3)^n$. Hence the difference \eqref{eq:difzero} vanishes and:
\[\frac{\dd}{\dd t}\,H\big(q(t),p(t)\big)=0.\]
We have proven:
\begin{prop}\label{thm:propHconstant}
If $H(t,q,p)$ does not depend on $t$ then $H$ is constant along the solutions of Hamilton's equations \eqref{eq:Hamilton3}.\qed
\end{prop}
In other words, the Hamiltonian $H$ (when it does not depend on time) is a {\em first integral\/} for Hamilton's equations. It follows that,
when the potential energy does not depend on time, the {\em total energy is conserved\/} in Classical Mechanics. Of course,
we could have discovered the conservation of total energy simply by computing the derivative of \eqref{eq:totalenergy},
but it is interesting to see that such conservation law is a particular case of the more general fact that
$H$ is a first integral of Hamilton's equations.

Hamilton's equations state that the curve $t\mapsto\big(q(t),p(t)\big)$ is an integral
curve of the time-dependent vector field over $(\R^3)^n\times(\R^3)^n$ defined by:
\begin{equation}\label{eq:Hamvec}
(t,q,p)\longmapsto\Big(\frac{\partial H}{\partial p}(t,q,p),-\frac{\partial H}{\partial q}(t,q,p)\Big)
\in(\R^3)^n\times(\R^3)^n.
\end{equation}
This vector field is {\em almost\/} the gradient of $H$, which is given by:
\[\nabla_{(q,p)}H(t,q,p)=\Big(\frac{\partial H}{\partial q}(t,q,p),\frac{\partial H}{\partial p}(t,q,p)\Big)
\in(\R^3)^n\times(\R^3)^n.\]
However, \eqref{eq:Hamvec} differs from the gradient by an order switch and by a minus sign.
We now introduce a modified notion of gradient that
yields exactly \eqref{eq:Hamvec}. This is done as follows: recall that the gradient of a map is the vector that,
when contracted with a given inner product, yields the differential of that map; in other words, the gradient is
the vector that corresponds to the differential via the isomorphism between the space and the dual space induced by
an inner product. For example, if $\langle\cdot,\cdot\rangle$
denotes the standard inner product of $(\R^3)^n$, then:
\[\langle\nabla_{(q,p)}H(t,q,p),(x,y)\rangle=\partial_{(q,p)}H(t,q,p)(x,y)
=\frac{\partial H}{\partial q}(t,q,p)x+\frac{\partial H}{\partial p}(t,q,p)y,\]
for all $(x,y)\in(\R^3)^n\times(\R^3)^n$. We obtain a new notion of gradient simply by replacing the inner product
with something else. We define an {\em anti-symmetric\/} bilinear map $\omega$ over the vector space $(\R^3)^n\times(\R^3)^n$ by setting:
\begin{multline}\label{eq:omegaphase}
\omega\big((x,y),(\bar x,\bar y)\big)=\langle x,\bar y\rangle-\langle\bar x,y\rangle,\\
(x,y),(\bar x,\bar y)\in(\R^3)^n\times(\R^3)^n.
\end{multline}
The bilinear form $\omega$ induces an isomorphism between $(\R^3)^n\times(\R^3)^n$ and its dual space
$\big((\R^3)^n\times(\R^3)^n\big)^*$ and we can therefore consider the vector:
\[\vec H(t,q,p)\in(\R^3)^n\times(\R^3)^n\]
that corresponds to the differential $\partial_{(q,p)}H(t,q,p)\in\big((\R^3)^n\times(\R^3)^n\big)^*$ via such isomorphism.
More explicitly, given $(t,q,p)$ in the domain of $H$, it is easily checked that there exists a unique vector
$\vec H(t,q,p)$ in $(\R^3)^n\times(\R^3)^n$ that satisfies the equality:
\[\omega\big(\vec H(t,q,p),(x,y)\big)=\partial_{(q,p)}H(t,q,p)(x,y)
=\frac{\partial H}{\partial q}(t,q,p)x+\frac{\partial H}{\partial p}(t,q,p)y,\]
for all $(x,y)\in(\R^3)^n\times(\R^3)^n$ and that such vector is given by:
\[\vec H(t,q,p)=\Big(\frac{\partial H}{\partial p}(t,q,p),-\frac{\partial H}{\partial q}(t,q,p)\Big).\]
Thus $\vec H$ is precisely the time-dependent vector field \eqref{eq:Hamvec} whose integral curves are the solutions
$t\mapsto\big(q(t),p(t)\big)$ of Hamilton's equations!

\begin{defin}
The bilinear form $\omega$ defined in \eqref{eq:omegaphase} is called
the {\em canonical symplectic form\/} of the phase space $(\R^3)^n\times(\R^3)^n$ and the vector field $\vec H$ is
called the {\em symplectic gradient\/} of the map $H$.
\end{defin}

In this section we have taken the first steps towards presenting what is called the {\em Hamiltonian formalism}.
Here is an overview of what are going to be our next steps.
Hamilton's equations can be formulated in the context of manifolds and for that purpose the manifold is required to be
endowed with what will be called a {\em symplectic form}. We will study Hamiltonians and symplectic forms on manifolds in Section~\ref{sec:symplmanifold},
but first we have to know what is meant by a symplectic form over a vector space, which is the subject of
Section~\ref{sec:symplvecspace}. After Section~\ref{sec:symplmanifold}, we will see that given a Lagrangian
$L$ over a manifold $Q$ (so that the domain of $L$ is an open subset of $\R\times TQ$) then, under certain conditions
(known as {\em hyper-regularity}), we can construct a map $H$ (the {\em Legendre transform\/} of $L$), called the {\em Hamiltonian\/}
corresponding to $L$, defined over an open subset of $\R\times TQ^*$, where $TQ^*$ denotes the cotangent bundle of $Q$.
When the Lagrangian $L$ is the difference between total kinetic energy and potential energy (so that --- assuming that the force exerted by the
constraint is normal to $Q$ --- the Euler--Lagrange
equation is the dynamical equation of Classical Mechanics), the corresponding Hamiltonian $H$ will be exactly the sum
of the total kinetic energy with the potential energy (with the total kinetic energy rewritten in terms of new variables
$p$). We will see that a cotangent bundle $TQ^*$ carries a canonical symplectic form, so that it makes sense to talk
about Hamilton's equations associated to a Hamiltonian $H$ over a cotangent bundle.
Moreover, we will see that when $H$ is constructed from $L$, the solutions of Hamilton's equations
associated to $H$ are the same as the solutions of the Euler--Lagrange equation associated to $L$;
more precisely, $q$ is a solution of the Euler--Lagrange equation associated to $L$ if and only if
$(q,p)$ is a solution of Hamilton's equations associated to $H$, for some curve $p$. There will be a explicit
description of such curve $p$. When $Q=(\R^3)^n$ and $L$ is the difference between total kinetic energy and potential energy,
$p$ agrees with the momentum \eqref{eq:defmomentum}.

\end{section}

\begin{section}{Symplectic forms over vector spaces}
\label{sec:symplvecspace}

In Section~\ref{sec:Hamformalism} we have defined the canonical symplectic form of the phase space
$(\R^3)^n\times(\R^3)^n$. Let us now explain what we mean by a symplectic form over a vector space and let us
prove some elementary results about those. This section is just a bunch of results from
elementary linear algebra (which are usually not taught in standard elementary linear algebra courses).
In Subsection~\ref{sub:symplvolume} we show how to construct a volume form from a symplectic form using the exterior product and that requires
a bit of multilinear algebra. For the reader's convenience, we have included a short review of multilinear algebra in the appendix
(Section~\ref{sec:quickmultilinear}).

\begin{defin}
Let $V$ be a real finite-dimensional vector space. A {\em symplectic form\/} over $V$ is an anti-symmetric bilinear
form $\omega:V\times V\to\R$ that is also {\em non degenerate}, i.e., given $v\in V$, if $\omega(v,w)=0$ for all
$w\in V$ then $v=0$. The pair $(V,\omega)$ is called a {\em symplectic vector space\/} (or just a {\em symplectic space}).
\end{defin}
We have restricted our definition to the context of finite-dimensional vector spaces over the field of real numbers.
Of course, the same definition makes sense for arbitrary vector spaces over any scalar field, but real finite-dimensional
vector spaces are sufficient for our purposes. Notice that the condition that $\omega$ be non degenerate is equivalent
to the condition that the linear map canonically associated to $\omega$:
\begin{equation}\label{eq:omegalinear}
V\ni v\longmapsto\omega(v,\cdot)\in V^*
\end{equation}
be injective. Since $V$ is finite-dimensional, this is the same as requiring that \eqref{eq:omegalinear} be an isomorphism.
In the infinite-dimensional case, one should talk about {\em weak non degeneracy\/} (when \eqref{eq:omegalinear}
is injective) and {\em strong non degeneracy\/} (when \eqref{eq:omegalinear} is an isomorphism\footnote{%
Of course, in the infinite-dimensional case, $V$ would normally be assumed to be a {\em topological\/} vector space
and one would consider only its topological dual space, consisting of {\em continuous\/} linear functionals over $V$.}).
But let us focus on the finite-dimensional case.
Just like inner products, symplectic forms induce an isomorphism between the space $V$ and its dual $V^*$, so one
can talk about the vector $v$ that {\em represents\/} a linear functional $\alpha\in V^*$ with respect to $\omega$,
i.e., $v\in V$ is the only vector such that $\omega(v,\cdot)=\alpha$. This is the crucial property that allowed
us to define the symplectic gradient $\vec H$ in Section~\ref{sec:Hamformalism}.
Notice that in the case of an inner product $\langle\cdot,\cdot\rangle$, the linear
functionals $\langle v,\cdot\rangle$ and $\langle\cdot,v\rangle$ are equal, while if $\omega$ is a symplectic
form then $\omega(v,\cdot)=-\omega(\cdot,v)$, so one must be more careful and pay attention to our convention of putting
the $v$ in the {\em first\/} variable of $\omega$ in the definition \eqref{eq:omegalinear}
of the isomorphism between $V$ and its dual space $V^*$.

\begin{example}\label{exa:cansymplform}
For any natural number $n$, we define the {\em canonical symplectic form\/} of $\R^{2n}=\R^n\times\R^n$ by:
\[\omega_0\big((x,y),(\bar x,\bar y)\big)=\langle x,\bar y\rangle-\langle\bar x,y\rangle,\quad x,y,\bar x,\bar y\in\R^n,\]
where $\langle\cdot,\cdot\rangle$ denotes the canonical inner product of $\R^n$. It is a simple exercise
to check that $\omega_0$ is indeed anti-symmetric and non degenerate. We also define
a {\em canonical symplectic form\/} for the space $\R^n\times{\R^n}^*$ (again denoted by $\omega_0$), by setting:
\[\omega_0\big((x,\alpha),(\bar x,\bar\alpha)\big)=\bar\alpha(x)-\alpha(\bar x),\quad x,\bar x\in\R^n,\ \alpha,\bar\alpha\in{\R^n}^*.\]
Notice that when using $\R^n\times{\R^n}^*$ instead of $\R^n\times\R^n$ one does not need an inner product to define the symplectic
form!
\end{example}

Not surprisingly, there is a notion of isomorphism for symplectic spaces.
\begin{defin}
Given symplectic spaces $(V,\omega)$, $(\widetilde V,\tilde\omega)$, then a {\em symplectomorphism\/} from $(V,\omega)$
to $(\widetilde V,\tilde\omega)$ is a linear isomorphism:
\[T:V\longrightarrow\widetilde V\]
such that:
\begin{equation}\label{eq:defsymplvecspace}
\tilde\omega\big(T(v),T(w)\big)=\omega(v,w),
\end{equation}
for all $v,w\in V$. One could rephrase \eqref{eq:defsymplvecspace} by saying that the {\em pull-back\/}
$T^*\tilde\omega$ (which is defined by the lefthand side of \eqref{eq:defsymplvecspace}) is equal to $\omega$.
\end{defin}
Clearly, the composition of symplectomorphisms is a symplectomorphism and the inverse of a symplectomorphism
is a symplectomorphism. Symplectomorphisms in the theory of symplectic spaces play the same role that orthogonal
transformations (i.e., linear isometries) play in the theory of vector spaces with inner product.
We now define the analogue for the theory of symplectic spaces of the notion of orthonormal basis.

\begin{defin}
If $(V,\omega)$ is a symplectic space, then a {\em symplectic basis\/} for $(V,\omega)$ is a basis
$(e_1,\ldots,e_n,e'_1,\ldots,e'_n)$ for $V$ such that:
\[\omega(e_i,e'_j)=\delta_{ij},\quad\omega(e_i,e_j)=0,\quad\omega(e'_i,e'_j)=0,\]
for all $i,j=1,\ldots,n$, where $\delta_{ij}=1$ for $i=j$ and $\delta_{ij}=0$ for $i\ne j$.
\end{defin}
Clearly, the canonical basis of $\R^{2n}$ (and the canonical basis of $\R^n\times{\R^n}^*$)
is symplectic with respect to the canonical symplectic form. Just like in the theory of spaces with inner product, we have:

\begin{prop}\label{thm:symplbasis}
Any symplectic space admits a symplectic basis.
\end{prop}
\begin{proof}
This is a simple linear algebra exercise whose details are left to the reader. The idea is to use induction
in the dimension of the symplectic space $(V,\omega)$. If $V$ is not the null space then we can find vectors
$e_1,e'_1\in V$ such that $\omega(e_1,e'_1)\ne0$ and obviously such vectors can be chosen with $\omega(e_1,e'_1)=1$.
The restriction of $\omega$ to the space $\Span\{e_1,e'_1\}$ spanned by $e_1$, $e'_1$ is easily seen to be non degenerate
and thus it follows from the result of Exercise~\ref{exe:perpspan} that $V$ is the direct sum of $\Span\{e_1,e'_1\}$
with the orthogonal complement (with respect to $\omega$) of $\Span\{e_1,e'_1\}$. Now apply the induction hypothesis
to the restriction of $\omega$ to the orthogonal complement of $\Span\{e_1,e'_1\}$ and conclude the proof.
\end{proof}
In Exercise~\ref{exe:symplbasisbetter} we ask the reader to prove a generalization of Proposition~\ref{thm:symplbasis}
that yields a special basis for anti-symmetric bilinear forms $\omega$ that are not necessarily non degenerate.

\begin{cor}\label{thm:corsympleven}
Any symplectic space is even dimensional.
\end{cor}
\begin{proof}
A symplectic basis has an even number of elements.
\end{proof}
Another proof of Corollary~\ref{thm:corsympleven} is given in the statement of Exercise~\ref{exe:proofeven}.

\medskip

We have the following result, which is analogous to the result that two vector spaces having the same dimension,
endowed with inner products, are linearly isometric.
\begin{cor}\label{thm:coronlyonesympl}
If $(V,\omega)$ and $(\widetilde V,\tilde\omega)$ are symplectic spaces having the same dimension, then there exists
a symplectomorphism from $(V,\omega)$ to $(\widetilde V,\tilde\omega)$.
\end{cor}
\begin{proof}
Choose a symplectic basis of $(V,\omega)$ and a symplectic basis of $(\widetilde V,\tilde\omega)$. Define a linear isomorphism
$T:V\to\widetilde V$ sending one basis to the other. Apply the result of Exercise~\ref{exe:equivsymplecto}.
\end{proof}
Because of Corollary~\ref{thm:coronlyonesympl}, in order to prove a theorem about arbitrary symplectic spaces, it suffices
to prove it for $\R^{2n}$, endowed with the canonical symplectic form.

\subsection{The volume form induced by a symplectic form}\label{sub:symplvolume}
Let $(V,\omega)$ be a symplectic space with
$\Dim(V)=2n$. The symplectic form $\omega$ is an element of $\bigwedge_2V^*$ and the wedge product:
\[\omega^n=\omega\wedge\cdots\wedge\omega\]
of $n$ copies of $\omega$ is an element of the one-dimensional space $\bigwedge_{2n}V^*$. We will show that
$\omega^n$ is not zero and therefore it is a volume form over $V$. Because of Corollary~\ref{thm:coronlyonesympl}
it suffices to prove that $\omega^n$ is not zero when $V=\R^{2n}$ and $\omega=\omega_0$ is the canonical
symplectic form of $\R^{2n}$. Let us denote\footnote{%
This is actually more than just a notation. One can see $\dd q^i$, $\dd p_j$ as the (constant) one-forms
over $\R^{2n}$ that are the differentials of the scalar functions $(q,p)\mapsto q^i$, $(q,p)\mapsto p_j$.}
by $(\dd q^1,\ldots,\dd q^n,\dd p_1,\ldots,\dd p_n)$ the dual basis of the canonical basis of $\R^{2n}$.
The canonical symplectic form $\omega_0$ of $\R^{2n}$ is given by:
\begin{equation}\label{eq:omega0dqidpi}
\omega_0=\sum_{i=1}^n\dd q^i\wedge\dd p_i.
\end{equation}
Namely:
\begin{multline*}
\sum_{i=1}^n(\dd q^i\wedge\dd p_i)\big((x,y),(\bar x,\bar y)\big)=
\sum_{i=1}^n\dd q^i(x,y)\dd p_i(\bar x,\bar y)-\dd q^i(\bar x,\bar y)\dd p_i(x,y)\\
=\sum_{i=1}^nx^i\bar y_i-\bar x^iy_i=\langle x,\bar y\rangle-\langle\bar x,y\rangle,
\end{multline*}
for all $x=(x^1,\ldots,x^n)$, $y=(y_1,\ldots,y_n)$, $\bar x=(\bar x^1,\ldots,\bar x^n)$, $\bar y=(\bar y_1,\ldots,\bar y_n)$ in $\R^n$.
The {\em canonical volume form\/} of $\R^{2n}$ is given by:
\[\dd q^1\wedge\cdots\wedge\dd q^n\wedge\dd p_1\wedge\cdots\wedge\dd p_n.\]

Let us compute the wedge product $\omega_0^n$ of $n$ copies of $\omega_0$. We have:
\[\omega_0^n=\sum_{i_1,\ldots,i_n=1}^n\dd q^{i_1}\wedge\dd p_{i_1}\wedge\cdots\wedge\dd q^{i_n}\wedge\dd p_{i_n}.\]
Clearly:
\[\dd q^{i_1}\wedge\dd p_{i_1}\wedge\cdots\wedge\dd q^{i_n}\wedge\dd p_{i_n}=0\]
unless $i_1,\ldots,i_n\in\{1,\ldots,n\}$ are pairwise distinct. Therefore:
\[\omega_0^n=\sum_{\sigma\in S^n}\dd q^{\sigma(1)}\wedge\dd p_{\sigma(1)}\wedge\cdots\wedge\dd q^{\sigma(n)}
\wedge\dd p_{\sigma(n)},\]
where $S^n$ denotes the group of all bijections of the set $\{1,\ldots,n\}$.
By counting order switches the reader can easily check that:
\begin{multline*}
\dd q^{\sigma(1)}\wedge\dd p_{\sigma(1)}\wedge\cdots\wedge\dd q^{\sigma(n)}\wedge\dd p_{\sigma(n)}\\
=(-1)^{\frac{n(n-1)}2}\dd q^{\sigma(1)}\wedge\cdots\wedge\dd q^{\sigma(n)}\wedge\dd p_{\sigma(1)}\wedge
\cdots\wedge\dd p_{\sigma(n)}.
\end{multline*}
We have:
\begin{multline*}
\dd q^{\sigma(1)}\wedge\cdots\wedge\dd q^{\sigma(n)}\wedge\dd p_{\sigma(1)}\wedge\cdots\wedge\dd p_{\sigma(n)}\\
=\sgn(\sigma)^2\dd q^1\wedge\cdots\wedge\dd q^n\wedge\dd p_1\wedge\cdots\wedge\dd p_n\\
=\dd q^1\wedge\cdots\wedge\dd q^n\wedge\dd p_1\wedge\cdots\wedge\dd p_n,
\end{multline*}
where $\sgn(\sigma)$ is the {\em sign\/} of the permutation $\sigma$, i.e., $\sgn(\sigma)=1$ if $\sigma$ is even and $\sgn(\sigma)=-1$ if $\sigma$ is odd.
Hence:
\[\omega_0^n=(-1)^{\frac{n(n-1)}2}n!\,\dd q^1\wedge\cdots\wedge\dd q^n\wedge\dd p_1\wedge\cdots\wedge\dd p_n.\]
This motivates the following:

\begin{defin}\label{thm:defvolume}
If $(V,\omega)$ is a symplectic space with $\Dim(V)=2n$ then the {\em volume form induced by $\omega$\/}
is:
\begin{equation}\label{eq:volumeform}
(-1)^{\frac{n(n-1)}2}\frac1{n!}\,\omega^n,
\end{equation}
where $\omega^n$ denotes the wedge product of $n$ copies of $\omega$.
\end{defin}
Our computations have shown that the volume form induced by the canonical symplectic form of $\R^{2n}$ is just
the canonical volume form of $\R^{2n}$ (obviously, the same holds if $\R^{2n}$ is replaced with $\R^n\times{\R^n}^*$).
Moreover, as remarked earlier, it follows from Corollary~\ref{thm:coronlyonesympl}
that \eqref{eq:volumeform} is indeed a volume form over $V$, i.e., it is not zero. Namely, the pull-back
of \eqref{eq:volumeform} by a symplectomorphism from $(\R^{2n},\omega_0)$ to $(V,\omega)$ is equal to the canonical volume
form of $\R^{2n}$, so that \eqref{eq:volumeform} cannot be zero.

\end{section}

\begin{section}{Symplectic manifolds and Hamiltonians}
\label{sec:symplmanifold}

In Section~\ref{sec:Hamformalism} we have seen that the solutions to Hamilton's equations are precisely
the integral curves of the symplectic gradient of the Hamiltonian. Let us now formulate Hamilton's equations
in the context of manifolds. The required ingredients are a Hamiltonian and a symplectic form.
In this section we are going to use several facts that are taught during courses on calculus on manifolds.
The reader is assumed to be familiar with such facts, but there is a short summary of those
in the appendix (Section~\ref{sec:quickmanifold}).

\begin{defin}
Let $M$ be a differentiable manifold. By a {\em symplectic form\/} over $M$ we mean a smooth two-form
$\omega$ over $M$ such that:
\begin{itemize}
\item[(a)] $\omega_x$ is non degenerate (i.e., $\omega_x$ is a symplectic form over the tangent space $T_xM$),
for all $x\in M$;
\item[(b)] $\omega$ is closed, i.e., the exterior derivative $\dd\omega$ vanishes.
\end{itemize}
The pair $(M,\omega)$ is called a {\em symplectic manifold}.
\end{defin}
Observe that if $(V,\omega)$ is a symplectic space (in the sense of Section~\ref{sec:symplvecspace})
and if we regard $\omega$ as a {\em constant\/} two-form over $V$ (i.e., the two-form that associates $\omega$
to every point $x\in V$) then $\omega$ is automatically closed and therefore
$(V,\omega)$ is a symplectic manifold. Actually, we have a bit of a terminological conflict here: if $V$
is a real finite-dimensional vector space then a symplectic form over the vector space $V$
(in the sense of Section~\ref{sec:symplvecspace}) is not the same thing as a symplectic form over $V$ when $V$ is
regarded as a manifold. When $V$ is regarded as a manifold, then a symplectic form $\omega$ over $V$ gives us a
symplectic form $\omega_x$ over the vector space $V$ {\em for each\/} $x\in V$. Terminological conflicts of this type are
common during courses on differentiable manifolds and they cause no disastrous misunderstandings.

It is an obvious consequence of Corollary~\ref{thm:corsympleven} that every symplectic manifold is even dimensional.

It is not at all obvious why we have chosen to require a symplectic form on a manifold to be closed. It happens that, because
of such assumption, all symplectic manifolds are locally alike (Darboux's theorem below). The reader will see that
the assumption that the symplectic form be closed is crucial for the most basic theorems on the subject.
A (not necessarily closed) smooth two-form that is non degenerate at every point is usually called {\em almost symplectic}.

Let us now formulate Hamilton's equations in a symplectic manifold.
\begin{defin}\label{thm:defsymplgrad}
Let $(M,\omega)$ be a symplectic manifold and $H:M\to\R$ be a smooth map. We call it a {\em Hamiltonian\/} over $M$
(we will use that terminology sometimes also when $H$ is only defined in an open subset of $M$). The
{\em symplectic gradient\/} of $H$ is the unique smooth vector field $\vec H$ over $M$ such that:
\[\omega\big(\vec H(x),\cdot\big)=\dd H(x)\in T_xM^*,\]
for all $x\in M$. A smooth map $H:\R\times M\to\R$ (perhaps defined only over some open subset of $\R\times M$)
is called a {\em time-dependent Hamiltonian\/} over $M$. It's {\em symplectic gradient\/} is defined to be the unique time-dependent
vector field $\vec H$ over $M$ such that:
\[\omega\big(\vec H(t,x),\cdot\big)=\partial_x H(t,x)\in T_xM^*,\]
for all $(t,x)\in\R\times M$, where $\partial_x H(t,x)$ denotes the differential at the point $x$ of the map
$H(t,\cdot)$. An integral curve of the symplectic gradient $\vec H$ of a (possibly time-dependent) Hamiltonian $H$
is called a {\em solution to Hamilton's equations}.
\end{defin}
If $H$ is a time-dependent Hamiltonian then, for each $t\in\R$, $H(t,\cdot)$ is a (time {\em in\/}dependent) Hamiltonian and the
symplectic gradient of $H(t,\cdot)$ is the vector field $\vec H(t,\cdot)$, where $\vec H$ denotes the symplectic gradient of $H$. In other words,
the symplectic gradient of a time-dependent Hamiltonian at $(t,x)\in\R\times M$ can be obtained by first ``freezing time'', obtaining
$H(t,\cdot)$, and then computing the symplectic gradient of $H(t,\cdot)$ at $x$.

\medskip

Here is the generalization of the notion of symplectomorphism to the context of manifolds.
\begin{defin}
Given symplectic manifolds $(M,\omega)$, $(\widetilde M,\tilde\omega)$ then a {\em symplectomorphism\/} from $(M,\omega)$
to $(\widetilde M,\tilde\omega)$ is a smooth diffeomorphism:
\[\Phi:M\longrightarrow\widetilde M\]
such that:
\[\Phi^*\tilde\omega=\omega,\]
i.e., such that $\dd\Phi_x:T_xM\to T_{\Phi(x)}\widetilde M$ is a symplectomorphism from the symplectic space $(T_xM,\omega_x)$
to the symplectic space $(T_{\Phi(x)}\widetilde M,\tilde\omega_{\Phi(x)})$. A {\em symplectic chart\/} over a symplectic manifold
$(M,\omega)$ is a local chart:
\[\Phi:U\subset M\longrightarrow\widetilde U\subset\R^{2n}\]
that is a symplectomorphism from $U$ endowed with (the restriction of) $\omega$ to $\widetilde U$ endowed with (the restriction of) the canonical
symplectic form of $\R^{2n}$ (we also allow the counter-domain of a symplectic chart to be an open subset of
$\R^n\times{\R^n}^*$).
\end{defin}

\begin{teo}[Darboux]\label{thm:Darboux}
If $(M,\omega)$ is a symplectic manifold then for every point of $M$ there exists a symplectic chart whose domain
contains that point.
\end{teo}
\begin{proof}
See Exercise~\ref{exe:Darboux}.
\end{proof}
Darboux's theorem won't be terribly important for us, since for what is going to be our central example of symplectic manifold (cotangent bundles), the symplectic charts can
be easily constructed. It is easy to see that the integral curves of a symplectic gradient $\vec H$ are represented, with respect to a symplectic chart $\Phi$,
by solutions of the (standard) Hamilton's equations
\eqref{eq:Hamilton3} corresponding to the Hamiltonian that represents $H$ with respect to $\Phi$ (see Exercise~\ref{exe:Hamiltcoords} for details).

\medskip

We have proven in Section~\ref{sec:Hamformalism} (Proposition~\ref{thm:propHconstant}) that a Hamiltonian that does not depend on time is a first integral of Hamilton's
equations. Such fact generalizes to Hamiltonians on symplectic manifolds.
\begin{teo}\label{thm:Hfirstint}
If $H$ is a (time {\em independent})
Hamiltonian over a symplectic manifold $(M,\omega)$ then $H$ is constant over the integral
curves of the symplectic gradient $\vec H$, i.e., $H$ is a first integral of $\vec H$.
\end{teo}
\begin{proof}
If $t\mapsto x(t)$ is an integral curve of $\vec H$ then:
\[\frac{\dd}{\dd t}H\big(x(t)\big)=\dd H_{x(t)}\big(\dot x(t)\big)=\dd H_{x(t)}(\vec H_{x(t)})=\omega\big(\vec H_{x(t)},\vec H_{x(t)})=0,\]
where $\dot x(t)=\frac{\dd x}{\dd t}(t)$.
\end{proof}
Theorem~\ref{thm:Hfirstint} does not hold for time-dependent Hamiltonians. Actually, the argument used in its proof shows that if $t\mapsto x(t)$ is an
integral curve of $\vec H$ then:
\[\frac{\dd}{\dd t}H\big(t,x(t)\big)=\frac{\partial H}{\partial t}\big(t,x(t)\big).\]

Now we establish a deeper relationship between the flow of $\vec H$ and the symplectic structure.
\begin{teo}
If $H$ is a time-dependent Hamiltonian over a symplectic manifold $(M,\omega)$ then the symplectic form $\omega$ is invariant
under the flow of the symplectic gradient $\vec H$.
\end{teo}
\begin{proof}
We just have to check that the Lie derivative $\mathbb L_{\vec H_t}\omega$ is zero, for all $t\in\R$, where $\vec H_t=\vec H(t,\cdot)$
(see Proposition~\ref{thm:invariantflowLie} if you are not familiar with this). We use the standard formula for the Lie derivative of a differential
form (see \eqref{eq:diid}):
\[\mathbb L_{\vec H_t}\omega=\dd i_{\vec H_t}\omega+i_{\vec H_t}\dd\omega.\]
Since $\omega$ is closed, the second term on the righthand side of the equality above vanishes. As for the first term,
observe that, by the definition of $\vec H_t$, $i_{\vec H_t}\omega$ is equal to $\dd H_t$, where $H_t=H(t,\cdot)$.
Since $\dd(\dd H_t)=0$, the proof is concluded.
\end{proof}

We have seen in Subsection~\ref{sub:symplvolume} (recall Definition~\ref{thm:defvolume}) that a symplectic form over a vector space induces a volume form over that vector
space. Obviously, the same construction can be used (pointwise) for manifolds.
\begin{defin}
The {\em volume form\/} induced by a symplectic form $\omega$ on a manifold $M$ is defined by:
\begin{equation}\label{eq:volformmanifold}
(-1)^{\frac{n(n-1)}2}\frac1{n!}\,\omega^n,
\end{equation}
where $\omega^n$ denotes the wedge product of $n$ copies of $\omega$ and $n$ denotes half the dimension of $M$.
\end{defin}
We have already shown that $\omega^n$ never vanishes, so that \eqref{eq:volformmanifold} is indeed a volume
form over $M$.

\begin{cor}[Liouville's theorem]\label{thm:corLiouville}
The volume form induced by the symplectic form is invariant under the flow of the symplectic gradient
of a (possibly time-dependent) Hamiltonian.
\end{cor}
\begin{proof}
A diffeomorphism that preserves the symplectic form preserves the volume form.
\end{proof}

Liouville's theorem is very important for Statistical Mechanics. Also, there are many theorems about the dynamics of a flow that preserves a measure
(which, by Liouville's theorem, is the case of a Hamiltonian flow).

\end{section}

\begin{section}{Canonical forms in a cotangent bundle}
\label{sec:canformscot}

In this section we will show that the cotangent bundle of a differentiable manifold carries a canonical one-form
and a canonical symplectic form; such symplectic form is, up to a sign, equal to the exterior differential of the canonical one-form.
The symplectic form will allow us to talk about Hamilton's equations in a cotangent bundle.

Let $Q$ be a differentiable manifold and denote by $TQ^*$ its cotangent bundle. A point of $TQ^*$ will be written as an ordered pair $(q,p)$, where
$q$ is a point of $Q$ and $p\in T_qQ^*$ is a linear functional over the tangent space $T_qQ$. Denote by $\pi:TQ^*\to Q$ the {\em canonical projection},
i.e., $\pi(q,p)=q$, for all $(q,p)\in TQ^*$.
The cotangent bundle $TQ^*$ is a differentiable manifold and the projection $\pi$ is a smooth map (it is also a smooth submersion).
We are going to define a one-form $\theta$ over the differentiable manifold $TQ^*$, i.e., for each point $(q,p)\in TQ^*$ we are going to associate a linear
functional $\theta_{(q,p)}$ over the tangent space $T_{(q,p)}TQ^*$. The construction of $\theta$ is straightforward: given a point $(q,p)\in TQ^*$, we consider
the differential $\dd\pi_{(q,p)}$ of the projection $\pi$ at the point $(q,p)$, which is a linear map:
\[\dd\pi_{(q,p)}:T_{(q,p)}TQ^*\longrightarrow T_qQ.\]
Now, $p$ is a linear functional over $T_qQ$ and therefore we can compose it with $\dd\pi_{(q,p)}$ to obtain a linear functional over $T_{(q,p)}TQ^*$.
We set:
\begin{equation}\label{eq:deftheta}
\theta_{(q,p)}=p\circ\dd\pi_{(q,p)}:T_{(q,p)}TQ^*\longrightarrow\R,
\end{equation}
for all $(q,p)\in TQ^*$, so that:
\[\theta_{(q,p)}(\zeta)=p\big(\dd\pi_{(q,p)}(\zeta)\big),\]
for all $(q,p)\in TQ^*$ and all $\zeta\in T_{(q,p)}TQ^*$.

\begin{defin}
The one-form $\theta$ defined in \eqref{eq:deftheta} is called the {\em canonical one-form\/} of the cotangent bundle $TQ^*$.
\end{defin}

\begin{example}\label{exa:thetaopen}
Let us compute explicitly the canonical one-form $\theta$ of the cotangent bundle $TQ^*$ of an open subset $Q$ of $\R^n$. Such cotangent bundle is
identified with the product $Q\times{\R^n}^*$ and the canonical projection $\pi:TQ^*\to Q$ is just the projection onto the first coordinate
of the product $Q\times{\R^n}^*$. Given
a point $(q,p)\in TQ^*$, i.e., $q$ is in $Q$ and $p$ is in ${\R^n}^*$, then the tangent space $T_{(q,p)}TQ^*$ is identified with $\R^n\times{\R^n}^*$ and
the differential $\dd\pi_{(q,p)}$ is the projection onto the first coordinate of the product $\R^n\times{\R^n}^*$. The canonical one-form $\theta$ is given by:
\begin{equation}\label{eq:thetaopen}
\theta_{(q,p)}(\zeta_1,\zeta_2)=p(\zeta_1),\quad q\in Q,\ p\in{\R^n}^*,\ \zeta_1\in\R^n,\ \zeta_2\in{\R^n}^*.
\end{equation}
Let us write the one-form $\theta$ using the standard way of writing down differential forms. Denote by\footnote{%
Yes, we are using $q$ and $p$ both for the names of the projection maps of the product $Q\times{\R^n}^*$
and for the names of points of $Q$ and of ${\R^n}^*$, respectively.}:
\[q:Q\times{\R^n}^*\longrightarrow Q,\quad p:Q\times{\R^n}^*\longrightarrow{\R^n}^*\]
the projection maps of the product $Q\times{\R^n}^*$ and write:
\[q=(q^1,\ldots,q^n),\quad p=(p_1,\ldots,p_n),\]
so that $q^i$ and $p_i$, $i=1,\ldots,n$, are real valued maps
over $Q\times{\R^n}^*$ (when we write $p=(p_1,\ldots,p_n)$, we are using the standard identification between ${\R^n}^*$ and $\R^n$).
Denote by $\dd q^i$, $\dd p_i$ the (ordinary or exterior) differential of such maps, which are (constant) one-forms
over $Q\times{\R^n}^*$. The one-forms $\dd q^i$, $\dd p_i$ are simply the dual basis of the canonical basis of $\R^n\times{\R^n}^*$, i.e., they
are the $2n$ projections of $\R^n\times{\R^n}^*\cong\R^{2n}$.
Given $\zeta=(\zeta_1,\zeta_2)\in\R^n\times{\R^n}^*$ then the $n$ coordinates of $\zeta_1\in\R^n$ are $\dd q^i(\zeta)$, $i=1,\ldots,n$ and therefore,
by \eqref{eq:thetaopen}, the one-form $\theta$ is equal to:
\begin{equation}\label{eq:thetapdq}
\theta=\sum_{i=1}^np_i\dd q^i.
\end{equation}
As we will see in one moment, by computing the canonical one-form of the cotangent bundle of an open subset of $\R^n$, we have
actually computed the canonical one-form of any cotangent bundle.
\end{example}

Given differentiable manifolds $Q$, $\widetilde Q$, if $\varphi:Q\to\widetilde Q$ is a smooth diffeomorphism then it induces a smooth diffeomorphism:
\[\dd\varphi:TQ\longrightarrow T\widetilde Q\]
between the tangent bundles $TQ$, $T\widetilde Q$ (which sends a point $(q,\dot q)$ in $TQ$ to the point
$\big(\varphi(q),\dd\varphi_q(\dot q)\big)$ in $T\widetilde Q$)
and it also induces a smooth diffeomorphism:
\[\dd^*\varphi:TQ^*\longrightarrow T\widetilde Q^*\]
between the cotangent bundles $TQ^*$, $T\widetilde Q^*$, defined by:
\begin{equation}\label{eq:dstarvarphi}
\dd^*\varphi(q,p)=\big(\varphi(q),p\circ\dd\varphi_q^{-1}\big)=\big(\varphi(q),(\dd\varphi_q^{-1})^*(p)\big),\quad(q,p)\in TQ^*.
\end{equation}
Be aware that our notation might be a little misleading: the restriction of $\dd^*\varphi$ to a fiber $T_qQ^*$ of $TQ^*$
is not the transpose $\dd\varphi_q^*$ of the differential $\dd\varphi_q$, but actually its {\em inverse\/}
$(\dd\varphi_q^*)^{-1}=(\dd\varphi_q^{-1})^*$ (the map $\dd\varphi_q^*$
goes in the wrong direction!). That is why we write $\dd^*\varphi$, instead of $\dd\varphi^*$; using $\dd\varphi^*$
would be too misleading.

We have a commutative diagram:
\begin{equation}\label{eq:diagdstarvarphi}
\vcenter{\xymatrix@C+15pt{%
TQ^*\ar[r]^{\dd^*\varphi}\ar[d]&T\widetilde Q^*\ar[d]\\
Q\ar[r]_\varphi&\widetilde Q}}
\end{equation}
in which the vertical arrows are the canonical projections.

If $\theta$ denotes the canonical one-form of $TQ^*$ and $\tilde\theta$ denotes the canonical one-form of $T\widetilde Q^*$ then it is a simple
exercise\footnote{%
Just differentiate the arrows of diagram \eqref{eq:diagdstarvarphi} and follow the definitions!}
to check that the diffeomorphism $\dd^*\varphi$ carries $\theta$ to $\tilde\theta$, i.e, the pull-back of $\tilde\theta$ by $\dd^*\varphi$ is
equal to $\theta$:
\begin{equation}\label{eq:dstarphiprestheta}
(\dd^*\varphi)^*\tilde\theta=\theta.
\end{equation}
The reader should have quickly guessed that \eqref{eq:dstarphiprestheta} must hold! This is just a particular case of the fact that any concept whose definition makes
sense for an arbitrary differentiable manifold must be preserved by smooth diffeomorphisms (just like any concept whose definition makes sense for rings must
be preserved by ring isomorphisms, and so on).

If $\varphi:U\subset Q\to\widetilde U\subset\R^n$ is a local chart on $Q$ then the smooth diffeomorphism:
\[\dd^*\varphi:TU^*\subset TQ^*\longrightarrow T\widetilde U^*=\widetilde U\times{\R^n}^*\]
is a local chart on the cotangent bundle $TQ^*$; it is the local chart {\em canonically associated\/} to $\varphi$ on the cotangent bundle.
The local chart $\dd^*\varphi$ on $TQ^*$ carries the restriction\footnote{%
Obviously, the canonical one-form of the cotangent bundle $TU^*$ of the open subset $U$ of $Q$ is the restriction to $TU^*$ of the canonical one-form
of the cotangent bundle $TQ^*$.} to the open set $TU^*$ of the canonical one-form $\theta$ of $TQ^*$ to the canonical one-form of the cotangent
bundle $T\widetilde U^*$ of the open subset $\widetilde U$ of $\R^n$. Thus, the canonical one-form of the cotangent
bundle $T\widetilde U^*$ (which we have computed in Example~\ref{exa:thetaopen})
is the representation of the canonical one-form $\theta$ of $TQ^*$ with respect to the local chart $\dd^*\varphi$.
Notice that our considerations have proven that {\em the canonical one-form of a cotangent bundle is smooth}.

There are two possibilities here for notation and we will let the reader pick her favorite
one. We can, as in Example~\ref{exa:thetaopen}, denote by:
\begin{equation}\label{eq:qipi}
q^1,\ldots,q^n,p_1,\ldots,p_n,
\end{equation}
the $2n$ projections of $\widetilde U\times{\R^n}^*\subset\R^n\times{\R^n}^*$,
so that the canonical one-form of the cotangent bundle $T\widetilde U^*$ is given by the righthand side of \eqref{eq:thetapdq} and the (restriction
to the open set $TU^*$ of the) canonical one-form $\theta$ of $TQ^*$ is given by the pull-back:
\[\theta=(\dd^*\varphi)^*\Big(\sum_{i=1}^np_i\dd q^i\Big).\]
The other possibility is to use \eqref{eq:qipi} to denote the coordinate functions of the map $\dd^*\varphi$, so that
$\dd^*\varphi=(q^1,\ldots,q^n,p_1,\ldots,p_n)$ and \eqref{eq:qipi} denote real valued maps over the open subset $TU^*$ of $TQ^*$.
Under such notation, the (restriction to $TU^*$ of) the canonical one-form $\theta$ of $TQ^*$ is given exactly by the righthand side of equality \eqref{eq:thetapdq}.

\begin{defin}
Given a differentiable manifold $Q$, then the {\em canonical symplectic form\/} of the cotangent bundle $TQ^*$ is defined by:
\begin{equation}\label{eq:defomega}
\omega=-\dd\theta,
\end{equation}
where $\theta$ denotes the canonical one-form of $TQ^*$.
\end{defin}
We use a minus sign in \eqref{eq:defomega} because we want $\omega$ to agree with the canonical symplectic form of $\R^n\times{\R^n}^*$ when $Q=\R^n$. We will
see in a moment that $\omega$ is indeed a symplectic form over $TQ^*$. It is already clear that it is smooth (because $\theta$ is smooth)
and that it is closed (because it is exact).
If $Q$, $\widetilde Q$ are differentiable manifolds and $\varphi:Q\to\widetilde Q$ is a smooth diffeomorphism then, by \eqref{eq:dstarphiprestheta},
since pull-backs commute with exterior differentiation, it follows that:
\[(\dd^*\varphi)^*\widetilde\omega=\omega,\]
where $\omega$ denotes the canonical symplectic form of $TQ^*$ and $\widetilde\omega$ the canonical symplectic form of $T\widetilde Q^*$.

\begin{example}
If $Q$ is an open subset of $\R^n$ then, using the same notation used in Example~\ref{exa:thetaopen}, we see (by taking the exterior derivative
on both sides of \eqref{eq:thetapdq}) that the canonical symplectic
form $\omega$ of $TQ^*$ is given by:
\begin{equation}\label{eq:omegadqidpi}
\omega=\sum_{i=1}^n\dd q^i\wedge\dd p_i,
\end{equation}
and therefore it agrees with the canonical symplectic form \eqref{eq:omega0dqidpi} of the space $\R^n\times{\R^n}^*$.
If $H:\Dom(H)\subset\R\times TQ^*=\R\times Q\times{\R^n}^*\to\R$ is a time-dependent Hamiltonian over $Q$ and if
$TQ^*$ is endowed with its canonical symplectic form $\omega$ then the symplectic gradient of $H$ is given by:
\[\Dom(H)\ni(t,q,p)\longmapsto\Big(\frac{\partial H}{\partial p}(t,q,p),-\frac{\partial H}{\partial q}(t,q,p)\Big)
\in\R^n\times{\R^n}^*.\]
Given a smooth curve $(q,p):I\to TQ^*$ (defined over some interval $I\subset\R$) then, for $t\in I$
with $\big(t,q(t),p(t)\big)\in\Dom(H)$, the condition:
\[\frac{\dd}{\dd t}\big(q(t),p(t)\big)=\vec H\big(t,q(t),p(t)\big)\]
is equivalent to (the satisfaction at $t$ of) Hamilton's equations:
\begin{equation}\label{eq:Hameqagain}
\frac{\dd q}{\dd t}(t)=\frac{\partial H}{\partial p}\big(t,q(t),p(t)\big),\quad
\frac{\dd p}{\dd t}(t)=-\frac{\partial H}{\partial q}\big(t,q(t),p(t)\big).
\end{equation}
\end{example}

If $\varphi:U\subset Q\to\widetilde U\subset\R^n$ is a local chart on $Q$ then the (restriction to $TU^*$ of the) symplectic form
$\omega$ of $TQ^*$ is the pull-back by the local chart $\dd^*\varphi$ of the (restriction to $\widetilde U\times{\R^n}^*$ of the)
canonical symplectic form of $\R^n\times{\R^n}^*$. It follows that the bilinear form $\omega_{(q,p)}$ on $T_{(q,p)}TQ^*$ is non degenerate,
for all $(q,p)\in TQ^*$.

Again, we have two possibilities for notation: use \eqref{eq:qipi} to denote the projections of $\widetilde U\times{\R^n}^*\subset\R^n\times{\R^n}^*$,
so that the (restriction to $TU^*$ of the) canonical symplectic form $\omega$ of $TQ^*$ is the pull-back by $\dd^*\varphi$ of the righthand
side of \eqref{eq:omegadqidpi} or use \eqref{eq:qipi} to denote the coordinate functions of the map $\dd^*\varphi$, so that
the righthand side of \eqref{eq:omegadqidpi} is precisely the (restriction to $TU^*$ of the) canonical symplectic form $\omega$ of $TQ^*$.

\medskip

Our considerations so far have proven the following:
\begin{prop}
Given a differentiable manifold $Q$ then:
\begin{itemize}
\item[(a)] the canonical symplectic form of $TQ^*$ is indeed a symplectic form;
\item[(b)] if $\widetilde Q$ is another differentiable manifold and if $\varphi:Q\to\widetilde Q$ is a smooth diffeomorphism then
$\dd^*\varphi:TQ^*\to T\widetilde Q^*$ is a symplectomorphism if both tangent bundles are endowed with their canonical symplectic forms;
\item[(c)] if $\varphi:U\subset Q\to\widetilde U\subset\R^n$ is a local chart on $Q$ then the local chart $\dd^*\varphi$ on $TQ^*$ is symplectic
if $TQ^*$ is endowed with its canonical symplectic form.\qed
\end{itemize}
\end{prop}

Let $H:\Dom(H)\subset\R\times TQ^*\to\R$ be a time-dependent Hamiltonian over the cotangent bundle $TQ^*$ of a differentiable manifold $Q$.
Given a local chart $\varphi:U\subset Q\to\widetilde U\subset\R^n$ on $Q$ then, since the local chart $\dd^*\varphi$ on $TQ^*$ is symplectic,
we have that a curve on $TQ^*$ is an integral curve of the symplectic gradient $\vec H$ if and only if the representation of such curve with
respect to the chart $\dd^*\varphi$ is a solution of Hamilton's equations \eqref{eq:Hameqagain} corresponding
to the Hamiltonian that represents $H$ with respect to the chart $\dd^*\varphi$ (see Exercise~\ref{exe:Hamiltcoords} for details).

\medskip

The first of Hamilton's equations can be formulated without the aid of a coordinate chart. More explicitly, given $t\in\R$, $q\in Q$, we can freeze
the first two variables of $H$, obtaining a map $H(t,q,\cdot)$ that sends each $p$ in the open subset:
\[\Dom\!\big(H(t,q,\cdot)\big)=\big\{p\in T_qQ^*:(t,q,p)\in\Dom(H)\big\}\]
of the cotangent space $T_qQ^*$ to the real number $H(t,q,p)$. The differential of the map $H(t,q,\cdot)$ at a point $p$ of its domain will be denoted
by $\frac{\partial H}{\partial p}(t,q,p)$ and it is a linear functional over $T_qQ^*$, i.e., it is an element of the bidual $T_qQ^{**}$, which we identify
with an element of the tangent space $T_qQ$. Unfortunately, the partial derivative $\frac{\partial H}{\partial q}(t,q,p)$
does not make sense for a time-dependent Hamiltonian $H$ on a cotangent bundle, as one can fix $q$ and move $p$, but one cannot fix $p$ and move $q$
(one would need a connection\footnote{%
A connection on $Q$ induces a direct sum decomposition of the tangent space $T_{(q,p)}TQ^*$ into a {\em vertical subspace\/}
(the space tangent to the fiber) and a {\em horizontal subspace\/} (which is defined by the connection). The derivative
$\frac{\partial H}{\partial p}(t,q,p)$ is the differential of $H$ along the vertical subspace and the derivative
$\frac{\partial H}{\partial q}(t,q,p)$ can be defined as the differential of $H$ along the horizontal subspace (which, obviously,
depends on the choice of connection).} to make sense out of $\frac{\partial H}{\partial q}(t,q,p)$ without
the aid of a local chart).

The following proposition simply says that if a curve
$t\mapsto\big(q(t),p(t)\big)$ on a cotangent bundle satisfies Hamilton's equations (formulated intrinsically,
in terms of the symplectic gradient with respect to the canonical symplectic form of the cotangent bundle) then it also
satisfies the first of Hamilton's equations (which can be formulated without the aid of a coordinate chart).
\begin{prop}\label{thm:firstHam}
If a smooth curve $(q,p):I\to TQ^*$ (defined on some interval $I\subset\R$) satisfies $\big(t,q(t),p(t)\big)\in\Dom(H)$ and:
\begin{equation}\label{eq:tangvecH}
\frac{\dd}{\dd t}\big(q(t),p(t)\big)=\vec H\big(t,q(t),p(t)\big)
\end{equation}
for a certain $t\in I$ then:
\begin{equation}\label{eq:firstHamintr}
\frac{\dd q}{\dd t}(t)=\frac{\partial H}{\partial p}\big(t,q(t),p(t)\big)\in T_{q(t)}Q.
\end{equation}
\end{prop}
\begin{proof}
If two manifolds $Q$, $\widetilde Q$ are diffeomorphic and if the thesis of the proposition holds for $\widetilde Q$ then it also holds for $Q$
(if this is not obvious to you, see Exercise~\ref{exe:firstHam}). Moreover, in order to prove the thesis for a certain manifold $Q$, we can replace $Q$
with an open neighborhood of the point $q(t)$ in $Q$. Since a sufficiently small neighborhood of a point of $Q$ is diffeomorphic to an open subset of $\R^n$,
it suffices to prove the proposition in the case when $Q$ is an open subset of $\R^n$. In that case, condition \eqref{eq:tangvecH}
means that the curve $(q,p)$ satisfies Hamilton's equations \eqref{eq:Hameqagain}
(at the given instant $t$) and condition \eqref{eq:firstHamintr} means that the curve $(q,p)$
satisfies the first of Hamilton's equations (at the given instant $t$), so the conclusion is obvious.
\end{proof}

The method that we used to prove Proposition~\ref{thm:firstHam} is a very nice strategy for proving theorems about
differentiable manifolds. It works as follows: suppose that we want to prove a certain statement about differentiable manifolds\footnote{%
More generally, the statement might be about a family of manifolds. The strategy can be easily adapted
for that situation as well.}.
For each differentiable manifold $Q$, the statement may or may not hold for $Q$, i.e., the statement
defines a subclass $\mathfrak C$ of the class of all differentiable manifolds. Proving the statement
means to prove that $\mathfrak C$ is the class of all differentiable manifolds.
For a statement that {\em really is\/} a statement about differentiable manifolds (i.e., it concerns the manifold
structure alone), it should be the case that if a manifold diffeomorphic to $Q$ is in $\mathfrak C$ than also $Q$ is in $\mathfrak C$.
Let us say in this case that the class $\mathfrak C$ is {\em closed under diffeomorphisms}.
Proving that the class $\mathfrak C$ is closed under diffeomorphisms normally is a completely follow-your-nose type of
exercise, i.e., one just needs to use a given diffeomorphism to keep carrying things over from one manifold to the other.
We say that the statement under consideration is {\em local\/} (or that the class $\mathfrak C$ is local)
if the class $\mathfrak C$ has, in addition, the following property: given a differentiable manifold $Q$, if every point
of $Q$ has an open neighborhood in $Q$ that (regarded as a manifold in its own right) belongs to $\mathfrak C$, then $Q$ belongs
to $\mathfrak C$. Normally, when a statement is local, it is very easy to check that it really is local.
Obviously, if $\mathfrak C$ is closed under diffeomorphisms and local then, if every open subset of $\R^n$ is in $\mathfrak C$,
it follows that every differentiable manifold is in $\mathfrak C$. We have therefore the following strategy for
proving a local statement about differentiable manifolds: (i) observe that the class $\mathfrak C$ defined by the statement
is closed under diffeomorphisms; (ii) check that the class $\mathfrak C$ is local; (iii) prove the statement for
open subsets of $\R^n$. There is nothing spectacular about this strategy, but it allows one to focuss on what is really
important, i.e., proving the statement for open subsets of $\R^n$. Many authors tend to present proofs of theorems
about differentiable manifolds in which the proof of the statement for open subsets of $\R^n$ gets mixed up with a lot
of procedures of carrying things around using coordinate charts. Our strategy relieves the mind
from such distractions.

\end{section}

\begin{section}{The Legendre transform}

The Legendre transform is the procedure that is used to transform a Lagrangian $L:\R\times TQ\to\R$ into a Hamiltonian $H:\R\times TQ^*\to\R$.
This procedure is performed {\em fiberwise}: for each $t\in\R$ and each $q\in Q$, we freeze the first two variables of $L$, obtaining a map
$L(t,q,\cdot)$ over the tangent space $T_qQ$; on such map, we perform a certain procedure (which we also call ``Legendre transform'')
that yields the map $H(t,q,\cdot)$ over the cotangent space $T_qQ^*$. We start by studying the procedure which is applied to the map $L(t,q,\cdot)$;
that is a general procedure that can be applied to real valued maps over a real finite-dimensional vector space.

In what follows, $E$ denotes a fixed real finite-dimensional vector space. Let $f:\Dom(f)\subset E\to\R$ be a map of class $C^2$ defined over an
open subset $\Dom(f)$ of $E$. The differential $\dd f(x)$ of $f$ at a point $x\in\Dom(f)$ is a linear functional over $E$, i.e., an element
of the dual space $E^*$. The differential $\dd f$ is therefore a map (of class $C^1$):
\begin{equation}\label{eq:ddfEEstar}
\dd f:\Dom(f)\subset E\longrightarrow E^*
\end{equation}
from the open subset $\Dom(f)$ of $E$ to the dual space $E^*$.

\begin{defin}
The map $f:\Dom(f)\subset E\to\R$ is called {\em regular\/} if its differential \eqref{eq:ddfEEstar} is a local diffeomorphism and it is called
{\em hyper-regular\/} if its differential is a diffeomorphism onto an open subset of $E^*$ (the image $\Img(\dd f$)
of the map $\dd f$).
\end{defin}
It follows from the inverse function theorem that $f$ is regular if and only if its second differential:
\[\dd(\dd f)(x):E\longrightarrow E^*\]
is a linear isomorphism for all $x\in\Dom(f)\subset E$. Moreover, $f$ is hyper-regular if and only if $f$ is regular and the map $\dd f$ is injective.

Given a map $f:\Dom(f)\subset E\to\R$ of class $C^2$ over an open subset $\Dom(f)$ of $E$, we consider the map $\phi:\Dom(f)\subset E\to\R$
(of class $C^1$) defined by:
\begin{equation}\label{eq:defphiLeg}
\phi(x)=\dd f(x)x-f(x),\quad x\in\Dom(f).
\end{equation}
The expression $\dd f(x)x$ denotes the evaluation of the linear functional $\dd f(x)$ at the vector $x$.
\begin{defin}\label{thm:defLegendre}
If $f:\Dom(f)\subset E\to\R$ is a hyper-regular map of class $C^2$ then its {\em Legendre transform\/} is the map
$f^*:\Dom(f^*)\subset E^*\to\R$ defined by:
\[f^*=\phi\circ(\dd f)^{-1},\]
with $\phi$ given by \eqref{eq:defphiLeg}.
The domain $\Dom(f^*)$ of $f^*$ is the image $\Img(\dd f)$ of the map $\dd f$.
\end{defin}
The map $f^*$ looks like just a map of class $C^1$ (since both $\phi$ and the inverse of $\dd f$ are maps of class $C^1$), but, curiously,
we will prove below that the map $f^*$ is of class $C^2$. In any case, we are really interested only in the case when $f$ is smooth and clearly in that case
$f^*$ is smooth as well. The reader might be a little puzzled by Definition~\ref{thm:defLegendre}.
It seems as if the formula for $f^*$ just fell from the sky.
Many authors define the Legendre transform in terms of the solution to a maximization problem\footnote{%
One defines the value of the Legendre transform $f^*$ at a linear functional $\alpha\in E^*$ to be the maximum of
$\alpha(x)-f(x)$, with $x$ running through the domain of $f$. When $f$ is convex, such maximum is attained precisely at the
point $x$ such that $\dd f(x)=\alpha$ (if such point exists). Therefore, in that case, the definition using the maximization problem agrees with ours.
The maximization problem might create a sense of motivation on some readers, but it is nevertheless a bad idea: it creates an illusion of difficulty in the
case when $f$ is not convex. It is an ``illusion of difficulty'', because the difficulty is being created by the bad choice of definition alone! With our definition,
there are no difficulties when $f$ is not convex and all the nice properties of the Legendre transform can be (easily) proven as well.}, but we do not.
In a moment, the reader should be convinced that the Legendre transform is a very clever construction.

\begin{example}\label{exa:LegtransLagMech}
Consider the Lagrangian:
\[L:\Dom(V)\times(\R^3)^n\subset\R\times(\R^3)^n\times(\R^3)^n\to\R\]
defined in Subsection~\ref{sub:varClassMech}. Its Euler--Lagrange equation is the dynamical equation for a system of $n$ particles subject to (the
forces with) a potential $V:\Dom(V)\subset\R\times(\R^3)^n\to\R$. Given $(t,q)\in\Dom(V)\subset\R\times(\R^3)^n$,
consider the map $f:(\R^3)^n\to\R$ obtained by freezing the first two variables of $L$:
\[f(\dot q)=L(t,q,\dot q),\quad\dot q\in(\R^3)^n.\]
The differential of $f$ at a point $\dot q$ is the partial differential $\frac{\partial L}{\partial\dot q}(t,q,\dot q)$
of $L$ with respect to its third variable and it is an element of the dual space ${(\R^3)^n}^*$; identifying ${(\R^3)^n}^*$ with
$(\R^3)^n$ in the standard way, we obtain:
\[\dd f(\dot q)=\big(\tfrac{\partial L}{\partial\dot q_1}(t,q,\dot q),\ldots,
\tfrac{\partial L}{\partial\dot q_n}(t,q,\dot q)\big)=(m_1\dot q_1,\ldots,m_n\dot q_n)\in(\R^3)^n,\]
i.e., $\dd f(\dot q)$ is the vector $p=(p_1,\ldots,p_n)$ containing the momenta of all the particles
(more precisely, $p$ becomes the vector containing the momenta of all the particles when $\dot q$ is replaced with
the derivative $\dot q(t)$ of the curve $t\mapsto q(t)$ describing the trajectories of all the particles). We have:
\[\dd f(\dot q)\dot q=\frac{\partial L}{\partial\dot q}(t,q,\dot q)\dot q=\sum_{j=1}^nm_j\Vert\dot q_j\Vert^2\]
and therefore the map $\phi$ corresponding to $f$ as in \eqref{eq:defphiLeg} is given by:
\[\phi(\dot q)=\frac{\partial L}{\partial\dot q}(t,q,\dot q)\dot q-L(t,q,\dot q)
=\sum_{j=1}^n\frac12\,m_j\Vert\dot q_j\Vert^2+V(t,q).\]
Thus, $\phi$ yields the total energy (after $q$ is replaced with $q(t)$ and $\dot q$ is replaced with $\dot q(t)$).
The Legendre transform $f^*=\phi\circ(\dd f)^{-1}$ is just the map $\phi$ written in terms of $p$, instead of $\dot q$:
\[f^*(p)=\sum_{j=1}^n\frac{\Vert p_j\Vert^2}{2m_j}+V(t,q).\]
Setting $H(t,q,p)=f^*(p)$ (with $f$ defined by $f(\dot q)=L(t,q,\dot q)$), we see that $H$ is precisely the Hamiltonian \eqref{eq:defH} of Classical Mechanics.
\end{example}

This is the first good news: the Legendre transform turns the Lagrangian of Classical Mechanics into
the Hamiltonian of Classical Mechanics. In a moment we are going to see what happens in the case of Classical
Mechanics with constraints (Subsection~\ref{sub:constraintsagain}).
But, first, let us prove a few more properties of the Legendre transform.

If $f:\Dom(f)\subset E\to\R$ is a hyper-regular map of class $C^2$ then the differential of its Legendre transform
$f^*:\Dom(f^*)\subset E^*\to\R$ is a continuous map:
\begin{equation}\label{eq:dfstar}
\dd f^*:\Dom(f^*)\subset E^*\to E^{**}
\end{equation}
taking values in the bidual space $E^{**}$. We have the following:
\begin{lem}\label{thm:dfdfstar}
Under the standard identification of $E^{**}$ with $E$, the differential \eqref{eq:dfstar} of the Legendre transform
$f^*$ of a hyper-regular map $f:\Dom(f)\subset E\to\R$ of class $C^2$ is equal to the inverse $(\dd f)^{-1}$ of the differential
of the map $f$.
\end{lem}
\begin{proof}
Let $\alpha\in\Dom(f^*)=\Img(\dd f)$ be given and let $x\in\Dom(f)$ be such that $\dd f(x)=\alpha$. We have to show
that $\dd f^*(\alpha)$ is the element of $E^{**}$ that is identified with $x$, i.e., that:
\begin{equation}\label{eq:dfstaralphathesis}
\dd f^*(\alpha)\dot\alpha=\dot\alpha(x),
\end{equation}
for all $\dot\alpha\in E^*$. Since $f^*=\phi\circ(\dd f)^{-1}$ (with $\phi$ defined in \eqref{eq:defphiLeg}), we have:
\begin{equation}\label{eq:dfstartalpha}
\dd f^*(\alpha)\dot\alpha=\dd\phi(x)\big[\dd\big((\dd f)^{-1}\big)(\alpha)\dot\alpha\big].
\end{equation}
Given $\dot x\in E$, we differentiate both sides of \eqref{eq:defphiLeg} with respect to $x$ and evaluate at $\dot x$
(for the differentiation of the term $\dd f(x)x$ we use the product rule\footnote{%
Here is how you do this: if $(v,w)\mapsto v\cdot w$ denotes any bilinear map, then the differential of a map
$p(x)=v(x)\cdot w(x)$ is given by:
\[\dd p(x)\dot x=\big(\dd v(x)\dot x\big)\cdot w(x)+v(x)\cdot\big(\dd w(x)\dot x\big).\]
In order to prove that, think about $p$ as the composite of the map $x\mapsto\big(v(x),w(x)\big)$ with the bilinear
map that is denoted by the dot; use the chain rule and the standard formula for the differential of a bilinear map.
In the case under consideration, $v$ is the map $\dd f$, $w$ is the identity map $x\mapsto x$ and the bilinear map
$E^*\times E\to\R$ is the evaluation map.}). The result is:
\begin{equation}\label{eq:dphix}
\dd\phi(x)\dot x=\big(\dd f(x)\dot x\big)x+\dd f(x)\dot x-\dd f(x)\dot x=\big(\dd f(x)\dot x\big)x.
\end{equation}
Now set $\dot x=\dd\big((\dd f)^{-1}\big)(\alpha)\dot\alpha$ in \eqref{eq:dphix} in order to compute the righthand
side of \eqref{eq:dfstartalpha}. Notice that, since the maps $\dd f$ and $(\dd f)^{-1}$ are mutually inverse,
the differentials $\dd f(x)$ and $\dd\big((\dd f)^{-1}\big)(\alpha)$ are also mutually inverse, so that:
\[\dd f(x)\dot x=\dd f(x)\big[\dd\big((\dd f)^{-1}\big)(\alpha)\dot\alpha\big]=\dot\alpha.\]
It follows that \eqref{eq:dfstaralphathesis} holds.
\end{proof}

\begin{cor}
If $f$ is a hyper-regular map of class $C^2$ then so is its Legendre transform.
\end{cor}
\begin{proof}
Since $\dd f$ is a diffeomorphism of class $C^1$, so is its inverse $(\dd f)^{-1}$, which, by the lemma,
is identified with $\dd f^*$. It follows that $f^*$ is a map of class $C^2$ and that $f^*$ is hyper-regular.
\end{proof}

Since the Legendre transform of a hyper-regular map of class $C^2$ is again a hyper-regular map of class $C^2$,
it makes sense to take the Legendre transform $f^{**}$ of the Legendre transform $f^*$. It turns out that $f^{**}$
is just $f$.
\begin{prop}\label{thm:Leginvolutive}
Under the standard identification of $E^{**}$ with $E$, the Legendre transform $f^{**}$ of the Legendre transform
$f^*$ of a hyper-regular map $f:\Dom(f)\subset E\to\R$ of class $C^2$ is equal to $f$ (i.e.,
the Legendre transform is {\em involutive}).
\end{prop}
\begin{proof}
Let $\psi:\Dom(f^*)\subset E^*\to\R$ be the analogue of the map \eqref{eq:defphiLeg} for $f^*$, i.e., $\psi$ is defined by:
\[\psi(\alpha)=\dd f^*(\alpha)\alpha-f^*(\alpha),\quad\alpha\in\Dom(f^*)=\Img(\dd f).\]
The double Legendre transform $f^{**}$ is equal to $\psi\circ(\dd f^*)^{-1}$. Let $x\in\Dom(f)$ be fixed and set $\alpha=\dd f(x)$.
By Lemma~\ref{thm:dfdfstar}, $(\dd f^*)^{-1}=\dd f$ (up to the identification between $E^{**}$ and $E$) and therefore:
\[f^{**}(x)=\psi(\alpha).\]
Since $\dd f^*=(\dd f)^{-1}$, we obtain that $\dd f^*(\alpha)$ is equal to (the element of $E^{**}$ that is identified with) $x$ and therefore:
\[\psi(\alpha)=\dd f^*(\alpha)\alpha-f^*(\alpha)=\alpha(x)-f^*(\alpha)=\alpha(x)-\phi(x),\]
with $\phi$ defined in \eqref{eq:defphiLeg}. But $\phi(x)=\alpha(x)-f(x)$ and the conclusion follows.
\end{proof}

Now we are ready to study the Legendre transform of a Lagrangian. Let $Q$ be a differentiable manifold and $L$ be a Lagrangian on $Q$, i.e.,
$L$ is a smooth real valued map defined over an open subset $\Dom(L)$ of $\R\times TQ$. For each $(t,q)\in\R\times Q$, we obtain a map $L(t,q,\cdot)$
by freezing the first two variables of $L$, i.e., $L(t,q,\cdot)$ is the map $\dot q\mapsto L(t,q,\dot q)$ defined over the open subset:
\[\Dom\!\big(L(t,q,\cdot)\big)=\big\{\dot q\in T_qQ:(t,q,\dot q)\in\Dom(L)\big\}\]
of the tangent space $T_qQ$. The differential of the map $L(t,q,\cdot)$ at a point $\dot q\in\Dom\!\big(L(t,q,\cdot)\big)$ will be denoted by
$\frac{\partial L}{\partial\dot q}(t,q,\dot q)$ and it is an element of the cotangent space $T_qQ^*$.

\begin{defin}
The Lagrangian $L$ is said to be {\em regular\/} (resp., {\em hyper-regular}) if the map $L(t,q,\cdot)$ is regular (resp., hyper-regular) for all
$(t,q)$ in $\R\times Q$.
\end{defin}
In other words, the Lagrangian $L$ is regular if the map:
\begin{equation}\label{eq:delLdeldotq}
T_qQ\supset\Dom\!\big(L(t,q,\cdot)\big)\ni\dot q\longmapsto\frac{\partial L}{\partial\dot q}(t,q,\dot q)\in T_qQ^*
\end{equation}
is a local diffeomorphism for all $(t,q)\in\R\times Q$ and the Lagrangian $L$ is hyper-regular if the map \eqref{eq:delLdeldotq} is
a diffeomorphism onto an open subset of $T_qQ^*$, for all $(t,q)\in\R\times Q$. We can join all the maps \eqref{eq:delLdeldotq} with $(t,q)$ running
over $\R\times Q$, obtaining the so called {\em fiber derivative\/} of $L$, which is the map:
\[\mathbb FL:\Dom(L)\subset\R\times TQ\longrightarrow\R\times TQ^*\]
defined by:
\[\mathbb FL(t,q,\dot q)=\Big(t,q,\frac{\partial L}{\partial\dot q}(t,q,\dot q)\Big),\quad(t,q,\dot q)\in\Dom(L).\]

The conditions of regularity and hyper-regularity of $L$ can be stated in terms of the fiber derivative $\mathbb FL$.
\begin{lem}\label{thm:LregFLdiff}
The Lagrangian $L$ is regular if and only if its fiber derivative $\mathbb FL$ is a local diffeomorphism and the Lagrangian $L$ is hyper-regular
if and only if its fiber derivative $\mathbb FL$ is a diffeomorphism onto an open subset of $\R\times TQ^*$ (such open subset is obviously
the image of $\mathbb FL$).
\end{lem}
\begin{proof}
It is obvious that $\mathbb FL$ is injective if and only if the map \eqref{eq:delLdeldotq} is injective for all $(t,q)\in\R\times Q$. Since
a smooth map is a diffeomorphism onto an open subset of its counter-domain if and only if the map is an injective local diffeomorphism, it follows that
in order to prove the lemma it suffices to prove the first part of its statement, i.e., to prove that $L$ is regular if and only if $\mathbb FL$ is a local
diffeomorphism. We employ the strategy discussed at the end of Section~\ref{sec:canformscot}. Observe that:
\begin{itemize}
\item[(a)] if two manifolds $Q$, $\widetilde Q$ are diffeomorphic and if the thesis (that a Lagrangian $L$ on the manifold is regular if and only if the map
$\mathbb FL$ is a local diffeomorphism) holds for the manifold $\widetilde Q$ then it also holds for the manifold $Q$ (if this is not obvious to you, see
Exercise~\ref{exe:lemregdiff});
\item[(b)] if every point of $Q$ has an open neighborhood $U$ in $Q$ such that the thesis holds for the manifold $U$ then the thesis holds for $Q$.
\end{itemize}
Because of (a) and (b), it suffices to prove the thesis if $Q$ is an open subset of $\R^n$. In that case, the thesis follows easily from the
inverse function theorem (see Exercise~\ref{exe:invfuncteo}).
\end{proof}

\begin{defin}
If $L:\Dom(L)\subset\R\times TQ\to\R$ is a hyper-regular Lagrangian, then its {\em Legendre transform\/} is the map:
\[L^*:\Dom(L^*)=\Img(\mathbb FL)\subset\R\times TQ^*\longrightarrow\R\]
defined by:
\[L^*(t,q,p)=\big(L(t,q,\cdot)\big)^*(p),\quad(t,q,p)\in\Dom(L^*)=\Img(\mathbb FL)\subset\R\times TQ^*,\]
where $\big(L(t,q,\cdot)\big)^*(p)$ denotes the value at the point $p\in T_qQ^*$ of the Legendre transform $\big(L(t,q,\cdot)\big)^*$
of the map $L(t,q,\cdot)$. The map $L^*$ is also denoted by $H$ and it is also called the {\em Hamiltonian\/} associated to the Lagrangian $L$.
\end{defin}
It follows straightforwardly from the definition above that the Hamiltonian $H:\Dom(H)=\Dom(L^*)\subset\R\times TQ^*\to\R$ associated to a hyper-regular Lagrangian
$L$ satisfies:
\[H\big(\mathbb FL(t,q,\dot q)\big)=H\Big(t,q,\frac{\partial L}{\partial\dot q}(t,q,\dot q)\Big)=
\frac{\partial L}{\partial\dot q}(t,q,\dot q)\dot q-L(t,q,\dot q),\]
for all $(t,q,\dot q)\in\Dom(L)$, i.e., the Hamiltonian $H$ is the composition of the map:
\begin{equation}\label{eq:phimanifold}
\R\times TQ\supset\Dom(L)\ni(t,q,\dot q)\longmapsto\frac{\partial L}{\partial\dot q}(t,q,\dot q)\dot q-L(t,q,\dot q)\in\R
\end{equation}
with the inverse of the map $\mathbb FL$. Lemma~\ref{thm:LregFLdiff} implies that the domain of $H$ is open and that $H$ is smooth
(since the map \eqref{eq:phimanifold} is clearly smooth). We can also define a {\em fiber derivetive}:
\[\mathbb FH:\Dom(H)\subset\R\times TQ^*\longrightarrow\R\times TQ\]
for $H$ by setting:
\[\mathbb FH(t,q,p)=\Big(t,q,\frac{\partial H}{\partial p}(t,q,p)\Big)\in\R\times TQ,\quad(t,q,p)\in\Dom(H).\]

The following proposition is an immediate consequence of the results that we have proven about the Legendre transform of maps on real finite-dimensional
vector spaces.
\begin{prop}\label{thm:LHproperties}
Let $L:\Dom(L)\subset\R\times TQ\to\R$ be a hyper-regular Lagrangian and $H:\Dom(H)\subset\R\times TQ^*\to\R$ be its Legendre transform. Then:
\begin{itemize}
\item[(a)] for each $(t,q)\in\R\times Q$, the map $H(t,q,\cdot)$ is hyper-regular and its differential:
\[T_qQ^*\supset\Dom\!\big(H(t,q,\cdot)\big)\ni p\longmapsto\frac{\partial H}{\partial p}(t,q,p)\in\Dom\!\big(L(t,q,\cdot)\big)\subset T_qQ\]
is the inverse of the map \eqref{eq:delLdeldotq};
\item[(b)] the fiber derivatives $\mathbb FL$, $\mathbb FH$ are mutually inverse smooth diffeomorphisms;
\item[(c)] for each $(t,q)\in\R\times Q$, the Legendre transform of the map $H(t,q,\cdot)$ is equal to the map $L(t,q,\cdot)$.
\end{itemize}
\end{prop}
\begin{proof}
Item~(a) follows from Lemma~\ref{thm:dfdfstar}, item~(b) follows from item~(a) and from Lemma~\ref{thm:LregFLdiff} and item~(c) follows from
Proposition~\ref{thm:Leginvolutive}.
\end{proof}

For a Hamiltonian that is the Legendre transform of a hyper-regular Lagrangian, Proposition~\ref{thm:firstHam} yields the following:
\begin{prop}\label{thm:pparLpardotq}
Under the conditions of Proposition~\ref{thm:LHproperties}, if a smooth curve $(q,p):I\to TQ^*$ (defined on some interval $I\subset\R$)
satisfies $\big(t,q(t),p(t)\big)\in\Dom(H)$ and:
\begin{equation}\label{eq:Hameqintr}
\frac{\dd}{\dd t}\big(q(t),p(t)\big)=\vec H\big(t,q(t),p(t)\big)
\end{equation}
for a certain $t\in I$ then $\big(t,q(t),\dot q(t)\big)\in\Dom(L)$ and:
\begin{equation}\label{eq:pparLpardotq}
p(t)=\frac{\partial L}{\partial\dot q}\big(t,q(t),\dot q(t)\big).
\end{equation}
\end{prop}
\begin{proof}
Follows immediately from Proposition~\ref{thm:firstHam} and item~(a) of Proposition~\ref{thm:LHproperties}.
\end{proof}

Proposition~\ref{thm:pparLpardotq} says that if $H$ is the Legendre transform of a hyper-regular Lagrangian $L$ then, for a solution $(q,p)$
to Hamilton's equations, the $p$ is determined from the $q$ by equality \eqref{eq:pparLpardotq}.

\medskip

Finally, we prove that when $H$ is the Legendre transform of $L$, the Euler--Lagrange equation and Hamilton's equations have the same solutions;
more precisely, a map $q$ is a solution to the Euler--Lagrange equation if and only if it is a solution
to Hamilton's equations, with $p$ defined by \eqref{eq:pparLpardotq}. The statement that a curve $q$ on a manifold
$Q$ be a solution to the Euler--Lagrange equation is to be understood in terms of coordinate charts. Recall from
Section~\ref{sec:Lagrmanifolds} that if $\varphi:U\subset Q\to\widetilde U\subset\R^n$ is a local chart on $Q$ and $L:\Dom(L)\subset\R\times TQ\to\R$
is a Lagrangian on $Q$ then the representation of $L$ with respect to $\varphi$ is the Lagrangian $L_\varphi$
with domain ($\Id$ denotes the identity map of $\R$):
\[\Dom(L_\varphi)=(\Id\times\dd\varphi)\big(\!\Dom(L)\cap(\R\times TU)\big)\subset\R\times T\widetilde U=
\R\times\widetilde U\times\R^n\]
defined by:
\[L_\varphi\big(t,\varphi(q),\dd\varphi_q(\dot q)\big)=L(t,q,\dot q),\]
for all $(t,q,\dot q)\in\Dom(L)\cap(\R\times TU)$.
\begin{teo}\label{thm:ELHam}
Under the conditions of Proposition~\ref{thm:LHproperties}, assume that we are given a smooth curve $q:I\to Q$ (defined over some interval
$I\subset\R$) and a local chart $\varphi:U\subset Q\to\widetilde U\subset\R^n$. For those $t\in I$ with $\big(t,q(t),\dot q(t)\big)\in\Dom(L)$,
define $p(t)\in T_{q(t)}Q^*$ by equality \eqref{eq:pparLpardotq}. If $L_\varphi$ denotes the representation of $L$
with respect to the chart $\varphi$ then, for all $t\in I$ with $q(t)\in U$ and $\big(t,q(t),\dot q(t)\big)\in\Dom(L)$,
the following conditions are equivalent:
\begin{itemize}
\item[(a)] the Euler--Lagrange equation:
\begin{equation}\label{eq:ELchart2}
\frac{\dd}{\dd t}\frac{\partial L_\varphi}{\partial\dot q}\big(t,\tilde q(t),\dot{\tilde q}(t)\big)=
\frac{\partial L_\varphi}{\partial q}\big(t,\tilde q(t),\dot{\tilde q}(t)\big)
\end{equation}
holds, where $\tilde q=\varphi\circ q\vert_{q^{-1}(U)}$;
\medskip
\item[(b)] equality \eqref{eq:Hameqintr} (Hamilton's equations) holds.
\end{itemize}
\end{teo}

Let us first prove the equivalence between the Euler--Lagrange equation and Hamilton's equations when $Q=\R^n$.
Theorem~\ref{thm:ELHam} will then follow easily.
\begin{lem}\label{thm:ELHamlemma}
Let $L:\Dom(L)\subset\R\times\R^n\times\R^n\to\R$ be a hyper-regular Lagrangian
and $H:\Dom(H)\subset\R\times\R^n\times{\R^n}^*\to\R$ be its Legendre transform. Let $q:I\to\R^n$ be
a smooth curve (defined over some interval $I\subset\R$) and, for all $t\in I$ with $\big(t,q(t),\dot q(t)\big)\in\Dom(L)$,
define $p(t)$ by equality \eqref{eq:pparLpardotq}. Then, for all $t\in I$ with $\big(t,q(t),\dot q(t)\big)\in\Dom(L)$,
we have:
\begin{equation}\label{eq:ELagain}
\frac{\dd}{\dd t}\frac{\partial L}{\partial\dot q}\big(t,q(t),\dot q(t)\big)=\frac{\partial L}{\partial q}\big(t,q(t),\dot q(t)\big)
\end{equation}
if and only if:
\begin{equation}\label{eq:Hamyetagain}
\frac{\dd q}{\dd t}(t)=\frac{\partial H}{\partial p}\big(t,q(t),p(t)\big),\quad
\frac{\dd p}{\dd t}(t)=-\frac{\partial H}{\partial q}\big(t,q(t),p(t)\big).
\end{equation}
\end{lem}
\begin{proof}
Given $t\in I$, since the map $H\big(t,q(t),\cdot\big)$ is the Legendre transform of the map $L\big(t,q(t),\cdot\big)$,
it follows from Lemma~\ref{thm:dfdfstar} that their differentials are mutually inverse. Thus
\eqref{eq:pparLpardotq} implies that the first equation in \eqref{eq:Hamyetagain} is satisfied. Moreover,
\eqref{eq:pparLpardotq} implies that the lefthand side of the second equation in \eqref{eq:Hamyetagain} equals the
lefthand side of the Euler--Lagrange equation \eqref{eq:ELagain}. Thus, the proof will be
completed once we check that:
\[\frac{\partial H}{\partial q}\big(t,q(t),p(t)\big)=-\frac{\partial L}{\partial q}\big(t,q(t),\dot q(t)\big).\]
By the definition of the Legendre transform (we have used above $q$, $\dot q$ for names of curves and now
we are going to use them also for names of points):
\begin{equation}\label{eq:HPhiLvarphi}
H\Big(t,q,\frac{\partial L}{\partial\dot q}(t,q,\dot q)\Big)=
\frac{\partial L}{\partial\dot q}(t,q,\dot q)\dot q-L(t,q,\dot q),
\end{equation}
for all $(t,q,\dot q)\in\Dom(L)$. Differentiating both sides of \eqref{eq:HPhiLvarphi} with respect to $q$ and
evaluating at a vector $v\in\R^n$, we obtain:
\begin{multline*}
\frac{\partial H}{\partial q}\big(t,q,p)v
+\frac{\partial H}{\partial p}(t,q,p)\Big(\frac{\partial^2L}{\partial q\partial\dot q}(t,q,\dot q)v\Big)\\
=\Big(\frac{\partial^2L}{\partial q\partial\dot q}(t,q,\dot q)v\Big)\dot q
-\frac{\partial L}{\partial q}(t,q,\dot q)v,
\end{multline*}
where $p=\frac{\partial L}{\partial\dot q}(t,q,\dot q)$. Since the differentials of the maps
$L(t,q,\cdot)$ and $H(t,q,\cdot)$ are mutually inverse, we obtain that $\frac{\partial H}{\partial p}(t,q,p)$
is (the element of the bidual of $\R^n$ that is identified with the vector) $\dot q\in\R^n$ and therefore:
\[\frac{\partial H}{\partial p}(t,q,p)\Big(\frac{\partial^2L}{\partial q\partial\dot q}(t,q,\dot q)v\Big)
=\Big(\frac{\partial^2L}{\partial q\partial\dot q}(t,q,\dot q)v\Big)\dot q.\]
The conclusion follows.
\end{proof}

\begin{proof}[Proof of Theorem~\ref{thm:ELHam}]
Denote by $H_\Phi$ the representation of the Hamiltonian $H$ with respect to the local chart $\Phi=\dd^*\varphi$ of $TQ^*$, i.e.,
the domain of $H_\Phi$ is the image of $\Dom(H)\cap(\R\times TU^*)$ under $\Id\times\Phi$ (with $\Id$ the identity map of $\R$) and:
\[H_\Phi\big(t,\Phi(q,p)\big)=H_\Phi\big(t,\varphi(q),(\dd\varphi_q^{-1})^*(p)\big)=H(t,q,p),\]
for all $(t,q,p)\in\Dom(H)\cap(\R\times TU^*)$. Set:
\[\big(\tilde q(t),\tilde p(t)\big)=\Phi\big(q(t),p(t)\big)\in\widetilde U\times{\R^n}^*\subset\R^n\times{\R^n}^*,\]
for all $t\in q^{-1}(U)\subset I$.
By the result of item~(c) of Exercise~\ref{exe:Hamiltcoords}, equality \eqref{eq:Hameqintr} is equivalent to:
\begin{equation}\label{eq:HameqPhi}
\frac{\dd\tilde q}{\dd t}(t)=\frac{\partial H_\Phi}{\partial p}\big(t,\tilde q(t),\tilde p(t)\big),\quad
\frac{\dd\tilde p}{\dd t}(t)=-\frac{\partial H_\Phi}{\partial q}\big(t,\tilde q(t),\tilde p(t)\big).
\end{equation}
We have to show that \eqref{eq:HameqPhi} is equivalent to \eqref{eq:ELchart2}. This will follow from Lemma~\ref{thm:ELHamlemma} once we check the validity
of the following facts:
\begin{itemize}
\item[(i)] $L_\varphi$ is hyper-regular and $H_\Phi$ is the Legendre transform of $L_\varphi$;
\item[(ii)] for all $t\in q^{-1}(U)\subset I$ with $\big(t,\tilde q(t),\dot{\tilde q}(t)\big)\in\Dom(L_\varphi)$, we have:
\[\tilde p(t)=\frac{\partial L_\varphi}{\partial\dot q}\big(t,\tilde q(t),\dot{\tilde q}(t)\big).\]
\end{itemize}
Let us check those facts. For $t\in\R$, $q\in U$, setting $\tilde q=\varphi(q)$, we have:
\begin{gather}
\label{eq:LLvarphi}L_\varphi(t,\tilde q,\cdot)\circ\dd\varphi_q=L(t,q,\cdot),\\
\label{eq:HHPhi}H_\Phi(t,\tilde q,\cdot)\circ(\dd\varphi_q^*)^{-1}=H(t,q,\cdot).
\end{gather}
Since $H(t,q,\cdot)$ is the Legendre transform of $L(t,q,\cdot)$, equalities \eqref{eq:LLvarphi}, \eqref{eq:HHPhi} and the result of
Exercise~\ref{exe:Legtransiso} imply that $H_\Phi(t,\tilde q,\cdot)$ is the Legendre transform of $L_\varphi(t,\tilde q,\cdot)$.
We have proven fact (i). Let us prove fact (ii).

Given $t\in q^{-1}(U)\subset I$ with $\big(t,\tilde q(t),\dot{\tilde q}(t)\big)\in\Dom(L_\varphi)$ (or, equivalently,
with $\big(t,q(t),\dot q(t)\big)\in\Dom(L)$), we replace $q$ with $q(t)$ and $\tilde q$ with $\tilde q(t)$
in \eqref{eq:LLvarphi}; after that, we differentiate both sides at the point $\dot q(t)$, obtaining:
\begin{equation}\label{eq:Lqpcharts}
\frac{\partial L_\varphi}{\partial\dot q}\big(t,\tilde q(t),\dot{\tilde q}(t)\big)\circ\dd\varphi_{q(t)}=\frac{\partial L}{\partial\dot q}\big(t,q(t),\dot q(t)\big).
\end{equation}
By \eqref{eq:pparLpardotq}, the righthand side of \eqref{eq:Lqpcharts} equals $p(t)$ and therefore:
\[\frac{\partial L_\varphi}{\partial\dot q}\big(t,\tilde q(t),\dot{\tilde q}(t)\big)=p(t)\circ\dd\varphi_{q(t)}^{-1}
=(\dd\varphi_{q(t)}^{-1})^*\big(p(t)\big)=\tilde p(t),\]
proving fact (ii).
\end{proof}

\subsection{Mechanics with constraints again}\label{sub:constraintsagain}
In Example~\ref{exa:LegtransLagMech} we have seen that
the Legendre transform of the Lagrangian of Classical Mechanics is precisely the Hamiltonian of Classical Mechanics.
Let us now see what happens in the presence of constraints. As in Subsection~\ref{sub:constraints} we consider
$n$ particles subject to a constraint defined by a submanifold $Q$ of $(\R^3)^n$ and to (the
forces with) a potential $V:\Dom(V)\subset\R\times(\R^3)^n\to\R$. We consider the restriction $L^{\mathrm{cons}}$
to $\R\times TQ$ of the Lagrangian $L$ of Classical Mechanics without constraints (defined in Subsection~\ref{sub:varClassMech}).
We have seen in Subsection~\ref{sub:constraints} that if the force exerted by the constraint is normal to $Q$ then
the curves $q=(q_1,\ldots,q_n)$ in $Q$ which are possible trajectories for the particles are precisely the critical
points of the action functional corresponding to $L^{\mathrm{cons}}$ (which are, in a given chart, solutions
to the Euler--Lagrange equation of the Lagrangian that represents $L^{\mathrm{cons}}$ with respect to that chart).
Let us now compute the Legendre transform $H^{\mathrm{cons}}$ of the Lagrangian $L^{\mathrm{cons}}$ (see also Exercise~\ref{exe:LegRiem}
for a slightly different approach).
Since $L^{\mathrm{cons}}$ is the restriction of $L$, given $(t,q,\dot q)\in\R\times TQ$ with $(t,q)$ in the domain
of $V$, the partial derivative:
\begin{equation}\label{eq:delLcons}
\frac{\partial L^{\hbox to 0pt{$\scriptstyle\mathrm{cons}$\hskip 0pt minus 1fil}}}{\partial\dot q}\;
(t,q,\dot q)\in T_qQ^*
\end{equation}
is simply the restriction to $T_qQ$ of the linear functional
$\frac{\partial L}{\partial\dot q}(t,q,\dot q)\in{(\R^3)^n}^*$ (which was computed in Example~\ref{exa:LegtransLagMech}).
Thus, \eqref{eq:delLcons} is the restriction to $T_qQ$ of the linear functional over $(\R^3)^n$ that is identified
(via the standard identification of $(\R^3)^n$ with its dual space) with the vector:
\[(m_1\dot q_1,\ldots,m_n\dot q_n)\in(\R^3)^n.\]
Let us check that $L^{\mathrm{cons}}$ is hyper-regular. The map:
\begin{equation}\label{eq:delLcons2}
T_qQ\ni\dot q\longmapsto\frac{\partial L^{\hbox to 0pt{$\scriptstyle\mathrm{cons}$\hskip 0pt minus 1fil}}}{\partial\dot q}\;
(t,q,\dot q)=\frac{\partial L}{\partial\dot q}(t,q,\dot q)\Big\vert_{T_qQ}\in T_qQ^*
\end{equation}
is linear (because the map $\dot q\mapsto\frac{\partial L}{\partial\dot q}(t,q,\dot q)$ is linear).
Given $\dot q\in T_qQ$, if \eqref{eq:delLcons} vanishes then:
\[\frac{\partial L^{\hbox to 0pt{$\scriptstyle\mathrm{cons}$\hskip 0pt minus 1fil}}}{\partial\dot q}\;
(t,q,\dot q)\dot q=0\]
and therefore:
\[\frac{\partial L^{\hbox to 0pt{$\scriptstyle\mathrm{cons}$\hskip 0pt minus 1fil}}}{\partial\dot q}\;
(t,q,\dot q)\dot q=\frac{\partial L}{\partial\dot q}(t,q,\dot q)\dot q=\sum_{j=1}^nm_j\Vert\dot q_j\Vert^2=0\]
which implies $\dot q=0$. This proves that the linear map \eqref{eq:delLcons2} is injective and it is therefore
an isomorphism (and a diffeomorphism). Hence $L^{\mathrm{cons}}$ is hyper-regular. Let us compute its Legendre transform
$H^{\mathrm{cons}}$. Clearly:
\begin{equation}\label{eq:phicons}
\frac{\partial L^{\hbox to 0pt{$\scriptstyle\mathrm{cons}$\hskip 0pt minus 1fil}}}{\partial\dot q}\;
(t,q,\dot q)\dot q-L^{\mathrm{cons}}(t,q,\dot q)=\frac{\partial L}{\partial\dot q}(t,q,\dot q)\dot q-L(t,q,\dot q),
\end{equation}
for every $(t,q,\dot q)\in\R\times TQ$ with $(t,q)\in\Dom(V)$. The righthand side of \eqref{eq:phicons} was
computed in Example~\ref{exa:LegtransLagMech}. The Hamiltonian $H^{\mathrm{cons}}$ is therefore given by:
\[H^{\mathrm{cons}}(t,q,p)=\sum_{j=1}^n\frac12\,m_j\Vert\dot q_j\Vert^2+V(t,q),\]
for all $(t,q)\in\Dom(V)$, $p\in T_qQ^*$, where $\dot q\in T_qQ$ is the unique vector such that:
\begin{equation}\label{eq:defpcons}
p=\frac{\partial L^{\hbox to 0pt{$\scriptstyle\mathrm{cons}$\hskip 0pt minus 1fil}}}{\partial\dot q}\;
(t,q,\dot q)\in T_qQ^*
\end{equation}
holds (the existence and uniqueness of $\dot q$ follows from the fact that \eqref{eq:delLcons2} is a linear
isomorphism). A solution of Hamilton's equations corresponding to $H^{\mathrm{cons}}$ (i.e., an integral
curve of the symplectic gradient of $H^{\mathrm{cons}}$ with respect to the canonical symplectic
form of the cotangent bundle $TQ^*$) is, by Proposition~\ref{thm:pparLpardotq} and Theorem~\ref{thm:ELHam},
a curve $I\ni t\mapsto\big(q(t),p(t)\big)$ in $TQ^*$ such that:
\begin{equation}\label{eq:ptHcons}
p(t)=\frac{\partial L^{\hbox to 0pt{$\scriptstyle\mathrm{cons}$\hskip 0pt minus 1fil}}}{\partial\dot q}\;
\big(t,q(t),\dot q(t)\big)
\end{equation}
for all $t\in I$ and such that the representation of $q$ with respect to any local chart of $Q$ is a solution of
the Euler--Lagrange equation corresponding to the Lagrangian that represents $L^{\mathrm{cons}}$ with respect to the given chart
(by Theorem~\ref{thm:ELmanifold}, such condition is equivalent to the condition that the restriction of $q$ to any closed interval contained in $I$ be
a critical point of the action functional corresponding to $L^{\mathrm{cons}}$).
In other words, a solution of Hamilton's equations corresponding to $H^{\mathrm{cons}}$ consists of a curve $(q,p)$ in $TQ^*$, with $q$ a curve
in $Q$ describing the trajectories of $n$ particles subject to the given constraints and the given potential (under the assumption that the force exerted by
the constraint is normal to $Q$) and with $p$ defined by \eqref{eq:ptHcons}. Notice that:
\[H^{\mathrm{cons}}\big(t,q(t),p(t)\big)=\sum_{j=1}^n\frac12\,m_j\Vert\dot q_j(t)\Vert^2+V\big(t,q(t)\big),\]
for all $t\in I$, i.e., $H^{\mathrm{cons}}\big(t,q(t),p(t)\big)$ equals the total energy at the instant $t$. By Theorem~\ref{thm:Hfirstint},
if the potential $V$ does not depend on time, we conclude that the total energy is conserved.

While it is standard to denote a solution to Hamilton's equations
by $(q,p)$, one should be aware that this notation might be misleading: in the absence of constraints (Example~\ref{exa:LegtransLagMech}), $p(t)$ was precisely the
(element of ${(\R^3)^n}^*$ that is identified with the) vector containing the momenta of all the particles. However, in the present context, $p(t)$
is the {\em restriction\/} to the tangent space $T_{q(t)}Q$ of the linear functional over $(\R^3)^n$ that is identified with the vector containing the
momenta of all the particles. Thus, in general, $p(t)$ is {\em not the momentum of anything\/}! It is customary, nevertheless, to call $p(t)$ the
{\em canonical momentum\/} or the {\em canonically conjugate momentum}. If one considers a local chart $\varphi$ on $Q$, then the
curve $(q,p)$ is represented (with respect to the local chart $\dd^*\varphi$ induced on $TQ^*$) by a curve:
\[(\tilde q,\tilde p)=(\tilde q^{\,1},\ldots,\tilde q^{\,m},\tilde p_1,\ldots,\tilde p_m)\]
in $\R^m\times{\R^m}^*$, with $m=\Dim(Q)$. Elementary Physics texts would normally call the curves $\tilde q^{\,j}$, $j=1,\ldots,m$, the description of the
evolution of the mechanical system in terms of {\em generalized coordinates}\footnote{%
In the standard notation used by elementary Physics texts for the description of the evolution of a mechanical system
in terms of generalized coordinates, there are no tildes over the $q$'s and the $p$'s. We need the tildes because we are using the $(q,p)$ without
the tildes for the curve in $TQ^*$.}.
One should pay attention to the following facts: the index $j$ in $\tilde q^{\,j}$
{\em does not\/} refer to the number of a particle; the number $m=\Dim(Q)$ is not the number of particles and
it is normally called the number of {\em degrees of freedom} of the system. The $\tilde p_j$ is not the ``momentum of the $j$-th particle'', since $p$ (and
$\tilde p$) is not momentum and $j$ does not refer to the number of a particle. The sum $\sum_{j=1}^m\tilde p_j$ is not the total momentum and in fact it
does not have any physically relevant meaning
(it is highly dependent on the choice of the coordinate chart) and it is not conserved. The actual total momentum $\sum_{j=1}^nm_j\dot q_j(t)$ is
not in general conserved either because there are external forces (the forces exerted by the constraint) acting upon the system of $n$ constrained particles.
In Section~\ref{sec:Nother} we will understand this violation of the conservation of the total momentum\footnote{%
Of course, there is no violation of the conservation of the total momentum if one considers the full (closed) system,
containing both the $n$ particles and the objects (also made out of particles) that are constraining the $n$ particles.}
as a break down of the spatial translations symmetry.

\end{section}

\begin{section}{Symmetry and conservation laws}
\label{sec:Nother}

In this section we will make a brief exposition of the relationship between symmetries and conservation laws. The result we present is a particular
case of the celebrated {\em N\"other's theorem\/} (whose most general formulation concerns not only Mechanics, but also field theories\footnote{%
We observe also that most presentations of N\"other's theorem use the Lagrangian formalism, while we are going to use the Hamiltonian formalism.}).
The idea is that to a continuous symmetry of
a mechanical system (described in terms of an element of the Lie algebra of a Lie group that acts on the configuration space and preserves the Hamiltonian)
we associate a conservation law (described in terms of a first integral of the equations
of motion defined over phase space).
The result that we are going to present is not the best that can be done. For instance, time translation
symmetry implies conservation of energy (Theorem~\ref{thm:Hfirstint}), but that important relationship between symmetry and conservation law does
not follow from the result we are going to present in this section. Nevertheless, such result is very simple and sufficient
for our limited purposes. During the presentation, we are going to use some very basic facts about Lie groups and actions of Lie groups on manifolds which, for the reader's
convenience, are recalled in the appendix (Section~\ref{sec:Liegroups}).

\medskip

Let $Q$ be a differentiable manifold and $H:\Dom(H)\subset\R\times TQ^*\to\R$ be a time-dependent Hamiltonian over $TQ^*$. Given a Lie group $G$ and a smooth
action:
\[\rho:G\times Q\ni(g,q)\longmapsto g\cdot q\in Q\]
of $G$ on $Q$ then, for each $g\in G$, the smooth diffeomorphism:
\[\rho_g:Q\ni q\longmapsto g\cdot q\in Q\]
of $Q$ induces a smooth diffeomorphism $\dd^*(\rho_g)$ of the cotangent bundle $TQ^*$ (see \eqref{eq:dstarvarphi}). For $g,h\in G$, we have
$\rho_g\circ\rho_h=\rho_{gh}$ and therefore (see Exercise~\ref{exe:dstarfunc}):
\[\dd^*(\rho_g)\circ\dd^*(\rho_h)=\dd^*(\rho_{gh}).\]
We thus obtain a smooth action of $G$ on $TQ^*$ defined by:
\[g\cdot(q,p)=\dd^*(\rho_g)(q,p)=\big(g\cdot q,\big(\dd(\rho_g)_q^{-1}\big)^{\!*}(p)\big),\quad g\in G,\ (q,p)\in TQ^*.\]
We say that the action of $G$ on $Q$ is a {\em symmetry\/} of the Hamiltonian $H$ if, for all $(t,q,p)\in\Dom(H)$ and all $g\in G$, we have
$\big(t,g\cdot(q,p)\big)\in\Dom(H)$ and:
\[H\big(t,g\cdot(q,p)\big)=H(t,q,p).\]
For each $X$ in the Lie algebra $\mathfrak g$ of $G$, the action of $G$ on $Q$ induces a vector field $X^Q$ on $Q$ and the action of $G$ on $TQ^*$ induces
a vector field $X^{TQ^*}$ on $TQ^*$ (see Subsection~\ref{sub:Lieactions}).
Recall from Section~\ref{sec:canformscot} that $\theta$ denotes the canonical one-form
of the cotangent bundle $TQ^*$. Here is the main result of the section.
\begin{teo}\label{thm:Nother}
Given a differentiable manifold $Q$, a time-depen\-dent Hamiltonian $H:\Dom(H)\subset\R\times TQ^*\to\R$ and a smooth action of a
Lie group $G$ on $Q$ that is a symmetry of $H$ then,
for all $X\in\mathfrak g$, the real valued map $\theta(X^{TQ^*})$ over $TQ^*$ is a first integral of $\vec H$, i.e., it is constant along the integral
curves of $\vec H$.
\end{teo}
\begin{proof}
It is sufficient to prove that for all $t\in\R$ the directional derivative:
\[\dd\big(\theta(X^{TQ^*})\big)(\vec H_t)=\vec H_t\big(\theta(X^{TQ^*})\big)\]
of the map $\theta(X^{TQ^*})$ along the vector field $\vec H_t=\vec H(t,\cdot)$ vanishes.
We start by observing that the canonical one-form $\theta$ of $TQ^*$ is invariant under the flow of the vector field $X^{TQ^*}$. Namely,
the flow at time $t$ of $X^{TQ^*}$ is the diffeomorphism $\dd^*(\rho_g)$ with $g=\exp(tX)$ and it follows from \eqref{eq:dstarphiprestheta} that the
pull-back of $\theta$ by $\dd^*(\rho_g)$ equals $\theta$. Thus, the Lie derivative of $\theta$ along $X^{TQ^*}$ vanishes:
\[\mathbb L_{X^{TQ^*}}\theta=0.\]
Since the action of $G$ on $Q$ is a symmetry of $H$, we see that for all $t\in\R$ the map $H_t=H(t,\cdot)$ is constant along
the integral curves of $X^{TQ^*}$ (because such integral curves are of the form $\R\ni s\mapsto\exp(sX)\cdot(q,p)\in TQ^*$) and therefore:
\begin{equation}\label{eq:dHXTQ}
X^{TQ^*}(H_t)=\dd H_t(X^{TQ^*})=0.
\end{equation}
We compute the Lie derivative of $\theta$ along $X^{TQ^*}$ in terms of exterior differentiation and interior products (formula \eqref{eq:diid}):
\[\mathbb L_{X^{TQ^*}}\theta=\dd i_{X^{TQ^*}}\theta+i_{X^{TQ^*}}\dd\theta=0,\]
and we evaluate the result at $\vec H_t$:
\begin{equation}\label{eq:calcLXtheta}
\dd\big(\theta(X^{TQ^*})\big)(\vec H_t)+\dd\theta(X^{TQ^*},\vec H_t)=0.
\end{equation}
In order to conclude the proof we have to check that the first term on the lefthand side of \eqref{eq:calcLXtheta} vanishes, which will follow
if we show that the second term vanishes. Since
$\dd\theta=-\omega$, keeping in mind the definition of the symplectic gradient $\vec H_t$, we obtain:
\[\dd\theta(X^{TQ^*},\vec H_t)=\omega(\vec H_t,X^{TQ^*})=\dd H_t(X^{TQ^*})\]
and the conclusion follows from \eqref{eq:dHXTQ}.
\end{proof}
Theorem~\ref{thm:Nother} can be easily generalized to the following result about Hamiltonians on symplectic manifolds: if $(M,\omega)$
is a symplectic manifold with exact symplectic form $\omega=-\dd\theta$ and if a Lie group $G$ acts on $M$ in such a way that the action preserves
the one-form $\theta$ and a time-dependent Hamiltonian $H$ over $M$ then, for all $X\in\mathfrak g$, the real valued map $\theta(X^{\!M})$ over $M$
is a first integral of $\vec H$. The proof of such result is identical to the proof of Theorem~\ref{thm:Nother}.
A generalization of Theorem~\ref{thm:Nother} to symplectic manifolds whose symplectic form is not exact (or to the case in which the symplectic form is
exact but the action of the Lie group preserves the symplectic form and not the corresponding one-form) is given in Exercise~\ref{exe:Nother}.

\medskip

The first integral $\theta(X^{TQ^*})$ of $\vec H$ given by Theorem~\ref{thm:Nother} can be made more explicit as shown by the following:
\begin{prop}\label{thm:Nothersimple}
If a Lie group $G$ acts smoothly on $Q$ then for all $X\in\mathfrak g$, we have:
\[\theta(X^{TQ^*})(q,p)=p\big(X^Q(q)\big),\]
for all $(q,p)\in TQ^*$.
\end{prop}
\begin{proof}
By the definition of $\theta$, we have:
\[\theta(X^{TQ^*})(q,p)=p\big[\dd\pi_{(q,p)}\big(X^{TQ^*}(q,p)\big)\big],\]
for all $(q,p)\in TQ^*$, where $\pi:TQ^*\to Q$ denotes the canonical projection.
Since $X^{TQ^*}(q,p)$ is the image of $X$ by the differential at $1\in G$ of the map $g\mapsto g\cdot(q,p)$, we have
(by the chain rule) that $\dd\pi_{(q,p)}\big(X^{TQ^*}(q,p)\big)$ equals the image of $X$ by the differential at $1\in G$ of the map
$g\mapsto\pi\big(g\cdot(q,p)\big)=g\cdot q$. Thus:
\[\dd\pi_{(q,p)}\big(X^{TQ^*}(q,p)\big)=X^Q(q)\]
and the conclusion follows.
\end{proof}

\begin{example}\label{exa:translations}
Let $V:\Dom(V)\subset\R\times(\R^3)^n\to\R$ be a smooth potential and consider the Hamiltonian:
\[H(t,q,p)=\sum_{j=1}^n\frac{\Vert p_j\Vert^2}{2m_j}+V(t,q),\quad(t,q)\in\Dom(V),\ p\in{(\R^3)^n}^*,\]
of Classical Mechanics. Let $G$ be the abelian Lie group $(\R^3,+)$, whose Lie algebra $\mathfrak g$ is $\R^3$ endowed with the identically vanishing Lie bracket.
We consider the action of $G$ on $Q=(\R^3)^n$ by {\em spatial translations}, i.e., we set:
\[g\cdot q=(g+q_1,\ldots,g+q_n),\quad g\in\R^3,\ q=(q_1,\ldots,q_n)\in(\R^3)^n.\]
The differential of the map $q\mapsto g\cdot q$ at a point $q\in Q$ is the identity map (and its inverse transpose is also the identity map)
and therefore the induced action of $G$ on the cotangent bundle $TQ^*$ is given by:
\[g\cdot(q,p)=(g\cdot q,p),\quad q\in(\R^3)^n,\ p\in{(\R^3)^n}^*.\]
Since the action of $g$ on $p$ is trivial, the kinetic term of $H$ is preserved by the action. Now, assume that the potential $V$ depends on $q$
only through the differences $q_i-q_j$, $i,j=1,\ldots,n$; this is the case of the electrical and gravitational potentials. Under such assumption the action of $G$
is a symmetry of the potential and thus also of the Hamiltonian. Let us determine the conservation law associated to an element $X\in\mathfrak g=\R^3$.
The value at $q\in(\R^3)^n$ of the vector field $X^Q$ is obtained by differentiating $g\mapsto g\cdot q$ at $g=0$ and evaluating at the vector $X$; thus:
\[X^Q(q)=(X,\ldots,X)\in(\R^3)^n.\]
By Proposition~\ref{thm:Nothersimple}, the conserved quantity $\theta(X^{TQ^*})$ is given by:
\[TQ^*\ni(q,p)\longmapsto p\big(X^Q(q)\big)=\sum_{j=1}^np_j(X)\in\R.\]
Replacing $X$ with the $k$-th vector of the canonical basis of $\R^3$ ($k=1,2,3$), we obtain that {\em the $k$-th component of the total momentum is conserved
if spatial translations are a symmetry of the Hamiltonian $H$}.
\end{example}

\begin{example}\label{exa:angmom}
Consider again the Hamiltonian of Classical Mechanics with a potential $V$ (as in Example~\ref{exa:translations}) and let now $G=\SO(3)$ be the Lie group of all orientation
preserving linear isometries of $\R^3$ (or, equivalently, the group of $3\times3$ orthogonal matrices whose determinant is equal to $1$).
We consider the action of $G$ on $Q=(\R^3)^n$ by {\em spatial rotations}, i.e., we set:
\[g\cdot q=\big(g(q_1),\ldots,g(q_n)\big),\quad g\in\SO(3),\ q=(q_1,\ldots,q_n)\in(\R^3)^n.\]
The map $q\mapsto g\cdot q$ is linear and thus its differential at any point equals itself; moreover, since $q\mapsto g\cdot q$ is a linear isometry of
$(\R^3)^n$ (endowed with its canonical inner product) then the transpose inverse of $q\mapsto g\cdot q$ equals itself (upon identification of
$(\R^3)^n$ with its dual space using the canonical inner product). Thus, the action of $G$ induced on the cotangent bundle $TQ^*$ is:
\[g\cdot(q,p)=(g\cdot q,g\cdot p),\quad g\in\SO(3),\ q\in(\R^3)^n,\ p\in(\R^3)^n\cong{(\R^3)^n}^*.\]
Since each $g\in G$ is a linear isometry of $\R^3$, the action of $G$ preserves the kinetic term of the Hamiltonian. Now, assuming that the potential
$V$ depends on $q$ only through the {\em norms\/} of the differences $q_i-q_j$ (which is the case of the electric and the gravitational potentials) then the action of
$G$ is also a symmetry of the potential and hence of the Hamiltonian. The Lie algebra $\mathfrak g=\so(3)$ of $G$ is the Lie algebra of anti-symmetric
linear endomorphisms of $\R^3$ (or, equivalently, of $3\times3$ anti-symmetric matrices) endowed with the standard commutator. Let us compute
the conserved quantity corresponding to an element $X\in\so(3)$. Given $q\in(\R^3)^n$ we differentiate the map $g\mapsto g\cdot q$ at the identity
and we evaluate it at $X$, obtaining:
\[X^Q(q)=\big(X(q_1),\ldots,X(q_n)\big),\quad q\in(\R^3)^n.\]
Thus, by Proposition~\ref{thm:Nothersimple}, the conserved quantity corresponding to $X$ is:
\[TQ^*\ni(q,p)\longmapsto p\big(X^Q(q)\big)=\sum_{j=1}^np_j\big(X(q_j)\big)\in\R.\]
Let us rewrite such conserved quantity in a nicer way. The Lie algebra of $\SO(3)$ can be identified with $\R^3$ endowed with the vector product
(see Exercise~\ref{exa:vecproduct}); more explicitly, given $X\in\so(3)$, there exists a unique $v\in\R^3$ such that:
\[X(w)=v\wedge w,\quad w\in\R^3,\]
where $\wedge$ denotes the vector product. Thus (identifying $p_j\in{\R^3}^*$ with a vector of $\R^3$):
\[p_j\big(X(q_j)\big)=\langle p_j,v\wedge q_j\rangle=\langle v,q_j\wedge p_j\rangle,\]
so that the conserved quantity associated to $v$ is:
\begin{equation}\label{eq:angmom}
TQ^*\ni(q,p)\longmapsto\sum_{j=1}^n\langle v,q_j\wedge p_j\rangle\in\R.
\end{equation}
This motivates the following:
\begin{defin}
The {\em angular momentum\/} of the $j$-th particle at time $t\in\R$ is defined by:
\[L_j(t)=q_j(t)\wedge p_j(t)=m_j\big(q_j(t)\wedge\dot q_j(t)\big).\]
\end{defin}
Replacing $v$ with the $k$-th vector of the canonical basis of $\R^3$ ($k=1,2,3$), we obtain that {\em the $k$-th component of the total angular momentum}:
\[\sum_{j=1}^nL_j(t)=\sum_{j=1}^nq_j(t)\wedge p_j(t)\]
{\em is conserved if spatial rotations are a symmetry of the Hamiltonian $H$}.
\end{example}

\end{section}

\begin{section}{The Poisson bracket}

In this section we define the Poisson bracket, which is a binary operation on the space of smooth real valued
maps over a symplectic manifold. We will need the Poisson bracket for our forthcoming discussion of {\em quantization}.
The concept of Poisson bracket allows one to establish some nice algebraic analogies between Classical and Quantum
Mechanics (in fact, such analogies are somewhat misleading, but interesting nevertheless). We will
prove some simple properties of the Poisson bracket, which should allow the reader to have an idea of the relevance
of the concept for Classical Mechanics.

In what follows, $(M,\omega)$ denotes a fixed symplectic manifold and $C^\infty(M)$ denotes the vector space of all smooth real valued maps
over $M$.

\begin{defin}
Given $f,g\in C^\infty(M)$, then the {\em Poisson bracket\/} $\{f,g\}$ is the element of $C^\infty(M)$ defined by:
\[\{f,g\}=\omega(\vec f,\vec g\microspace),\]
where $\vec f$, $\vec g$ denote the symplectic gradients of $f$ and $g$, respectively (recall Definition~\ref{thm:defsymplgrad}).
\end{defin}

Here is an alternative definition of the Poisson bracket: for each $x$ in $M$, the symplectic form $\omega_x$
over the tangent space $T_xM$ induces a linear isomorphism:
\[T_xM\ni v\longmapsto\omega_x(v,\cdot)\in T_xM^*,\]
which is precisely the isomorphism that carries the symplectic gradient $\vec f(x)$ to the differential $\dd f(x)$.
Such isomorphism can be used to carry the symplectic form $\omega_x$ over $T_xM$ to a symplectic form $\pi_x$ over the dual
space $T_xM^*$, i.e., $\pi_x$ satisfies:
\[\pi_x\big(\omega_x(v,\cdot),\omega_x(w,\cdot)\big)=\omega_x(v,w),\quad v,w\in T_xM.\]
We have that $\pi$ is a smooth anti-symmetric $(0,2)$-tensor field (or twice contravariant tensor field)
over $M$ (see Subsection~\ref{sub:TensorForms}). Obviously, the Poisson bracket of two maps $f,g\in C^\infty(M)$ is given by:
\[\{f,g\}=\pi(\dd f,\dd g).\]

The direct relationship between the Poisson bracket and Classical Mechanics is that the Poisson bracket can be used to describe the time evolution of
the value of a smooth function $f:M\to\R$ along the flow of a Hamiltonian.
\begin{prop}\label{thm:derfvecH}
Let $H:\Dom(H)\subset\R\times M\to\R$ be a time-dependent Hamiltonian and let $t\mapsto x(t)$ be an integral
curve of $\vec H$. Then, given a smooth map $f:M\to\R$, we have:
\begin{equation}\label{eq:brackfH}
\frac{\dd}{\dd t}f\big(x(t)\big)=\{f,H_t\}\big(x(t)\big),
\end{equation}
where $H_t=H(t,\cdot)$.
\end{prop}

Notice that the map $H_t$ in the statement of Proposition~\ref{thm:derfvecH}
is defined only over some open subset of $M$, so that, to be completely precise,
the Poisson bracket in \eqref{eq:brackfH} is the Poisson bracket between {\em the restriction\/} of $f$ to the domain of $H_t$
and the map $H_t$. Of course, the Poisson bracket is also defined for maps whose domain is an open subset of the
symplectic manifold $M$ (in fact, open subsets of a symplectic manifold are themselves symplectic manifolds,
endowed with the restriction of the symplectic form).

\begin{proof}[Proof of Proposition~\ref{thm:derfvecH}]
It is a straightforward computation:
\[\frac{\dd}{\dd t}f\big(x(t)\big)=\dd f_{x(t)}\big[\vec H\big(t,x(t)\big)\big]=
\omega_{x(t)}\big[\vec f\big(x(t)\big),\vec H_t\big(x(t)\big)\big].\qedhere\]
\end{proof}

\begin{cor}
A smooth map $f:M\to\R$ is a first integral of the symplectic gradient $\vec H$ of a time-dependent Hamiltonian
$H$ if and only if the Poisson bracket $\{f,H_t\}$ vanishes, for all $t\in\R$.\qed
\end{cor}

Poisson brackets are also useful for writing down in a nice way the condition that a local chart be symplectic.
\begin{prop}
Let $\Phi:U\subset M\to\widetilde U\subset\R^{2n}$ be a local chart on $M$; write $\Phi=(q^1,\ldots,q^n,p^1,\ldots,p^n)$. The local chart $\Phi$
is symplectic if and only if:
\[\{q^i,q^j\}=0,\quad\{p^i,p^j\}=0,\quad\{q^i,p^j\}=\delta^{ij},\quad i,j=1,\ldots,n,\]
where $\delta^{ij}=1$ for $i=j$ and $\delta^{ij}=0$ for $i\ne j$.
\end{prop}
\begin{proof}
For $x\in U$, let:
\begin{equation}\label{eq:basischartqp}
\frac{\partial}{\partial q^1}(x),\ldots,\frac{\partial}{\partial q^n}(x),\frac{\partial}{\partial p^1}(x),\ldots,\frac{\partial}{\partial p^n}(x),
\end{equation}
denote the basis of $T_xM$ that is carried by $\dd\Phi(x)$ to the canonical basis of $\R^{2n}$. The dual basis of \eqref{eq:basischartqp} is:
\begin{equation}\label{eq:dualbasisqp}
\dd q^1(x),\ldots,\dd q^n(x),\dd p^1(x),\ldots,\dd p^n(x).
\end{equation}
The chart $\Phi$ is symplectic if and only if \eqref{eq:basischartqp} is a symplectic basis of $(T_xM,\omega_x)$, for all $x\in U$.
By the result of Exercise~\ref{exe:symplomegapi}, the basis \eqref{eq:basischartqp} is symplectic for $\omega_x$ if and only if the dual basis
\eqref{eq:dualbasisqp} is symplectic for $\pi_x$, which happens (for all $x\in U$) if and only if:
\begin{gather*}
\pi(\dd q^i,\dd q^j)=\{q^i,q^j\}=0,\quad\pi(\dd p^i,\dd p^j)=\{p^i,p^j\}=0,\\
\pi(\dd q^i,\dd p^j)=\{q^i,p^j\}=\delta^{ij},
\end{gather*}
for all $i,j=1,\ldots,n$.
\end{proof}

Let us now investigate some algebraic properties of the Poisson bracket and its relationship with other operations
defined on manifolds. We start by noticing that, for $f,g\in C^\infty(M)$, we have:
\[\{f,g\}=\omega(\vec f,\vec g\microspace)=-\omega(\vec g,\vec f\,)=-\dd g(\vec f\,)=-\vec f(g).\]
In other words, if we identify the vector field $\vec f$ with the linear endomorphism of $C^\infty(M)$ that sends
$g$ to $\vec f(g)=\dd g(\vec f\,)$ then:
\begin{equation}\label{eq:Poissonvec}
\{f,\cdot\}=-\vec f,
\end{equation}
where $\{f,\cdot\}$ denotes the linear endomorphism of $C^\infty(M)$ that sends $g$ to $\{f,g\}$. It follows that $\{f,\cdot\}$ is a {\em derivation\/}
of the algebra $C^\infty(M)$, i.e.:
\begin{equation}\label{eq:Poifderiv}
\{f,g_1g_2\}=\{f,g_1\}g_2+g_1\{f,g_2\},\quad f,g_1,g_2\in C^\infty(M).
\end{equation}

Another interesting algebraic property of the Poisson bracket is given by the following theorem.
\begin{teo}
The real vector space $C^\infty(M)$ endowed with the Poisson bracket is a Lie algebra, i.e., the Poisson bracket is bilinear, anti-symmetric and
satisfies the {\em Jacobi identity}:
\begin{equation}\label{eq:JacobiPoisson}
\{f,\{g,h\}\}+\{g,\{h,f\}\}+\{h,\{f,g\}\}=0,\quad f,g,h\in C^\infty(M).
\end{equation}
\end{teo}
\begin{proof}
The only non trivial part of the statement is the Jacobi identity. Since the symplectic form $\omega$ is closed, we have:
\begin{equation}\label{eq:omegafgh}
\dd\omega(\vec f,\vec g,\vec h)=0.
\end{equation}
We use formula \eqref{eq:Cartanextdiff} for computing the lefthand side of \eqref{eq:omegafgh}:
\begin{multline}\label{eq:computeomegafgh}
\vec f\big(\omega(\vec g,\vec h)\big)-\vec g\big(\omega(\vec f,\vec h)\big)+\vec h\big(\omega(\vec f,\vec g\microspace)\big)\\
-\omega\big([\vec f,\vec g\,],\vec h\big)
+\omega\big([\vec f,\vec h],\vec g\big)-\omega\big([\vec g,\vec h],\vec f\,\big)=0.
\end{multline}
We rewrite the six terms in the lefthand side of \eqref{eq:computeomegafgh} in terms of iterated Poisson brackets. Notice that:
\[\vec f\big(\omega(\vec g,\vec h)\big)=\vec f\big(\{g,h\}\big),\]
and using \eqref{eq:Poissonvec} we obtain:
\[\vec f\big(\omega(\vec g,\vec h)\big)=-\{f,\{g,h\}\}.\]
The second and third terms in the lefthand side of \eqref{eq:computeomegafgh} can in an analogous way be rewritten in terms of iterated Poisson brackets.
Let us work with the remaining terms. We have:
\[-\omega\big([\vec f,\vec g\,],\vec h\big)=\omega\big(\vec h,[\vec f,\vec g\,]\big)=\dd h\big([\vec f,\vec g\,]\big)=
[\vec f,\vec g\,](h)=\vec f\big(\vec g\microspace(h)\big)-\vec g\big(\vec f(h)\big).\]
Using \eqref{eq:Poissonvec} we obtain:
\[-\omega\big([\vec f,\vec g\,],\vec h\big)=\{f,\{g,h\}\}-\{g,\{f,h\}\}.\]
The fifth and sixth terms in the lefthand side of \eqref{eq:computeomegafgh} can in an analogous way be rewritten in terms of iterated Poisson brackets.
Once all terms in the lefthand side of \eqref{eq:computeomegafgh} are rewritten in terms of iterated Poisson brackets and the appropriate cancelations
are performed (and taking into account the anti-symmetry of the Poisson bracket), one obtains the Jacobi identity \eqref{eq:JacobiPoisson}.
\end{proof}

We have then that the vector space $C^\infty(M)$ is an associative algebra, endowed with the pointwise product of real valued functions, and
also a Lie algebra, endowed with the Poisson bracket. Moreover, the two structures are related by the fact that, for all $f\in C^\infty(M)$, the linear
endomorphism $\{f,\cdot\}$ of $C^\infty(M)$ is a derivation with respect to the associative product, i.e., \eqref{eq:Poifderiv} holds.

\begin{defin}
A {\em Poisson algebra\/} is a vector space $V$ endowed with both an associative algebra structure
$V\times V\ni(x,y)\mapsto xy\in V$ and a Lie algebra structure
$V\times V\ni(x,y)\mapsto[x,y]\in V$, in such a way that, for all $x\in V$, the linear endomorphism $[x,\cdot]$ of $V$ is a derivation of the associative product,
i.e.:
\[[x,y_1y_2]=[x,y_1]y_2+y_1[x,y_2],\quad x,y_1,y_2\in V.\]
\end{defin}
We have shown that if $(M,\omega)$ is a symplectic manifold, then $C^\infty(M)$ is a Poisson algebra, endowed with the pointwise product and the Poisson bracket.
Notice that the associative product of the Poisson algebra $C^\infty(M)$ is also commutative, but such commutativity is not a requirement of the definition of Poisson algebra.
In Exercise~\ref{exe:noncomPoisson} we give an example of a family of Poisson algebras whose associative product might not
be commutative (such example is related to Quantum Theory, as we will learn later).

\medskip

The Jacobi identity \eqref{eq:JacobiPoisson} (or, more generally, the Jacobi identity in any Lie algebra) can be interpreted in two ways: first, it says
that for all $f\in C^\infty(M)$, the linear endomorphism $\{f,\cdot\}$ is a derivation of the Poisson bracket:
\[\{f,\{g,h\}\}=\{\{f,g\},h\}+\{g,\{f,h\}\},\quad f,g,h\in C^\infty(M).\]
The second interpretation is that the map $f\mapsto\{f,\cdot\}$ (the {\em adjoint representation\/} of the Lie algebra) is a Lie algebra homomorphism
from the Lie algebra $C^\infty(M)$ (endowed with the Poisson bracket) to the Lie algebra of linear endomorphisms of $C^\infty(M)$ (endowed with the
standard commutator of linear operators):
\begin{equation}\label{eq:adjPoisson}
[\{f,\cdot\},\{g,\cdot\}]=\{\{f,g\},\cdot\},
\end{equation}
or, more explicitly:
\[\{f,\{g,h\}\}-\{g,\{f,h\}\}=\{\{f,g\},h\},\quad f,g,h\in C^\infty(M).\]
This second interpretation, coupled with \eqref{eq:Poissonvec}, yields the following:
\begin{prop}
For all $f,g\in C^\infty(M)$, the Lie bracket of the symplectic gradients $\vec f$, $\vec g$ is given by minus the symplectic gradient of the Poisson
bracket $\{f,g\}$:
\[[\vec f,\vec g\,]=-\overrightarrow{\{f,g\}\,}.\]
\end{prop}
\begin{proof}
Use \eqref{eq:adjPoisson} and \eqref{eq:Poissonvec}.
\end{proof}

\end{section}

\begin{section}{Measurements and Observables}



\end{section}

\section*{Exercises}

\subsection*{Affine spaces and Galilean spacetimes}

\begin{exercise}\label{exe:canaffine}
Let $V$ be a vector space and consider the action of the additive group $(V,+)$ on the set $E=V$ given by:
\[V\times E\ni(v,e)\longmapsto v+e\in E.\]
Show that $E=V$ is an affine space with underlying vector space $V$.
This is called the affine space {\em canonically obtained\/} from the vector space $V$.
\end{exercise}

\begin{exercise}\label{exe:vvplusO}
Let $E$ be an affine space with underlying vector space $V$. Show that for every point $O\in E$, the map:
\[V\ni v\longmapsto v+O\in E\]
is an affine isomorphism from the affine space canonically obtained from $V$ onto the affine space $E$.
\end{exercise}

\begin{exercise}\label{exe:affinesubspace}
Let $E$ be an affine space with underlying vector space $V$ and let $V_0$ be a vector subspace of $V$. Let $E_0$
be an orbit of the action of $V_0$ on $E$. Show that $E_0$ can be made into an affine space with underlying vector
space $V_0$ in such a way that the inclusion map of $E_0$ into $E$ is affine, with the inclusion map of $V_0$ into $V$
its underlying linear map. We call $E_0$ an {\em affine subspace\/} of $E$.
\end{exercise}

\begin{exercise}\label{eq:quotientaffine}
Let $E$ be an affine space with underlying vector space $V$ and let $V_0$ be a vector subspace of $V$. Denote
by $E/V_0$ the quotient of $E$ by the action of $V_0$ (i.e., $E/V_0$ is the set of orbits of the action of $V_0$ on $E$).
Show that the action:
\[(V/V_0)\times(E/V_0)\ni(v+V_0,e+V_0)\longmapsto(e+v)+V_0\in E/V_0\]
is well-defined and turns $E/V_0$ into an affine space with underlying vector space $V/V_0$. Show that the
quotient map $E\to E/V_0$ is an affine map whose underlying linear map is the quotient map $V\to V/V_0$.
\end{exercise}

\begin{exercise}\label{exe:shortexact}
Given an affine space $E$ with underlying vector space $V$, show that there exists a short exact sequence of groups:
\[1\longrightarrow V\longrightarrow\Aff(E)\longrightarrow\GL(V)\longrightarrow1,\]
where $1$ denotes the trivial group.
\end{exercise}

\subsection*{Units of measurement}

\begin{exercise}\label{exe:units}
Let $M$ be a real one-dimensional vector space. Every non zero vector $m\in M$ defines a basis of $M$, a basis
$m\otimes m$ of $M\otimes M$ and a basis $m^{-1}$ of the dual space $M^*$, where the linear functional
$m^{-1}$ satisfies $m^{-1}(m)=1$ ($m^{-1}$ is simply
the dual basis of $m$). Given non zero vectors $m_1,m_2\in M$ with $m_2=cm_1$, check that:
\[m_2\otimes m_2=c^2m_1\otimes m_1,\quad m_2^{-1}=\tfrac1c\,m_1^{-1}.\]
Conclude that if the elements of $M$ are to be interpreted as lengths then the elements of $M\otimes M$ are to
be interpreted as square lengths and the elements of $M^*$ are to be interpreted as inverse lengths.
\end{exercise}

\subsection*{Inertial coordinate systems and the Galileo group}

\begin{exercise}\label{exe:isoGalileo}
Show that any two Galilean spacetimes are isomorphic.
\end{exercise}

\begin{exercise}\label{exe:isoAutAut}
Let $X$, $Y$ be objects of an arbitrary category (for instance, Galilean spacetimes) and assume that we are given
an isomorphism $f:X\to Y$. Show that $f$ induces a group isomorphism:
\[f_*:\Aut(X)\longrightarrow\Aut(Y)\]
from the group of automorphisms of $X$ to the group of automorphisms of $Y$ defined by:
\[f_*(g)=f\circ g\circ f^{-1},\quad g\in\Aut(X).\]
Show that a subgroup $G$ of $\Aut(X)$ has the property that the subgroup $f_*(G)$ of $\Aut(Y)$ is independent
of the isomorphism $f:X\to Y$ if and only if $G$ is normal in $\Aut(X)$.
\end{exercise}

\begin{exercise}\label{exe:boosts}
Given an inertial coordinate system $\phi:E\to\R^4$, then the inverse image under $\phi$ of
$\R\times\{0\}\subset\R\times\R^3=\R^4$ is an affine one-dimensional subspace of $E$ which we call the {\em moving origin\/}
of the inertial coordinate system $\phi$. Let $\phi_1$, $\phi_2$ be inertial coordinate systems related by
an element $A$ of the passive Galileo group as in diagram \eqref{eq:phi1phi2A}.
The moving origin of the inertial coordinate system $\phi_2$ is mapped by $\phi_1$
onto the following one-dimensional affine subspace of $\R^4$:
\begin{equation}\label{eq:originphi2phi1}
\phi_1\big[\phi_2^{-1}\big(\R\times\{0\}\big)\big]=A^{-1}\big(\R\times\{0\}\big).
\end{equation}
Assume that $A$ is the Galilean boost \eqref{eq:boostv}. Show that \eqref{eq:originphi2phi1} is equal to:
\[\big\{(t,vt):t\in\R\big\}.\]
This means that {\em the origin of $\phi_2$ moves with uniform velocity $v$ with respect to the coordinate system $\phi_1$}.
\end{exercise}

\subsection*{Ontology and dynamics}

\begin{exercise}\label{exe:qtildeq}
Let $\phi_1$, $\phi_2$ be inertial coordinate systems related by
an element $A$ of the passive Galileo group as in diagram \eqref{eq:phi1phi2A}. Let $A$ be given as in
\eqref{eq:defAtx} and \eqref{eq:defLtx}.
Given maps $q,\tilde q:\R\to\R^3$, show that:
\begin{equation}\label{eq:phi1phi2qqtilde}
\phi_1^{-1}\big(\!\Gr(q)\big)=\phi_2^{-1}\big(\!\Gr(\tilde q)\big)
\end{equation}
if and only if $q$ and $\tilde q$ are related by:
\begin{equation}\label{eq:qtildeq}
\tilde q(t+t_0)=L_0\big(q(t)\big)-vt+x_0,\quad t\in\R.
\end{equation}
Equality \eqref{eq:phi1phi2qqtilde} means that $q$ and $\tilde q$ are representations with respect to the inertial
coordinate systems $\phi_1$, $\phi_2$, respectively, of the {\em same\/} particle worldline.
\end{exercise}

\begin{exercise}\label{exe:condForce}
Suppose that the force maps $F_j$, $j=1,\ldots,n$, satisfy the condition:
\begin{multline*}
L_0\big(F_j(t,q_1,\ldots,q_n,\dot q_1,\ldots,\dot q_n)\big)=F_j\big(t+t_0,L_0(q_1)-vt+x_0,\ldots,\\
L_0(q_n)-vt+x_0,L_0(\dot q_1)-v,\ldots,L_0(\dot q_n)-v\big),
\end{multline*}
for all $(t,q_1,\ldots,q_n,\dot q_1,\ldots,\dot q_n)\in\Dom(F_j)\subset\R\times(\R^3)^n\times(\R^3)^n$,
where $t_0\in\R$, $x_0\in\R^3$, $v\in\R^3$ and a linear isometry $L_0:\R^3\to\R^3$ are fixed. Show that if:
\[q_j:\R\longrightarrow\R^3,\quad\tilde q_j:\R\longrightarrow\R^3,\quad j=1,\ldots,n,\]
are smooth maps related as in \eqref{eq:qtildeq} then the maps $q_j$ satisfy the differential equation
\eqref{eq:dynamClMec} if and only if the maps $\tilde q_j$ satisfy the differential equation \eqref{eq:dynamClMec}.
\end{exercise}

\begin{exercise}\label{exe:condForcesat}
Show that the gravitational and the electrical forces satisfy the condition given in the statement
of Exercise~\ref{exe:condForce}.
\end{exercise}

\subsection*{Intrinsic formulation}

\begin{exercise}\label{exe:affspacemanifold}
Let $E$ be a finite-dimensional real affine space with underlying vector space $V$. Show that there is a unique
way to turn $E$ into a differentiable manifold in such a way that, for every point $O\in E$, the map defined
in Exercise~\ref{exe:vvplusO} is a smooth diffeomorphism. Show that, given a point $e\in E$,
then the linear isomorphism from $V$ to the tangent
space $T_eE$ given by the differential of the map defined in Exercise~\ref{exe:vvplusO} at the point $e-O$ is independent of the choice
of the point $O$. We use such linear isomorphism to identify once and for all the tangent space $T_eE$ with the vector space
$V$.
\end{exercise}

\begin{exercise}\label{exe:affinesmooth}
Let $E$, $E'$ be real finite-dimensional affine spaces with underlying vector spaces $V$, $V'$, respectively.
Show that every affine map $A:E\to E'$ is smooth and that for every $e\in E$, the differential:
\[\dd A(e):T_eE\cong V\longrightarrow V'\cong T_{A(e)}E'\]
is the underlying linear map of $A$.
\end{exercise}

\begin{exercise}
Let $q:\mathfrak T\to E$ be a smooth section of $\bar{\mathfrak t}$. Given a velocity
$v\in\mathfrak t^{-1}(1)$, show that the
following statements are equivalent:
\begin{itemize}
\item[(a)] $\dot q(t)=v$, for all $t\in\mathfrak T$;
\item[(b)] $q$ is an affine map whose underlying linear map $\R\to V$ is given by multiplication by the vector $v$;
\item[(c)] the image of $q$ is an affine subspace of $E$ with underlying vector space spanned by $v$.
\end{itemize}
A particle whose worldline is the image of a section $q:\mathfrak T\to E$ satisfying one of the equivalent conditions
above is said to have {\em rectilinear uniform motion with velocity $v$}.
\end{exercise}

\begin{exercise}
Let $\phi:E\to\R^4$ be an inertial coordinate system. The affine isomorphism $\phi$ passes to the quotient
and defines an affine isomorphism $\tau$ from $\mathfrak T=E/\Ker(\mathfrak t)$ to $\R\cong\R^4/\big(\{0\}\times\R^3\big)$.
Let $q_0:\mathfrak T\to E$ be a smooth section of $\bar{\mathfrak t}$ and let $q:\R\to\R^3$ be the smooth map such that:
\[\phi\big(q_0(\mathfrak T)\big)=\Gr(q).\]
We have a commutative diagram:
\[\xymatrix@C+20pt{%
\mathfrak T\ar[r]^{q_0}\ar[d]_\tau&E\ar[d]^\phi\\
\R\ar[r]_{(\Id,q)}&\R^4}\]
where $\Id$ denotes the identity map of $\R$. Consider the moving origin of the inertial coordinate system
$\phi$ (see Exercise~\ref{exe:boosts}); it is like the worldline of a particle having rectilinear uniform
motion with some velocity $v\in\mathfrak t^{-1}(1)$. Given $t_0\in\mathfrak T$, show that the relative velocity
$\dot q_0(t_0)-v\in\Ker(\mathfrak t)$ is mapped by the underlying linear map of $\phi$ to the vector
$\big(0,\frac{\dd q}{\dd t}(t)\big)$, where $t=\tau(t_0)$.
\end{exercise}

\subsection*{Lagrangians on manifolds}

\begin{exercise}\label{exe:genfundlemcalcvar}
Let $Q$ be a differentiable manifold, $q:[a,b]\to Q$ be a smooth curve and $\alpha:[a,b]\to TQ^*$ be a continuous map,
where $\alpha(t)$ belongs to the dual space $T_{q(t)}Q^*$, for all $t\in[a,b]$. Assume that:
\[\int_a^b\alpha(t)v(t)=0,\]
for any smooth vector field $v:[a,b]\to TQ$ along $q$ whose support is contained in the open interval
$\left]a,b\right[$. Show that $\alpha=0$.
\end{exercise}

\begin{exercise}\label{exe:transfLag1}
Let $U$ be an open subset of $\R^n$, $L:\R\times U\times\R^n\to\R$ be a Lagrangian, $\sigma:U\to\sigma(U)$
be a smooth diffeomorphism onto an open subset $\sigma(U)$ of $\R^n$ and $L_\sigma:\R\times\sigma(U)\times\R^n\to\R$
be the Lagrangian obtained by pushing $L$ using $\sigma$, i.e.:
\[L_\sigma\big(t,\sigma(q),\dd\sigma_q(\dot q)\big)=L(t,q,\dot q),\]
for all $t\in\R$, $q\in U$, $\dot q\in\R^n$.
Let $q:[a,b]\to U$ be a smooth curve and set $\tilde q=\sigma\circ q$. Show that:
\begin{multline*}
\frac{\dd}{\dd t}\frac{\partial L_\sigma}{\partial\dot q}\big(t,\tilde q(t),\dot{\tilde q}(t)\big)-
\frac{\partial L_\sigma}{\partial q}\big(t,\tilde q(t),\dot{\tilde q}(t)\big)=\\
[(\dd\sigma_{q(t)})^*]^{-1}\Big(\frac{\dd}{\dd t}\frac{\partial L}{\partial\dot q}\big(t,q(t),\dot q(t)\big)-
\frac{\partial L}{\partial q}\big(t,q(t),\dot q(t)\big)\Big),
\end{multline*}
for all $t\in[a,b]$, where $(\dd\sigma_{q(t)})^*:{\R^n}^*\to{\R^n}^*$ denotes the transpose of the linear
map $\dd\sigma_{q(t)}$.
\end{exercise}

\begin{exercise}\label{exe:transfLag2}
Let $Q$ be a differentiable manifold, $L:\R\times TQ\to\R$ be a Lagrangian, $q:[a,b]\to Q$ be a smooth map,
$\varphi:U\to\widetilde U\subset\R^n$ be a local chart on $Q$ and $\tilde q=\varphi\circ q\vert_{q^{-1}(U)}$. Use
the result of Exercise~\ref{exe:transfLag1} (with $\sigma$ the transition function between two coordinate charts)
to show that, for all $t\in q^{-1}(U)$, the linear functional:
\[(\dd\varphi_{q(t)})^*\Big(\frac{\dd}{\dd t}\frac{\partial L_\varphi}{\partial\dot q}
\big(t,\tilde q(t),\dot{\tilde q}(t)\big)-\frac{\partial L_\varphi}{\partial q}
\big(t,\tilde q(t),\dot{\tilde q}(t)\big)\Big)\in T_{q(t)}Q^*\]
does not depend on the choice of the chart $\varphi$.
\end{exercise}

\begin{exercise}[the double pendulum]\label{exe:doublependulum}
For this exercise, let us pretend that physical space is two-dimensional and let us identify it with the complex
plane $\C$, which is more convenient for writing down the formulas (three-dimensional physical space will
be considered in Exercise~\ref{exe:doublespherependulum} below).
The {\em double pendulum\/} is the system consisting of two particles in $\C$ constrained in the following
way: the first particle remains over a circle (say, centered at the origin of $\C$) of radius $r_1>0$ and the
second particle remains over a circle of radius $r_2>0$ centered at the position of the first particle. The set
$Q\subset\C^2$ of allowed pairs of positions for the two particles is then the image of the map:
\begin{equation}\label{eq:paramdoubpend}
\R\times\R\ni(\theta_1,\theta_2)\longmapsto(r_1e^{i\theta_1},r_1e^{i\theta_1}+r_2e^{i\theta_2})\in\C\times\C.
\end{equation}
\begin{itemize}
\item[(a)] Compute the differential of the map \eqref{eq:paramdoubpend} and show that such map is a smooth immersion.
\item[(b)] Show that the map \eqref{eq:paramdoubpend} passes to the quotient and induces an embedding from the
torus $(\R/2\pi\Z)^2$ to $\C^2$. Conclude that the image $Q$ of \eqref{eq:paramdoubpend} is a smooth submanifold
of $\C^2$ diffeomorphic (through the map \eqref{eq:paramdoubpend}) to the torus.
\item[(c)] Compute the tangent space of $Q$ at a point $q=(q_1,q_2)\in\C^2$. Show that a force $R=(R_1,R_2)\in\C^2$
is orthogonal to $T_qQ$ if and only if $R_2$ is parallel to the line connecting the points $q_1,q_2\in\C$ and
$R_1$ is the sum of $-R_2$ with a vector parallel to the line connecting the point $q_1$ and the origin.
\item[(d)] Consider the potential $V:\C^2\to\R$ defined by:
\[V(q_1,q_2)=-m_1g\Re(q_1)-m_2g\Re(q_2),\quad q_1,q_2\in\C,\]
where $g$ is a positive constant, $m_1,m_2>0$ denote the masses of the particles and $\Re(z)$ denotes the
real part of $z\in\C$ (this corresponds to the potential of a force of magnitude $m_jg$ pointing to the
direction of the positive real axis, acting on the $j$-th particle. For instance, this could arise from
what is called a {\em homogeneous gravitational field of magnitude $g$\/} pointing to the direction of the
positive real axis). Write down the representation of the Lagrangian $L^{\mathrm{cons}}$ (corresponding
to the potential $V$ and the constraint given by $Q$) with respect to a local chart on
$Q$ given by the inverse of the restriction of the map \eqref{eq:paramdoubpend} to some open set in which
it is injective (for instance, an open square of side $2\pi$). Write down also the corresponding Euler--Lagrange equation.
\end{itemize}
The condition on the forces $R_1$, $R_2$ that you discovered when solving item~(c) is precisely the condition that
one would expect under the assumption that particle number $1$ is attached to the origin and particle number $2$ is
attached to particle number $1$ by means of inextensible strings of negligible mass. The force $R_2$ is the {\em tension\/}
exerted upon particle number $2$ by the string connecting the two particles and it should be parallel to that string
(i.e., parallel to the line connecting the two particles) --- that is the standard assumption about string tension.
The force $R_1$ is the sum of two string tensions, one exerted upon particle number $1$
by the string connecting particle number $1$ to the origin (which should be parallel to that string) and the
other exerted upon particle number $1$ by the string connecting both particles (which should\footnote{%
Here's the argument: assuming Newton's law of reciprocal actions, then the force exerted by particle number $2$ upon the
string is equal to $-R_2$. If $R_2'$ denotes the force exerted by the string upon particle number $1$ then the force
exerted by particle number $1$ upon the string is $-R_2'$. The total force on the string is then $-R_2-R_2'$ and since
the mass of the string is being neglected, we take such total force to be zero, i.e., $R_2'=-R_2$.}
be equal to $-R_2$).
\end{exercise}

\begin{exercise}[the double spherical pendulum]\label{exe:doublespherependulum}
The {\em double spherical pendulum\/} is the system consisting of two particles in $\R^3$ constrained in the following
way: the first particle remains over a sphere (say, centered at the origin of $\R^3$) of radius $r_1>0$ and the
second particle remains over a sphere of radius $r_2>0$ centered at the position of the first particle. Check that
the submanifold $Q$ of $\R^3\times\R^3$ corresponding to such constraint is the image under the linear isomorphism:
\begin{equation}\label{eq:isospheres}
\R^3\times\R^3\ni(u_1,u_2)\longmapsto(r_1u_1,r_1u_1+r_2u_2)\in\R^3\times\R^3
\end{equation}
of the product $S^2\times S^2$, where $S^2\subset\R^3$ denotes the unit sphere centered at the origin.
Conclude that $Q$ is a smooth submanifold of $\R^3\times\R^3$.
Compute the tangent space $T_qQ$ at a point $q=(q_1,q_2)\in Q$ and show that a force $R=(R_1,R_2)\in\R^3\times\R^3$
is orthogonal to $T_qQ$ if and only if it satisfies the condition that appears in item~(c) of Exercise~\ref{exe:doublependulum}
above. If you want to have some fun, you can choose a potential $V$ and write down the representation of the Lagrangian
$L^{\mathrm{cons}}$ and the Euler--Lagrange equation with respect to your favorite chart on $Q$ (for instance,
you can obtain one by means of the map \eqref{eq:isospheres} and of spherical coordinates on the sphere $S^2$).
\end{exercise}

\begin{exercise}\label{exe:massinnprod}
Let $m_1,\ldots,m_n>0$ denote the masses of the particles and consider the linear isomorphism $M:(\R^3)^n\to(\R^3)^n$
defined by:
\[M(q_1,\ldots,q_n)=(m_1q_1,\ldots,m_nq_n),\quad q_1,\ldots,q_n\in\R^3.\]
Denote by $\langle\cdot,\cdot\rangle$ the standard inner product of $(\R^3)^n$ and by $\langle\cdot,\cdot\rangle_M$
the inner product of $(\R^3)^n$ defined by:
\[\langle q,q'\rangle_M=\langle M(q),q'\rangle,\quad q,q'\in(\R^3)^n.\]
If $V:\Dom(V)\subset\R\times(\R^3)^n\to\R$ is a smooth map,
we denote by $\nabla^M_qV(t,q)$ the {\em gradient of $V$ relative to the inner product $\langle\cdot,\cdot\rangle_M$\/}
(with respect to the second variable), i.e.:
\[\langle\nabla^M_qV(t,q),\cdot\rangle_M=\frac{\partial V}{\partial q}(t,q)\in{(\R^3)^n}^*,\quad (t,q)\in\Dom(V).\]
Show that:
\begin{itemize}
\item[(a)] $\nabla^M_qV(t,q)=M^{-1}\big(\nabla_qV(t,q)\big)$, for all $(t,q)\in\Dom(V)$;
\item[(b)] given a subspace $S$ of $(\R^3)^n$, then $M$ maps the orthogonal complement of $S$ with respect
to the inner product $\langle\cdot,\cdot\rangle_M$ onto the orthogonal complement of $S$ with respect to the
standard inner product of $(\R^3)^n$;
\item[(c)] the vector \eqref{eq:vecR} is orthogonal to $T_{q(t)}Q$ with respect to the standard inner
product of $(\R^3)^n$ if and only if the vector:
\[\frac{\dd^2q}{\dd t^2}(t)+\nabla^M_qV(t,q)\]
is orthogonal to $T_{q(t)}Q$ with respect to the inner product $\langle\cdot,\cdot\rangle_M$.
\end{itemize}
\end{exercise}

\begin{exercise}\label{exe:secfundform}
Let $Q$ be a smooth submanifold of $\R^n$. Given a point $q_0\in Q$, show that there exists a unique map:
\[\alpha:T_{q_0}Q\times T_{q_0}Q\longrightarrow\R^n/T_{q_0}Q\]
having the following property: if $q:I\to Q$, $v:I\to\R^n$
are smooth curves defined in an interval $I$ such that $v(t)\in T_{q(t)}Q$ for all $t\in I$ and if $q(t_0)=q_0$
for some $t_0\in I$ then:
\[\alpha\big(\dot q(t_0),v(t_0)\big)=\dot v(t_0)+T_{q_0}Q\in\R^n/T_{q_0}Q.\]
Show that such map $\alpha$ is bilinear ({\em hint}: let $f$ be an $\R^m$-valued smooth map defined in an open
neighborhood $U$ of $q_0$ in $\R^n$ such that $0\in\R^m$ is a regular value of $f$ and $Q\cap U=f^{-1}(0)$.
Differentiate the equality $\dd f_{q(t)}\big(v(t)\big)=0$ at $t=t_0$). The map $\alpha$ is called the
{\em second fundamental form\/} of the submanifold $Q$ at the point $q_0$ and it is denoted by
$\alpha^Q_{q_0}$. If one chooses an inner product on $\R^n$ (not necessarily the standard one) then
one can identify the quotient $\R^n/T_{q_0}Q$ with the orthogonal complement of $T_{q_0}Q$ with respect
to that inner product, obtaining from $\alpha^Q_{q_0}$ a bilinear form taking values in that orthogonal
complement. That's the {\em second fundamental form relative to the chosen inner product}.
\end{exercise}

\begin{exercise}
Consider the map $M$ and the inner product $\langle\cdot,\cdot\rangle_M$ defined in Exercise~\ref{exe:massinnprod}. Assuming
that the trajectories $q=(q_1,\ldots,q_n)$ of the particles are obtained as critical points of the action functional
$S_{L^{\mathrm{cons}}}$, show that the forces $R(t)$ exerted by the constraint are given by:
\[R(t)=M\Big(\alpha^Q_{q(t)}\big(\dot q(t),\dot q(t)\big)+P_{q(t)}\big[\nabla^M_qV\big(t,q(t)\big)\big]\Big),\]
where the second fundamental form of $Q$ is taken to be relative to the inner product $\langle\cdot,\cdot\rangle_M$
(see Exercise~\ref{exe:secfundform}) and $P_{q(t)}$ denotes the orthogonal projection with respect to $\langle\cdot,\cdot\rangle_M$
onto the orthogonal complement of $T_{q(t)}Q$ with respect to $\langle\cdot,\cdot\rangle_M$.
\end{exercise}

\begin{exercise}\label{exe:geodesics}
Consider the inner product $\langle\cdot,\cdot\rangle_M$ defined in Exercise~\ref{exe:massinnprod} and the Riemannian
metric on the submanifold $Q$ of $(\R^3)^n$ induced by such inner product. Show that if the potential $V$ is zero
then a curve $q:[a,b]\to Q$ is a critical point of the action functional $S_{L^{\mathrm{cons}}}$ if and only if
$q$ is a geodesic of $Q$ ({\em hint}: what is the variational problem whose solutions are the geodesics of a Riemannian
manifold?).
\end{exercise}

\subsection*{Symplectic forms over vector spaces}

\begin{exercise}\label{exe:bilinbasis}
Let $V$, $W$ be finite-dimensional real vector spaces and $B:V\times W\to\R$ be a bilinear map. Consider the linear map:
\begin{equation}\label{eq:canlinB}
V\ni v\longmapsto B(v,\cdot)\in W^*
\end{equation}
canonically associated to $B$. Given bases $\mathcal E=(e_i)_{i=1}^n$, $\mathcal F=(f_j)_{j=1}^m$ of $V$ and $W$,
respectively, we can associate an $n\times m$ matrix $[B]_{\mathcal E\mathcal F}$ to $B$ whose entry at row
$i$ and column $j$ is $B(e_i,f_j)$. Show that:
\begin{itemize}
\item[(a)] given $v\in V$, $w\in W$, then:
\[B(v,w)=\sum_{i=1}^n\sum_{j=1}^mB(e_i,f_j)v_iw_j=\begin{pmatrix}v_1&\cdots&v_n\end{pmatrix}
[B]_{\mathcal E\mathcal F}\begin{pmatrix}w_1\\\vdots\\w_m\end{pmatrix},\]
where $(v_1,\ldots,v_n)$ denotes the coordinates of $v$ with respect to $\mathcal E$ and $(w_1,\ldots,w_m)$
denotes the coordinates of $w$ with respect to $\mathcal F$;
\item[(b)] if $V$ is endowed with the basis $\mathcal E$ and $W^*$ with the dual basis of $\mathcal F$, show that
the matrix that represents the linear map \eqref{eq:canlinB} is the transpose of $[B]_{\mathcal E\mathcal F}$;
\item[(c)] if $V$ and $W$ have the same dimension, show that $B$ is {\em non degenerate\/} (i.e., given $v\in V$,
if $B(v,w)=0$ for all $w\in W$ then $v=0$) if and only if the matrix $[B]_{\mathcal E\mathcal F}$ is invertible
({\em hint}: $B$ is non degenerate if and only if the linear map \eqref{eq:canlinB} is injective).
\end{itemize}
\end{exercise}

\begin{exercise}\label{exe:proofeven}
If $(V,\omega)$ is a symplectic space then, by the result of item~(c) in Exercise~\ref{exe:bilinbasis},
the matrix $H=[\omega]_{\mathcal E\mathcal E}$ is invertible, where $\mathcal E$ is any basis of $V$. Conclude
from the fact that $H$ is both invertible and anti-symmetric that $V$ is even-dimensional ({\em hint}: take the
determinant on both sides of $H^\transp=-H$).
\end{exercise}

\begin{exercise}\label{exe:perpspan}
Let $V$ be a (not necessarily finite-dimensional) real vector space and $S$ be a subspace of $V$.
Given a bilinear map $B:V\times V\to\R$, then the {\em orthogonal complement\/} of $S$ with respect to $B$ is the
subspace $S^\perp$ of $V$ defined by:
\[S^\perp=\big\{v\in V:\text{$B(v,w)=0$, for all $w\in S$}\big\}.\]
Show that, if $S$ is finite-dimensional, then the following conditions are equivalent:
\begin{itemize}
\item[(a)] $S\cap S^\perp=\{0\}$;
\item[(b)] $B\vert_{S\times S}$ is non degenerate;
\item[(c)] $V=S\oplus S^\perp$.
\end{itemize}
({\em hint}: the only non trivial part is to prove that (b) implies $V=S+S^\perp$. For that, notice that, assuming (b),
since $S$ is finite-dimensional, the linear map $S\ni v\mapsto B(v,\cdot)\vert_S\in S^*$ is an isomorphism.
Given $w\in V$, we can then find $v\in S$ such that the linear functional $B(v,\cdot)\vert_S$ is equal to
$B(w,\cdot)\vert_S$. Notice that $w-v\in S^\perp$).
\end{exercise}

\begin{exercise}\label{exe:equivsymplecto}
Given symplectic spaces $(V,\omega)$, $(\widetilde V,\tilde\omega)$, show that the following conditions are equivalent
for a linear map $T:V\to\widetilde V$:
\begin{itemize}
\item[(a)] $T$ is a symplectomorphism;
\item[(b)] the image under $T$ of {\em any\/} symplectic basis of $(V,\omega)$ is a symplectic basis of $(\widetilde V,\tilde\omega)$;
\item[(c)] {\em there exists\/} a symplectic basis of $(V,\omega)$ whose image under $T$ is a symplectic basis of $(\widetilde V,\tilde\omega)$.
\end{itemize}
\end{exercise}

\begin{exercise}\label{exe:symplbasisbetter}
Let $V$ be a real finite-dimensional vector space and $\omega:V\times V\to\R$ be a (not necessarily non degenerate)
anti-symmetric bilinear form.
\begin{itemize}
\item[(a)] Show that there exists a basis $(e_1,\ldots,e_n,e'_1,\ldots,e'_n,f_1,\ldots,f_m)$
of $V$ such that:
\begin{gather*}
\omega(e_i,e'_j)=\delta_{ij},\quad\omega(e_i,e_j)=0,\quad\omega(e'_i,e'_j)=0,\\
\omega(e_i,f_k)=0,\quad\omega(e'_i,f_k)=0,\quad\omega(f_k,f_l)=0,
\end{gather*}
for all $i,j=1,\ldots,n$, $k,l=1,\ldots,m$, where $\delta_{ij}=1$ for $i=j$ and $\delta_{ij}=0$ for $i\ne j$
({\em hint}: the proof is almost identical to the proof of Proposition~\ref{thm:symplbasis}).
\item[(b)] Given a basis of $V$ as above, show that $(f_1,\ldots,f_m)$ is a basis of the {\em kernel\/}
of $\omega$, i.e., the kernel of the linear map \eqref{eq:omegalinear}.
\item[(c)] Given a basis of $V$ as above, show that $\omega$ is non degenerate if and only if $m=0$.
\end{itemize}
\end{exercise}

\subsection*{Symplectic manifolds and Hamiltonians}

\begin{exercise}\label{exe:Hamiltcoords}
Let $(M,\omega)$, $(\widetilde M,\tilde\omega)$ be symplectic manifolds and let $\Phi:M\to\widetilde M$ be a symplectomorphism. Given
a time-dependent Hamiltonian $H:\Dom(H)\subset\R\times M\to\R$, we can push it to the manifold
$\widetilde M$ using $\Phi$, obtaining a time-dependent Hamiltonian
$H_\Phi:\Dom(H_\Phi)\subset\R\times\widetilde M\to\R$ such that $\Dom(H_\Phi)=(\Id\times\Phi)\big(\!\Dom(H)\big)$
($\Id$ denotes the identity map of $\R$) and:
\[H_\Phi\big(t,\Phi(x)\big)=H(t,x),\]
for all $(t,x)\in\Dom(H)$.
\begin{itemize}
\item[(a)] Show that, for $(t,x)\in\Dom(H)$, we have:
\[\dd\Phi_x\big(\vec H(t,x)\big)=\overrightarrow{H_\Phi}\big(t,\Phi(x)\big).\]
\item[(b)] Let $x:I\to M$ be a smooth curve (defined over some interval $I\subset\R$) and let $t\in I$ be such that $\big(t,x(t)\big)\in\Dom(H)$.
Setting $\tilde x=\Phi\circ x$, show that:
\begin{equation}\label{eq:xprimevecH}
\frac{\dd x}{\dd t}(t)=\vec H\big(t,x(t)\big),
\end{equation}
if and only if:
\[\frac{\dd\tilde x}{\dd t}(t)=\overrightarrow{H_\Phi}\big(t,\tilde x(t)\big).\]
\item[(c)] Let now $\Phi:U\subset M\to\widetilde U\subset\R^{2n}$ be a symplectic chart and define $H_\Phi$ as above (of course,
before pushing $H$ using $\Phi$, we have to restrict $H$ to $\Dom(H)\cap(\R\times U)$).
Let $x:I\to M$ be a smooth curve and set
$\tilde x=\Phi\circ x\vert_{x^{-1}(U)}$. Write $\tilde x=(\tilde q,\tilde p)$, with:
\[\tilde q:x^{-1}(U)\subset I\longrightarrow\R^n,\quad\tilde p:x^{-1}(U)\subset I\longrightarrow\R^n.\]
Given $t\in I$ with $x(t)\in U$ and $\big(t,x(t)\big)\in\Dom(H)$, show that \eqref{eq:xprimevecH} holds if and only if:
\[\frac{\dd\tilde q}{\dd t}(t)=\frac{\partial H_\Phi}{\partial p}\big(t,\tilde q(t),\tilde p(t)\big),\quad
\frac{\dd\tilde p}{\dd t}(t)=-\frac{\partial H_\Phi}{\partial q}\big(t,\tilde q(t),\tilde p(t)\big).\]
\end{itemize}
\end{exercise}

\begin{exercise}\label{exe:Darboux}
The goal of this exercise is to prove Darboux's theorem (Theorem~\ref{thm:Darboux}).
\begin{itemize}
\item[(a)] Show that, in order to prove Darboux's theorem, it is sufficient to prove the following result: if $\omega$
is a symplectic form over an open subset $U$ of $\R^{2n}$ with $0\in U$ and if $\omega(0)$ is the canonical symplectic form
$\omega_0$ of $\R^{2n}$ then there exists a smooth diffeomorphism
$\Phi$ from an open neighborhood of $0$ in $\R^{2n}$ onto an open neighborhood of $0$ in $U$ such that $\Phi^*\omega$
is (the restriction to the domain of $\Phi$ of) $\omega_0$. ({\em hint}: use Corollary~\ref{thm:coronlyonesympl}).
\item[(b)] Let $\omega$ and $U$ be as in item~(a). Show that there exists a smooth one-form $\lambda$ over some open
neighborhood $V$ of $0$ in $U$ such that $\lambda(0)=0$ and $\dd\lambda=\omega_0-\omega$.
\item[(c)] For $t\in\R$, consider the smooth two-form over $U$ defined by:
\[\omega_t=(1-t)\omega_0+t\omega.\]
The set:
\begin{equation}\label{eq:omegatnondeg}
\big\{(t,x)\in\R\times V:\text{$\omega_t(x)$ is non degenerate}\!\big\}
\end{equation}
is open in $\R\times V$ and contains $\R\times\{0\}$. We can define a smooth time-dependent vector field
$X$ with domain \eqref{eq:omegatnondeg} such that:
\[i_{X_t}\omega_t=\lambda,\]
where $X_t=X(t,\cdot)$.
Denote by $F$ the flow of $X$ with initial time $t_0=0$ (i.e., for each $x$, $t\mapsto F(t,x)$ is the maximal integral curve of $X$ passing through $x$
at $t=0$).
Show that for all $t\in\R$, the map $F_t=F(t,\cdot)$ is defined over an open neighborhood of the origin ({\em hint}:
the domain of $F$ is open in $\R\times V$ and contains $\R\times\{0\}$, since $X_t(0)=0$, for all $t\in\R$).
\item[(d)] Use the result of Exercise~\ref{exe:ddtFtkappat} and formulas \eqref{eq:Lietimedep} and \eqref{eq:diid} to prove that:
\[\frac{\dd}{\dd t}(F_t^*\omega_t)=0.\]
\end{itemize}
\smallskip
Conclude the proof of Darboux's theorem by observing that:
\[F_1^*\omega=\omega_0.\]
\end{exercise}

\subsection*{Canonical forms in a cotangent bundle}

\begin{exercise}\label{exe:dstarfunc}
Let $Q_1$, $Q_2$, $Q_3$ be differentiable manifolds and:
\[\varphi:Q_1\longrightarrow Q_2,\quad\psi:Q_2\longrightarrow Q_3\]
be smooth diffeomorphisms. Show that:
\[\dd^*(\psi\circ\varphi)=\dd^*\psi\circ\dd^*\varphi.\]
\end{exercise}

\begin{exercise}\label{exe:firstHam}
Let $Q$, $\widetilde Q$ be differentiable manifolds and $\varphi:Q\to\widetilde Q$ be a smooth diffeomorphism.
Let $H:\Dom(H)\subset\R\times TQ^*\to\R$ be a time-dependent Hamiltonian and $H_\Phi:\Dom(H_\Phi)\subset\R\times T\widetilde Q^*\to\R$ be the
time-dependent Hamiltonian obtained by pushing $H$ using $\Phi=\dd\varphi^*$ (see Exercise~\ref{exe:Hamiltcoords}).
Given $(t,q,p)\in\Dom(H)$, show that:
\[\dd\varphi_q\Big(\frac{\partial H}{\partial p}(t,q,p)\Big)=\frac{\partial H_\Phi}{\partial p}\big(t,\tilde q,\tilde p),\]
where $(\tilde q,\tilde p)=\dd^*\varphi(q,p)$. Conclude, using also the result of Exercise~\ref{exe:Hamiltcoords}, that if the thesis
of Proposition~\ref{thm:firstHam} holds for $\widetilde Q$ then it also holds for $Q$.
\end{exercise}

\subsection*{The Legendre transform}

\begin{exercise}\label{exe:Legtransiso}
Let $E$, $E'$ be real finite-dimensional vector spaces, let $f:\Dom(f)\subset E\to\R$ be a map of class $C^2$ defined over some open subset $\Dom(f)$ of $E$
and let $T:E'\to E$ be a linear isomorphism. Consider the map $f\circ T:T^{-1}\big(\!\Dom(f)\big)\to\R$. Show that:
\begin{itemize}
\item[(a)] $f$ is regular (resp., hyper-regular) if and only if $f\circ T$ is regular (resp., hyper-regular);
\item[(b)] if $f$ is hyper-regular then the Legendre transform $(f\circ T)^*$ of $f\circ T$ is defined in the open subset $T^*\big(\!\Dom(f^*)\big)$
of ${E'}^*$ and it is equal to $f^*\circ{T^*}^{-1}$, where $T^*:E^*\to{E'}^*$ denotes the transpose
of the linear map $T$ and $f^*$ denotes the Legendre transform of $f$.
\end{itemize}
\end{exercise}

\begin{exercise}\label{exe:lemregdiff}
Let $Q$, $\widetilde Q$ be differentiable manifolds and $\varphi:Q\to\widetilde Q$ be a smooth diffeomorphism. Let $L:\Dom(L)\subset\R\times TQ\to\R$ be a
Lagrangian on $Q$ and let $L_\varphi:\Dom(L_\varphi)\subset\R\times T\widetilde Q\to\R$ be the Lagrangian on $\widetilde Q$ obtained by pushing $L$
using $\varphi$, i.e., $\Dom(L_\varphi)=(\Id\times\dd\varphi)\big(\!\Dom(L)\big)$ ($\Id$ denotes the identity map of $\R$) and:
\[L_\varphi\circ(\Id\times\dd\varphi)\vert_{\Dom(L)}=L.\]
Show that the diagram:
\[\xymatrix@C+15pt{%
\Dom(L)\ar[r]^{\mathbb FL}\ar[d]_{(\Id\times\dd\varphi)\vert_{\Dom(L)}}&\R\times TQ^*\ar[d]^{\Id\times\dd^*\varphi}\\
\Dom(L_\varphi)\ar[r]_{\mathbb FL_\varphi}&\R\times T\widetilde Q^*}\]
commutes. Conclude that claim (a) in the proof of Lemma~\ref{thm:LregFLdiff} is true.
\end{exercise}

\begin{exercise}\label{exe:invfuncteo}
Let $f:U\subset\R^m\times\R^n\to\R^n$ be a smooth map defined over an open subset $U$ of $\R^m\times\R^n$. Show that the following conditions are equivalent:
\begin{itemize}
\item[(a)] for all $x\in\R^m$, the map:
\[f(x,\cdot):\big\{y\in\R^n:(x,y)\in U\big\}\ni y\longmapsto f(x,y)\in\R^n\]
is a local diffeomorphism;
\item[(b)] the map $U\ni(x,y)\mapsto\big(x,f(x,y)\big)\in\R^m\times\R^n$ is a local diffeomorphism.
\end{itemize}
({\em hint}: compute the differential of the map $(x,y)\mapsto\big(x,f(x,y)\big)$ and use the inverse function theorem).
\end{exercise}

\begin{exercise}\label{exe:LegRiem}
Let $(Q,g)$ be a Riemannian manifold and:
\[V:\Dom(V)\subset\R\times Q\longrightarrow\R\]
be a smooth map defined over an open subset $\Dom(V)$ of $\R\times Q$.
Define a Lagrangian $L$ on $Q$ by setting:
\begin{equation}\label{eq:LRiem}
L(t,q,\dot q)=\frac12\,g_q(\dot q,\dot q)-V(t,q),
\end{equation}
for all $(t,q)\in\Dom(V)$ and all $\dot q\in T_qQ$. If the Riemannian metric of $Q$ is the one defined in Exercise~\ref{exe:geodesics} (and if $V$
is the restriction to $\R\times Q$ of the potential defined over an open subset of $\R\times(\R^3)^n$) then the Lagrangian \eqref{eq:LRiem} is precisely
the Lagrangian $L^{\mathrm{cons}}$ of Subsection~\ref{sub:constraints}. Show that $L$ is hyper-regular and that its Legendre transform $H$ is given by:
\[H(t,q,p)=\frac12\,g_q^{-1}(p,p)+V(t,q),\]
for all $(t,q)\in\Dom(V)$ and all $p\in T_qQ^*$, where $g_q^{-1}$ is the inner product on the dual space $T_qQ^*$ that turns the linear isomorphism:
\[T_qQ\ni\dot q\longmapsto g_q(\dot q,\cdot)\in T_qQ^*\]
into a linear isometry\footnote{%
Given a basis of $T_qQ$ and considering $T_qQ^*$ to be endowed with the dual basis, then the matrix that represents $g_q^{-1}$ is precisely
the inverse of the matrix that represents $g_q$ (see Exercise~\ref{exe:basisBprime}).}.
\end{exercise}

\subsection*{Symmetry and conservation laws}

\begin{exercise}\label{exe:Nother}
Let $(M,\omega)$ be a symplectic manifold and:
\[H:\Dom(H)\subset\R\times M\longrightarrow\R\]
be a time-dependent Hamiltonian over $M$. Let $G$ be a Lie group
and:
\[\rho:G\times M\ni(g,x)\longmapsto g\cdot x\in M\]
be a smooth action of $G$ on $M$. We say that $\rho$ is a {\em symmetry\/} of the triple $(M,\omega,H)$ if for
all $g\in G$ the diffeomorphism $\rho_g=\rho(g,\cdot)$ is a symplectomorphism of $(M,\omega)$ and if for all $(t,x)\in\Dom(H)$ and all $g\in G$ we have
$(t,g\cdot x)\in\Dom(H)$ and $H(t,g\cdot x)=H(t,x)$.
\begin{itemize}
\item[(a)] Given $X\in\mathfrak g$, show that the one-form $i_{X^{\!M}}\omega=\omega(X^{\!M},\cdot)$ is closed
({\em hint}: the Lie derivative $\mathbb L_{X^{\!M}}\omega$ vanishes)
and therefore locally given as a differential of a real valued smooth map.
\item[(b)] If $f$ is a real valued smooth map over an open subset $\Dom(f)$ of $M$ whose differential $\dd f$ equals (the restriction to $\Dom(f)$ of)
$i_{X^{\!M}}\omega$, show that $f$ is a first integral of the vector field
$\vec H$, i.e., $f$ is constant along integral curves of $\vec H$ that stay inside $\Dom(f)$.
\end{itemize}
When the symplectic form $\omega$ is exact, so that $\omega=-\dd\theta$ for some smooth one-form $\theta$, and the action of $G$ preserves $\theta$ then,
for all $X\in\mathfrak g$, the one-form $i_{X^{\!M}}\omega$ is exact, since the map $f=\theta(X^{\!M})$ satisfies $\dd f=i_{X^{\!M}}\omega$
(see item~(a) of Exercise~\ref{exe:homoPoisson}).
\end{exercise}

\begin{exercise}\label{exa:vecproduct}
Show that:
\begin{itemize}
\item[(a)] for each $v\in\R^3$, the linear endomorphism $X_v:w\mapsto v\wedge w$ of $\R^3$ is anti-symmetric and it is therefore an element
of the Lie algebra $\so(3)$ of the Lie group $\SO(3)$;
\item[(b)] the map $\R^3\ni v\mapsto X_v\in\so(3)$ is a linear isomorphism;
\item[(c)] given $v,w\in\R^3$, then the commutator:
\[[X_v,X_w]=X_v\circ X_w-X_w\circ X_v\]
is equal to $X_{v\wedge w}$. Conclude that $\R^3$ endowed with the vector product is a Lie algebra and that the map $v\mapsto X_v$ is an isomorphism
from the Lie algebra $(\R^3,{\wedge})$ onto the Lie algebra $\so(3)$.
\end{itemize}
\end{exercise}

\subsection*{The Poisson bracket}

\begin{exercise}\label{exe:basisBprime}
Let $V$ be a real finite-dimensional vector space and $B:V\times V\to\R$ be a non degenerate bilinear form, so that the linear map:
\begin{equation}\label{eq:isoBcarry}
V\ni v\longmapsto B(v,\cdot)\in V^*
\end{equation}
canonically associated to $B$ is an isomorphism. Define a bilinear form:
\[B':V^*\times V^*\longrightarrow\R\]
by carrying $B$ to $V^*$ using the linear isomorphism
\eqref{eq:isoBcarry}, i.e.:
\[B'\big(B(v,\cdot),B(w,\cdot)\big)=B(v,w),\quad v,w\in V.\]
Show that the linear map:
\[V^*\ni\alpha\longmapsto B'(\alpha,\cdot)\in V^{**}\cong V\]
canonically associated to $B'$ is the inverse of the transpose of the linear map \eqref{eq:isoBcarry} ({\em hint}: check first that
the transpose of \eqref{eq:isoBcarry} is given by $V\ni v\mapsto B(\cdot,v)\in V^*$).
Conclude that, if $V$ is endowed with a certain basis and $V^*$ is endowed with the corresponding dual basis, then the matrix
that represents $B'$ is the inverse of the transpose of the matrix that represents $B$ (see Exercise~\ref{exe:bilinbasis}).
\end{exercise}

\begin{exercise}\label{exe:symplomegapi}
Let $(V,\omega)$ be a symplectic space and $\pi:V^*\times V^*\to\R$ be the symplectic form over $V^*$ for which the linear isomorphism:
\[V\ni v\longmapsto\omega(v,\cdot)\in V^*\]
is a symplectomorphism, i.e.:
\[\pi\big(\omega(v,\cdot),\omega(w,\cdot)\big)=\omega(v,w),\quad v,w\in V.\]
Show that a basis of $V$ is symplectic for $\omega$ if and only if its dual basis is symplectic for $\pi$ ({\em hint}: a basis is symplectic if and only
if the matrix that represents the symplectic form with respect to that basis is $A=\big(\begin{smallmatrix}0_n&\I_n\\-\I_n&0_n\end{smallmatrix}\big)$,
where $0_n$ and $\I_n$ denote the $n\times n$ zero matrix and the $n\times n$ identity matrix, respectively. Use the result of Exercise~\ref{exe:basisBprime}
and the fact that $A^{-1}=-A$).
\end{exercise}

\begin{exercise}\label{exe:noncomPoisson}
Let $A$ be an associative algebra (for instance, the algebra of linear endomorphisms of a vector space) and define:
\begin{equation}\label{eq:bracket}
[x,y]=xy-yx,\quad x,y\in A.
\end{equation}
Show that $A$ is a Poisson algebra endowed with its associative product and with the commutator \eqref{eq:bracket}
(it is also a Poisson algebra if we replace the commutator \eqref{eq:bracket} with any scalar multiple of it).
\end{exercise}

\begin{exercise}
Let $(M,\omega)$, $(\widetilde M,\tilde\omega)$ be symplectic manifolds and let $\Phi:M\to\widetilde M$ be a symplectomorphism. Show that the map:
\[\Phi^*:C^\infty(\widetilde M)\ni f\longmapsto f\circ\Phi\in C^\infty(M)\]
is an {\em isomorphism of Poisson algebras}, i.e., it is a linear isomorphism that preserves both the associative (pointwise) product:
\[\Phi^*(fg)=\Phi^*(f)\Phi^*(g),\quad f,g\in C^\infty(\widetilde M),\]
and the Poisson bracket:
\[\Phi^*\big(\{f,g\}\big)=\{\Phi^*(f),\Phi^*(g)\},\quad f,g\in C^\infty(\widetilde M).\]
\end{exercise}

\begin{exercise}
Let $M$ be an open subset of $\R^{2n}$ endowed with the canonical symplectic form (Example~\ref{exa:cansymplform}). Given a map
$f:M\to\R$, denote its $2n$ partial derivatives by:
\[\frac{\partial f}{\partial q^1},\ldots,\frac{\partial f}{\partial q^n},\frac{\partial f}{\partial p^1},\ldots,\frac{\partial f}{\partial p^n}.\]
Given $f,g\in C^\infty(M)$, show that their Poisson bracket is given by:
\[\{f,g\}=\sum_{j=1}^n\frac{\partial f}{\partial q^j}\frac{\partial g}{\partial p^j}-\frac{\partial f}{\partial p^j}\frac{\partial g}{\partial q^j}.\]
\end{exercise}

\begin{exercise}\label{exe:homoPoisson}
Let $(M,\omega)$ be a symplectic manifold whose symplectic form is exact and let $\theta$ be a smooth one-form over $M$ with $\omega=-\dd\theta$.
Assume that we are given a smooth action $\rho:G\times M\to M$ of a Lie group $G$ on $M$ that preserves the one-form $\theta$, i.e.,
$\rho_g^*\theta=\theta$ for all $g\in G$, where $\rho_g=\rho(g,\cdot)$ (this is the case, for instance, if $M$ is a cotangent bundle $TQ^*$,
$\theta$ is the canonical one-form and the action of $G$ on $M=TQ^*$ is obtained from an action of $G$ on $Q$). Each $X$ in the Lie algebra $\mathfrak g$
induces a vector field $X^{\!M}$ on $M$ (when $M=TQ^*$ and the action of $G$ is a symmetry of a time-dependent Hamiltonian then the map $\theta(X^{\!M})$ is
precisely the conserved quantity given by Theorem~\ref{thm:Nother}).
\begin{itemize}
\item[(a)] For $X\in\mathfrak g$, show that the differential of the map $\theta(X{\!^M})$ is:
\[i_{X^{\!M}}\omega=\omega(X^{\!M},\cdot)\]
({\em hint}: write the Lie derivative
of $\theta$ along $X^{\!M}$ using formula \eqref{eq:diid} and observe that such Lie derivative must vanish). Conclude that
the symplectic gradient of $\theta(X^{\!M})$ is $X^{\!M}$.
\item[(b)] For $X,Y\in\mathfrak g$, show that the Poisson bracket $\{\theta(X^{\!M}),\theta(Y^{\!M})\}$ equals $\omega(X^{\!M},Y^{\!M})$.
\item[(c)] For $X,Y\in\mathfrak g$, show that $\omega(X^{\!M},Y^{\!M})=-\theta\big([X^{\!M},Y^{\!M}]\big)$ ({\em hint}: compute $\dd\theta(X^{\!M},Y^{\!M})$ using
formula \eqref{eq:Cartanextdiff}. Use the result of item~(a) to conclude that $X^{\!M}\big(\theta(Y^{\!M})\big)=\omega(Y^{\!M},X^{\!M})$).
\item[(d)] Show that the map $\mathfrak g\ni X\longmapsto\theta(X^{\!M})\in C^\infty(M)$ is a Lie algebra homomorphism if $C^\infty(M)$ is endowed with
the Poisson bracket ({\em hint}: use formula \eqref{eq:brackXYM}).
\end{itemize}
\end{exercise}

\end{chapter}

\appendix

\begin{chapter}{A summary of certain prerequisites}

This notes are intendend as a course for mathematicians and graduate students in Mathematics. Therefore, a lot of standard material from graduate mathematical
courses are taken as prerequisites. Nevertheless, in this appendix we make a quick presentation of some of those prerequisites. If you don't have any familiarity
with such prerequisites, you probably won't be able to learn them using this appendix, but if you have some familiarity with them, this appendix might be useful
for a quick review or as a quick reference guide. Most results will be stated without proof.

\begin{section}{Quick review of multilinear algebra}\label{sec:quickmultilinear}

Throughout the section, $V$ denotes a fixed real finite-dimensional vector space. For most of what is presented in the section, the field of real numbers
can be replaced with an arbitrary field\footnote{%
In the definition of wedge product, the factorial of the degree of the forms appears in the denominator and that doesn't make sense in general if the characteristic of the
field is not zero. Also, if the characteristic of the field is two, then anti-symmetry is the same as symmetry and it is not true that an anti-symmetric map vanishes
when two of its entries are equal.} and for {\em everything\/} that is presented in the section it can be replaced with an arbitrary field of characteristic zero.
We choose to use the field of real numbers for the presentation only for psychological reasons.

Given natural numbers $r$, $s$, then an {\em $(r,s)$-tensor over $V$\/}
(also called a tensor that is {\em r times covariant and $s$ times contravariant}) is a multilinear map:
\[\tau:V\times\cdots\times V\times V^*\times\cdots\times V^*\longrightarrow\R\]
in which there are $r$ copies of $V$ and $s$ copies of the dual space $V^*$.
The set of $(r,s)$-tensors over $V$ is, in a natural way, a real
vector space. Such vector space is naturally isomorphic to the tensor product of $r$ copies of the dual space $V^*$ and $s$ copies of the space $V$:
\begin{equation}\label{eq:notationtensors}
\Big(\bigotimes_rV^*\Big)\otimes\Big(\bigotimes_sV\Big).
\end{equation}
We won't need such identification between spaces of multilinear maps and tensor products of vector spaces, but we will use \eqref{eq:notationtensors}
as a notation for the space of $(r,s)$-tensors over $V$. The space of $(0,0)$-tensors over $V$
is simply the field of real numbers. The space of $(0,1)$-tensors over $V$ is the bidual $V^{**}$ and it will be identified in the usual way with the space $V$
itself, so that elements of $V$ are regarded as $(0,1)$-tensors over $V$. The space of $(1,0)$-tensors over $V$ is simply the dual space $V^*$.
The space of linear endomorphisms of $V$ can be naturally identified with the space of $(1,1)$-tensors over $V$: namely, we identify a linear endomorphism
$T:V\to V$ with the bilinear map $V\times V^*\ni(v,\alpha)\mapsto\alpha\big(T(v)\big)\in\R$.

Given some other real finite-dimensional vector space $W$, a linear isomorphism $T:W\to V$ and an $(r,s)$-tensor $\tau$ over $V$ then the {\em pull-back\/}
$T^*\tau$ is the $(r,s)$-tensor over $W$ defined by:
\begin{multline}\label{thm:defTstartau}
(T^*\tau)(w_1,\ldots,w_r,\beta_1,\ldots,\beta_s)\\
=\tau\big(T(w_1),\ldots,T(w_r),\beta_1\circ T^{-1},\ldots,\beta_s\circ T^{-1}\big),
\end{multline}
for all $w_1,\ldots,w_r\in W$, $\beta_1,\ldots,\beta_s\in W^*$. When the tensor $\tau$ is {\em purely covariant},
i.e., when $s=0$, then the pull-back $T^*\tau$ is defined for {\em any\/} linear map $T:W\to V$
(because we don't need to use $T^{-1}$ in \eqref{thm:defTstartau} when $s=0$). The operation $\tau\mapsto T^*\tau$ defines
a linear map:
\[T^*:\Big(\bigotimes_rV^*\Big)\otimes\Big(\bigotimes_sV\Big)\longrightarrow\Big(\bigotimes_rW^*\Big)\otimes\Big(\bigotimes_sW\Big)\]
in which it is assumed that $T$ be an isomorphism if $s\ne0$. Given some other real finite-dimensional vector space $P$ and a linear map $S:P\to W$ then:
\[(T\circ S)^*\tau=S^*T^*\tau,\]
for any tensor $\tau$ over $V$, in which it is necessary to assume that $T$ and $S$ be isomorphisms if $\tau$ is not purely covariant.

Given an $(r,s)$-tensor $\tau$ over $V$ and
an $(r',s')$-tensor $\tau'$ over $V$, then their {\em tensor product\/}
is the $(r+r',s+s')$-tensor $\tau\otimes\tau'$ over $V$ defined by:
\begin{multline*}
(\tau\otimes\tau')(v_1,\ldots,v_{r+r'},\alpha_1,\ldots,\alpha_{s+s'})\\
=\tau(v_1,\ldots,v_r,\alpha_1,\ldots,\alpha_s)\tau'(v_{r+1},\ldots,v_{r+r'},\alpha_{s+1},\ldots,\alpha_{s+s'}),
\end{multline*}
for all $v_1,\ldots,v_{r+r'}\in V$, $\alpha_1,\ldots,\alpha_{s+s'}\in V^*$. The product $c\tau$ of a tensor $\tau$ by a real number $c$
coincides with the tensor product $c\otimes\tau$ (and also with $\tau\otimes c$), where the real number $c$ is seen as a $(0,0)$-tensor.
Clearly, the tensor product operation is associative, i.e., if $\tau$, $\tau'$, $\tau''$ are tensors over $V$ then:
\[(\tau\otimes\tau')\otimes\tau''=\tau\otimes(\tau'\otimes\tau''),\]
so that we can write tensor products of several tensors without parenthesis.
If $(e_1,\ldots,e_n)$ is a basis of $V$ and $(e^1,\ldots,e^n)$ denotes its dual basis then:
\begin{equation}\label{eq:basistensors}
e^{i_1}\otimes\cdots\otimes e^{i_r}\otimes e_{j_1}\otimes\cdots\otimes e_{j_s},\quad i_1,\ldots,i_r,j_1,\ldots,j_s=1,\ldots,n,
\end{equation}
is a basis of the space of $(r,s)$-tensors over $V$. The dimension of the space of $(r,s)$-tensors over $V$ is therefore equal to $n^{r+s}$.
The coordinates of an $(r,s)$-tensor $\tau$ over $V$ with respect to the basis \eqref{eq:basistensors} are given by:
\begin{equation}\label{eq:reprtau}
\tau_{i_1\ldots i_r}^{j_1\ldots j_s}=\tau(e_{i_1},\ldots,e_{i_r},e^{j_1},\ldots,e^{j_s}).
\end{equation}
We say that the family of real numbers \eqref{eq:reprtau} {\em represents\/} the tensor $\tau$ with respect to the basis $(e_1,\ldots,e_n)$.
The $r$ lower indexes in \eqref{eq:reprtau} are normally called the {\em covariant\/} indexes and the $s$ upper indexes the
{\em contravariant\/} indexes.

Pull-backs preserve tensor products: if $T:W\to V$ is a linear map and $\tau$, $\tau'$ are tensors over $V$ then:
\begin{equation}\label{eq:Tstarthomo}
T^*(\tau\otimes\tau')=(T^*\tau)\otimes(T^*\tau'),
\end{equation}
where we have to assume that $T$ be an isomorphism if either $\tau$ or $\tau'$ is not purely covariant.

In what follows, we will focus on purely covariant tensors and we will
use the symbol $\bigotimes_k V^*$ to denote the space of all $k$-linear maps $\tau:V^k\to\R$ (i.e., the space of
all $(k,0)$-tensors over $V$). The subspace of $\bigotimes_k V^*$ consisting of {\em anti-symmetric\/} $k$-linear maps will
be denoted by $\bigwedge_k V^*$. For $k=0$, both $\bigotimes_kV^*$ and $\bigwedge_kV^*$ are just the scalar field $\R$. If $k$ is
larger than the dimension of $V$, then $\bigwedge_kV^*$ is the null space.

If $\kappa\in\bigwedge_kV^*$ is an anti-symmetric $k$-linear map and $\lambda\in\bigwedge_lV^*$ is an anti-symmetric
$l$-linear map then the {\em wedge product\/} $\kappa\wedge\lambda\in\bigwedge_{k+l}V^*$ is the $(k+l)$-linear
anti-symmetric map defined by\footnote{%
Some authors use $\frac1{(k+l)!}$ in front of the summation sign. This difference in the definition of the
exterior product also influences the definition of exterior differentiation of differential forms
(see Subsection~\ref{sub:TensorForms}), since the properties
that characterize the exterior differential depend on the definition of the wedge product.}:
\[(\kappa\wedge\lambda)(v_1,\ldots,v_{k+l})=\frac1{k!l!}\sum_{\sigma\in S^{k+l}}
\sgn(\sigma)(\kappa\otimes\lambda)(v_{\sigma(1)},\ldots,v_{\sigma(k+l)}),\]
for all $v_1,\ldots,v_{k+l}\in V$, where $S^{k+l}$ denotes the group of all bijections of the set $\{1,\ldots,k+l\}$ and $\sgn(\sigma)$ denotes the {\em sign\/}
of the permutation $\sigma$, i.e., $\sgn(\sigma)=1$ if $\sigma$ is even and $\sgn(\sigma)=-1$ if $\sigma$ is odd.
The product $c\kappa$ of an anti-symmetric $k$-linear map $\kappa$ by a real number $c$
coincides with the wedge product $c\wedge\kappa$ (and also with $\kappa\wedge c$), where the real number $c$ is seen as
an element of $\bigwedge_0V^*$. The wedge product operation is associative, i.e., if $\kappa$, $\lambda$ and $\mu$ are anti-symmetric purely covariant
tensors over $V$ then:
\[(\kappa\wedge\lambda)\wedge\mu=\kappa\wedge(\lambda\wedge\mu),\]
so that we can write wedge products of several anti-symmetric covariant tensors without parenthesis. Given $\kappa_i\in\bigwedge_{k_i}V^*$, $i=1,\ldots,r$, then
the following formula holds:
\begin{multline*}
(\kappa_1\wedge\cdots\wedge\kappa_r)(v_1,\ldots,v_k)\\
=\frac1{k_1!\cdots k_r!}\sum_{\sigma\in S^k}
\sgn(\sigma)(\kappa_1\otimes\cdots\otimes\kappa_r)(v_{\sigma(1)},\ldots,v_{\sigma(k)}),
\end{multline*}
for all $v_1,\ldots,v_k\in V$, where $k=k_1+\cdots+k_r$. In particular, given linear functionals $\alpha_i\in V^*=\bigwedge_1V^*$, $i=1,\ldots,r$, then:
\[(\alpha_1\wedge\cdots\wedge\alpha_r)(v_1,\ldots,v_r)=\sum_{\sigma\in S^r}\sgn(\sigma)
\alpha_1(v_{\sigma(1)})\cdots\alpha_r(v_{\sigma(r)}),\]
i.e.:
\begin{equation}\label{eq:wedgedet}
(\alpha_1\wedge\cdots\wedge\alpha_r)(v_1,\ldots,v_r)=\det\big(\alpha_i(v_j)\big)_{r\times r},
\end{equation}
for all $v_1,\ldots,v_r\in V$. If $(e_1,\ldots,e_n)$ is a basis of $V$ and $(e^1,\ldots,e^n)$ denotes its dual basis then:
\begin{equation}\label{eq:basiswedge}
e^{i_1}\wedge\cdots\wedge e^{i_k},\quad 1\le i_1<i_2<\cdots<i_k\le n,
\end{equation}
is a basis of $\bigwedge_kV^*$. Thus, for $0\le k\le n$, the dimension of $\bigwedge_kV^*$ is $\binom nk$ and the dimension of $\bigwedge_nV^*$ is equal to $1$.
The non zero elements of $\bigwedge_nV^*$ are called {\em volume forms\/} over $V$. The coordinates of $\kappa\in\bigwedge_kV^*$ with respect
to the basis \eqref{eq:basiswedge} are:
\[\kappa_{i_1\ldots i_k}=\kappa(e_{i_1},\ldots,e_{i_k}).\]

For $\kappa\in\bigwedge_kV^*$, $\lambda\in\bigwedge_lV^*$, we have:
\[\kappa\wedge\lambda=(-1)^{kl}\lambda\wedge\kappa,\]
so that $\kappa\wedge\kappa=0$ if $\kappa\in\bigwedge_k V^*$ and $k$ is {\em odd}.

Set:
\[\bigotimes V^*=\bigoplus_{k=0}^\infty\bigotimes_kV^*,\quad\bigwedge V^*=\bigoplus_{k=0}^\infty\bigwedge_kV^*
=\bigoplus_{k=0}^n\bigwedge_kV^*,\]
where $n=\Dim(V)$. The tensor product operation of purely covariant tensors extends in a unique way
to a bilinear binary operation in the space $\bigotimes V^*$ and the wedge product operation of
anti-symmetric purely covariant tensors extends in a unique way to a bilinear binary operation in the space
$\bigwedge V^*$. Both $\bigotimes V^*$ and $\bigwedge V^*$ become associative (graded) real algebras with unit endowed
with such binary operations. Observe that $\bigwedge V^*$ is a vector subspace but {\em not a subalgebra\/} of
$\bigotimes V^*$.

Given a linear map $T:W\to V$, then the pull-back operation
$\tau\mapsto T^*\tau$ (on purely covariant tensors over $V$) extends to an algebra homomorphism:
\[T^*:\bigotimes V^*\longrightarrow\bigotimes W^*,\]
so that \eqref{eq:Tstarthomo} holds for all $\tau,\tau'\in\bigotimes V^*$.
The restriction of $T^*$ to $\bigwedge V^*$ gives an algebra homomorphism:
\[T^*:\bigwedge V^*\longrightarrow\bigwedge W^*,\]
so that:
\[T^*(\kappa\wedge\lambda)=(T^*\kappa)\wedge(T^*\lambda),\]
for all $\kappa,\lambda\in\bigwedge V^*$.

Given a vector $v\in V$ and a tensor $\tau\in\bigotimes_kV^*$, $k\ge1$, we define the {\em interior product of $\tau$ by $v$\/} to be the
tensor $i_v\tau\in\bigotimes_{k-1}V^*$ given by:
\[(i_v\tau)(v_1,\ldots,v_{k-1})=\tau(v,v_1,\ldots,v_{k-1}),\]
for all $v_1,\ldots,v_{k-1}\in V$. For $\tau\in\bigotimes_0V^*$ we set $i_v\tau=0$. The map $i_v$ extends in a unique way to a linear endomorphism
(not an algebra homomorphism!) of $\bigotimes V^*$. Such endomorphism sends $\bigwedge V^*$ to $\bigwedge V^*$ and for $\kappa\in\bigwedge_kV^*$,
$\lambda\in\bigwedge_lV^*$ we have:
\begin{equation}\label{eq:propiv}
i_v(\kappa\wedge\lambda)=(i_v\kappa)\wedge\lambda+(-1)^k\kappa\wedge(i_v\lambda).
\end{equation}

\end{section}

\begin{section}{Quick review of calculus on manifolds}\label{sec:quickmanifold}

We have selected for presentation some topics which are taught during courses on calculus on manifolds. We won't present either the definition of differentiable manifold
or the construction of the tangent bundle. The word ``smooth'' always refers to ``class $C^\infty$''. Differentiable manifolds are always
assumed to be smooth (i.e., endowed with a smooth atlas) and to have a Hausdorff topology. For maps, the word ``differentiable'' is to be taken literally,
i.e., a differentiable map is map that can be differentiated once.

\subsection{Vector fields and flows} Let $M$ be a differentiable manifold and let $X$ be a smooth
vector field over $M$, i.e., $X$ is a smooth map from $M$ to the tangent bundle $TM$
such that $X(x)\in T_xM$, for all $x\in M$. By an {\em integral curve\/} of $X$ we mean a differentiable map
$x:I\to M$, defined over some interval $I\subset\R$, such that:
\[\frac{\dd x}{\dd t}(t)=X\big(x(t)\big),\]
for all $t\in I$. Given $t_0\in\R$ and $x_0\in M$, there exists a unique integral curve $x:I\to M$ of $X$
with $t_0\in I$, $x(t_0)=x_0$ and that is {\em maximal}, i.e., it cannot be extended to an integral curve of $X$
defined in a strictly larger interval. Every integral curve of $X$ is a restriction of a maximal integral curve.
The {\em flow\/} of $X$ is the map $F:\Dom(F)\subset\R\times M\to M$
such that, for all $x_0\in M$:
\[\big\{t\in\R:(t,x_0)\in\Dom(F)\big\}\ni t\longmapsto F(t,x_0)\in M\]
is the maximal integral curve of $X$ passing through $x_0$ at $t=0$. The domain of $F$ is open in $\R\times M$ and the map
$F$ is smooth. For each $t\in\R$, the map:
\[F_t:\Dom(F_t)=\big\{x\in M:(t,x)\in\Dom(F)\big\}\ni x\longmapsto F(t,x)\in M\]
is a smooth diffeomorphism between open subsets of $M$ whose inverse is the map $F_{-t}$ (the image of $F_t$
is precisely the domain of $F_{-t}$). The map $F_0$ is the identity map of $M$.
Given $t,s\in\R$, if $x\in M$ is in the domain of $F_t$ and $F_t(x)$ is in the domain of $F_s$ then $x$ is in the domain of $F_{t+s}$ and:
\[F_{t+s}(x)=F_s\big(F_t(x)\big).\]

Let $X$ be a smooth {\em time-dependent vector field\/} over a differentiable manifold $M$, i.e., $X$ is a smooth
map from an open subset $\Dom(X)$ of $\R\times M$ to the tangent bundle $TM$ such that
$X(t,x)\in T_xM$, for all $(t,x)\in\Dom(X)$.
For each $t\in\R$, we obtain from $X$ a smooth vector field $X_t=X(t,\cdot)$ over the open set:
\[\Dom(X_t)=\big\{x\in X:(t,x)\in\Dom(X)\big\}.\]
By an {\em integral curve\/} of $X$ we mean a differentiable map
$x:I\to M$, defined over some interval $I\subset\R$, such that $\big(t,x(t)\big)\in\Dom(X)$ and:
\[\frac{\dd x}{\dd t}(t)=X\big(t,x(t)\big),\]
for all $t\in I$. Given $t_0\in\R$ and $x_0\in M$, if $(t_0,x_0)\in\Dom(X)$,
there exists a unique integral curve $x:I\to M$ of $X$
with $t_0\in I$, $x(t_0)=x_0$ and that is maximal (in the sense explained above). Again, every integral curve of $X$
is a restriction of a maximal integral curve. For vector fields that {\em do not\/} depend on time, it is true that the
time translation of an integral curve is an integral curve (i.e., if $t\mapsto x(t)$ is an integral curve and $t_0\in\R$
is given then $t\mapsto x(t_0+t)$ is an integral curve); for that reason, one only considers integral curves satisfying
some initial condition at $t=0$ when defining the flow. For time-dependent vector fields it is not true that the
time translation of an integral curve is an integral curve, so it is relevant to consider a flow with an arbitrary
initial time $t_0$: for a fixed $t_0\in\R$, we define the {\em flow with initial time $t_0$\/}
of the time-dependent vector field $X$ to be the map $F^{t_0}:\Dom(F^{t_0})\subset\R\times\Dom(X_{t_0})\subset
\R\times M\to M$ such that, for all $x_0\in\Dom(X_{t_0})$:
\[\big\{t\in\R:(t,x_0)\in\Dom(F^{t_0})\big\}\ni t\longmapsto F^{t_0}(t,x_0)\in M\]
is the maximal integral curve of $X$ passing through $x_0$ at $t=t_0$. The domain of $F^{t_0}$ is open in $\R\times M$
and the map $F^{t_0}$ is smooth. In fact, we can say more; the set:
\[\big\{(t_0,t,x)\in\R\times\R\times M:(t,x)\in\Dom(F^{t_0})\big\}\]
is open in $\R\times\R\times M$ and the map $(t_0,t,x)\mapsto F^{t_0}(t,x)\in M$ (defined on such set) is smooth\footnote{%
Such properties of the flow of a time-dependent vector field $X$ are easily established as a corollary of the properties of the flow of the (time independent)
vector field over the manifold $\Dom(X)\subset\R\times M$ defined by $(t,x)\mapsto\big(1,X(t,x)\big)$.}.
For each $t\in\R$, the map:
\[F^{t_0}_t:\Dom(F^{t_0}_t)=\big\{x\in M:(t,x)\in\Dom(F^{t_0})\big\}\ni x\longmapsto F^{t_0}(t,x)\in M\]
is a smooth diffeomorphism between open subsets of $M$ whose inverse is the map $F^t_{t_0}$ (the image
of $F^{t_0}_t$ is the domain of $F^t_{t_0}$). Given $t_0,t,s\in\R$, if $x$ is in the domain of $F^{t_0}_t$ and
$F^{t_0}_t(x)$ is in the domain of $F^t_s$ then $x$ is in the domain of $F^{t_0}_s$ and:
\[F^{t_0}_s(x)=F^t_s\big(F^{t_0}_t(x)\big).\]
For any $t_0\in\R$, the map $F^{t_0}_{t_0}$ is the identity map of $\Dom(X_{t_0})$.
When $X$ {\em do not\/} depend on time, we have that $F^{t_0}_t$ coincides with $F^0_{t-t_0}$.

\subsection{Tensor fields and differential forms}
\label{sub:TensorForms}

Given a differentiable manifold $M$, then an {\em $(r,s)$-tensor field over $M$\/} (also called a tensor field that is
{\em r times covariant and $s$ times contravariant}) is a map $\tau$
that associates to each point $x\in M$ an $(r,s)$-tensor $\tau(x)$ (denoted also by $\tau_x$) over the tangent space $T_xM$ (see Section~\ref{sec:quickmultilinear}).
Scalar fields (i.e., real valued functions) are $(0,0)$-tensor fields and
vector fields are (identified with) $(0,1)$-tensor fields. By a {\em differential form of degree $k$\/}
(or simply a {\em $k$-form}) over $M$ we mean an {\em anti-symmetric\/} $(k,0)$-tensor field $\kappa$ over $M$ (i.e., $\kappa$
associates to each $x\in M$ an element $\kappa(x)$ of $\bigwedge_kT_xM^*$). A zero-form is the same as a scalar field.

The natural counter-domain for an $(r,s)$-tensor field over $M$ is the disjoint union:
\begin{multline}\label{eq:tensorbundle}
\Big(\bigotimes_rTM^*\Big)\otimes\Big(\bigotimes_sTM\Big)\\
=\bigcup_{x\in M}\{x\}\times\Big[\Big(\bigotimes_rT_xM^*\Big)\otimes\Big(\bigotimes_sT_xM\Big)\Big].
\end{multline}
The set \eqref{eq:tensorbundle} can, in a natural way, be turned into a differentiable manifold, so that it makes sense to talk about {\em smooth\/}
tensor fields. An atlas for \eqref{eq:tensorbundle} is obtained as follows: given a local chart $\varphi:U\subset M\to\widetilde U\subset\R^n$
on $M$, we define a local chart:
\[\bigcup_{x\in U}\{x\}\times\Big[\Big(\bigotimes_rT_xM^*\Big)\otimes\Big(\bigotimes_sT_xM\Big)\Big]\longrightarrow\widetilde U\times\Big[\Big(\bigotimes_r{\R^n}^*\Big)\otimes
\Big(\bigotimes_s\R^n\Big)\Big]\]
over \eqref{eq:tensorbundle} by:
\[(x,\tau)\longmapsto\big(\varphi(x),(\dd\varphi_x^{-1})^*\tau\big).\]

\medskip

The operations of tensor product and wedge product can be defined (pointwise)
for fields, i.e., if $\tau$ is an $(r,s)$-tensor field over $M$ and $\tau'$ is an $(r',s')$-tensor field over $M$ then
$\tau\otimes\tau'$ is the $(r+r',s+s')$-tensor field over $M$ defined by:
\[(\tau\otimes\tau')_x=\tau_x\otimes\tau'_x,\quad x\in M,\]
and, similarly, if $\kappa$ is a $k$-form over $M$ and $\lambda$ is an $l$-form over $M$ then
$\kappa\wedge\lambda$ is the $(k+l)$-form over $M$ defined by:
\[(\kappa\wedge\lambda)_x=\kappa_x\wedge\lambda_x,\quad x\in M.\]
The tensor product $f\otimes\tau$ (or $\tau\otimes f$) of a scalar field $f$ by a tensor field $\tau$ is just
the ordinary product $f\tau$ (i.e., the map that sends $x\in M$ to $f(x)\tau(x)$) and the wedge product $f\wedge\kappa$
(or $\kappa\wedge f$) of a scalar field $f$ by a differential form $\kappa$ is the same as the ordinary product
$f\kappa$. The tensor product of smooth tensor fields is smooth and the wedge product of smooth differential forms is smooth.

The operation of interior product by vectors can also be defined (pointwise) for fields: if $X$ is a vector field
over $M$ and $\tau$ is a $(k,0)$-tensor field over $M$ then, for $k\ge1$, $i_X\tau$ denotes the $(k-1,0)$-tensor field over $M$
defined by:
\[(i_X\tau)(x)=i_{X(x)}\tau_x,\quad x\in M.\]
We set $i_X\tau=0$ for $k=0$. The interior product $i_X\tau$ of a smooth tensor field $\tau$ by a smooth vector field $X$ is a smooth tensor field.

If $\tau$ is an $(r,s)$-tensor field over $M$, $X_1$, \dots, $X_r$ are vector fields over $M$ and $\alpha_1$, \dots,
$\alpha_s$ are one-forms over $M$ then $\tau(X_1,\ldots,X_r,\alpha_1,\ldots,\alpha_s)$ denotes the scalar field
over $M$ defined by:
\begin{equation}\label{eq:tauXialphaj}
M\ni x\longmapsto\tau_x\big(X_1(x),\ldots,X_r(x),\alpha_1(x),\ldots,\alpha_s(x)\big)\in\R.
\end{equation}
The scalar field \eqref{eq:tauXialphaj} is smooth if $\tau$, $X_1$, \dots, $X_r$, $\alpha_1$, \dots, $\alpha_s$ are smooth.

If $\varphi:N\to M$ is a smooth local diffeomorphism defined over a differentiable manifold $N$ and if $\tau$ is an
$(r,s)$-tensor field over $M$, we define the {\em pull-back\/} $\varphi^*\tau$ to be the $(r,s)$-tensor field over $N$ given by:
\[(\varphi^*\tau)_y=(\dd\varphi_y)^*\tau_{\varphi(y)},\quad y\in N.\]
When the tensor field $\tau$ is purely covariant (i.e., when $s=0$) then the pull-back $\varphi^*\tau$ is defined for {\em any\/} smooth map $\varphi:N\to M$.
In particular, the pull-back $\varphi^*\kappa$ is well-defined for any smooth map $\varphi$ if $\kappa$ is a differential form. If the tensor field $\tau$ is smooth
then the pull-back $\varphi^*\tau$ is also smooth.
Given a smooth local diffeomorphism $\psi:P\to N$ defined over a differentiable manifold $P$ then:
\[(\varphi\circ\psi)^*\tau=\psi^*\varphi^*\tau.\]
The assumption that $\varphi$, $\psi$ be local diffeomorphisms is not necessary if $\tau$ is purely covariant.

A (smooth) {\em local frame\/} over an open subset $U$ of a differentiable manifold $M$ is a sequence $(e_1,\ldots,e_n)$ of (smooth) vector fields
over $U$ such that $\big(e_1(x),\ldots,e_n(x)\big)$ is a basis of $T_xM$, for all $x\in U$. An $(r,s)$-tensor field $\tau$ over $M$ is
represented with respect to such a frame by a family of maps:
\begin{equation}\label{eq:reprtaufunc}
\tau_{i_1\ldots i_r}^{j_1\ldots j_s}:U\longrightarrow\R,\quad i_1,\ldots,i_r,j_1,\ldots,j_s=1,\ldots,n,
\end{equation}
in which, for $x\in U$, the real numbers $\tau_{i_1\ldots i_r}^{j_1\ldots j_s}(x)$ represent $\tau(x)$ with respect to the basis
$\big(e_1(x),\ldots,e_n(x)\big)$ (see \eqref{eq:reprtau}). If $\tau$ is smooth and the local frame $(e_1,\ldots,e_n)$ is smooth
then the maps \eqref{eq:reprtaufunc} are smooth. Conversely, if for some family
of smooth local frames whose domains cover $M$ the corresponding maps \eqref{eq:reprtaufunc} representing $\tau$ are
smooth then $\tau$ is smooth.

\medskip

Given a smooth map $f:M\to\R$ and a vector field $X$ over $M$, we set:
\[X(f)=\dd f(X).\]
Given two smooth vector fields $X$, $Y$ over $M$, then there exists a unique vector field
$Z$ over $M$ such that:
\begin{equation}\label{eq:defLiebracket}
Z(f)=X\big(Y(f)\big)-Y\big(X(f)\big),
\end{equation}
for any smooth map $f:M\to\R$. Such vector field $Z$ is smooth.

\begin{defin}
The only vector field $Z$ satisfying \eqref{eq:defLiebracket} is denoted by $[X,Y]$ and it
is called the {\em Lie bracket\/} of the vector fields $X$, $Y$.
\end{defin}

\begin{defin}\label{thm:phirelated}
Let $\varphi:N\to M$ be a smooth map defined over a differentiable manifold $N$. If $X$ is a vector field over $M$ and $X'$ is a vector field over
$N$ then we say that $X'$ and $X$ are {\em $\varphi$-related\/} (or {\em related by $\varphi$}) if:
\[X\big(\varphi(y)\big)=\dd\varphi_y\big(X'(y)\big),\]
for all $y\in N$.
\end{defin}
If $\varphi$ is a local diffeomorphism then $X$ and $X'$ are $\varphi$-related if and only if $X'$ equals the pull-back $\varphi^*X$. We have the following:

\begin{prop}\label{thm:brackphirel}
If $\varphi:N\to M$ is a smooth map, $X'$, $Y'$ are smooth vector fields over $N$ that are $\varphi$-related, respectively, to smooth vector fields
$X$, $Y$ over $M$ then the Lie bracket $[X',Y']$ is $\varphi$-related to the Lie bracket $[X,Y]$.\qed
\end{prop}

\begin{defin}
Let $\tau$ be a smooth $(r,s)$-tensor field over a differentiable manifold $M$ and let $X$ be a smooth vector field over
$M$. The {\em Lie derivative\/} of $\tau$ with respect to $X$ is the smooth $(r,s)$-tensor field $\mathbb L_X\tau$ over
$M$ defined by:
\begin{equation}\label{eq:defLieder}
\mathbb L_X\tau=\left.\frac{\dd}{\dd t}F_t^*\tau\right\vert_{t=0},
\end{equation}
where $F$ denotes the flow of $X$.
\end{defin}
Since the map $F_t$ is a smooth diffeomorphism between open subsets of $M$, the pull-back $F_t^*\tau$
is always well-defined. The righthand side of \eqref{eq:defLieder}
is to be understood as follows: given any $x\in M$, the value of the righthand side of \eqref{eq:defLieder} at
the point $x$ is the derivative at $t=0$ of the curve:
\[t\longmapsto(F_t^*\tau)(x)\in\Big(\bigotimes_rT_xM^*\Big)\otimes\Big(\bigotimes_sT_xM\Big).\]

Here are the main properties of the Lie derivative. In what follows, $X$ denotes a smooth vector field over the differentiable manifold $M$.
\begin{enumerate}
\item the Lie derivative commutes with restriction to open sets, i.e., if $\tau$ is a smooth tensor field over $M$
and $U$ is an open subset of $M$ then the Lie derivative of $\tau\vert_U$ with respect to $X\vert_U$ is the restriction
of $\mathbb L_X\tau$ to $U$:
\[\mathbb L_{(X\vert_U)}(\tau\vert_U)=(\mathbb L_X\tau)\vert_U.\]
\item If $f:M\to\R$ is a smooth map (regarded as a $(0,0)$-tensor field over $M$) then:
\[\mathbb L_Xf=X(f).\]
\item If $Y$ is a smooth vector field over $M$ (regarded as a $(0,1)$-tensor field over $M$) then:
\[\mathbb L_XY=[X,Y].\]
\item If $\tau$ is a smooth $(r,s)$-tensor field over $M$, $X_1$, \dots, $X_r$ are smooth vector fields over $M$
and $\alpha_1$, \dots, $\alpha_s$ are smooth one-forms over $M$ then:
\begin{multline*}
X\big(\tau(X_1,\ldots,X_r,\alpha_1,\ldots,\alpha_s)\big)=\sum_{i=1}^r\tau(X_1,\ldots,\mathbb L_XX_i,\ldots,X_r,
\alpha_1,\ldots,\alpha_s)\\
+\sum_{i=1}^s\tau(X_1,\ldots,X_r,\alpha_1,\ldots,\mathbb L_X\alpha_i,\ldots,\alpha_s).
\end{multline*}
\item If $\tau$, $\tau'$ are smooth tensor fields over $M$ then:
\[\mathbb L_X(\tau\otimes\tau')=(\mathbb L_X\tau)\otimes\tau'+\tau\otimes(\mathbb L_X\tau').\]
\item If $\kappa$, $\lambda$ are smooth differential forms over $M$ then:
\[\mathbb L_X(\kappa\wedge\lambda)=(\mathbb L_X\kappa)\wedge\lambda+\kappa\wedge(\mathbb L_X\lambda).\]
\end{enumerate}

If $\tau$ is a smooth tensor field over $M$ and $F$ is the flow of a smooth vector field $X$ over $M$ then the derivative of $t\mapsto F_t^*\tau$ at an arbitrary
instant can also be written in terms of the Lie derivative. In fact, this can be done even when $X$ is a time-dependent vector field.
\begin{prop}
Let $\tau$ be a smooth $(r,s)$-tensor field over a differentiable manifold $M$ and let $X$ be a smooth
time-dependent vector field over $M$. Given $t_0\in\R$, if $F^{t_0}$ denotes the flow of $X$ with initial time $t_0$
then:
\begin{equation}\label{eq:Lietimedep}
\frac{\dd}{\dd t}(F^{t_0}_t)^*\tau=(F^{t_0}_t)^*\mathbb L_{X_t}\tau,
\end{equation}
where $X_t=X(t,\cdot)$.
\end{prop}
\begin{proof}
Let $t_1\in\R$ be fixed and let us show that \eqref{eq:Lietimedep} holds at $t=t_1$. Let $G$ denote the flow of
the vector field $X_{t_1}$. Set:
\[\widetilde F_t(x)=G_{t-t_1}\big(F^{t_0}_{t_1}(x)\big),\]
for all $(t,x)\in\R\times M$ for which the righthand side of the equality is well-defined. We have
$\widetilde F_{t_1}=F^{t_0}_{t_1}$ and:
\[\left.\frac{\dd}{\dd t}\widetilde F_t(x)\right\vert_{t=t_1}=\left.\frac{\dd}{\dd t}F^{t_0}_t(x)\right\vert_{t=t_1},\]
for all $x\in\Dom(F^{t_0}_{t_1})$. It follows from the result of Exercise~\ref{exe:tildeFinstead} that:
\[\left.\frac{\dd}{\dd t}(F^{t_0}_t)^*\tau\right\vert_{t=t_1}=
\left.\frac{\dd}{\dd t}\widetilde F_t^*\tau\right\vert_{t=t_1}.\]
Moreover:
\[(\widetilde F_t^*\tau)(x)=\dd F^{t_0}_{t_1}(x)^*\big[(G_{t-t_1}^*\tau)\big(F^{t_0}_{t_1}(x)\big)\big],\]
for all $t\in\R$ and all $x\in M$ in the domain of $\widetilde F_t$.
Taking the derivative at $t=t_1$ on both sides and taking into account that the map $\dd F^{t_0}_{t_1}(x)^*$ is linear,
we obtain:
\begin{multline*}
\left.\frac{\dd}{\dd t}(\widetilde F_t^*\tau)(x)\right\vert_{t=t_1}=
\dd F^{t_0}_{t_1}(x)^*\Big[\left.\frac{\dd}{\dd t}(G_{t-t_1}^*\tau)\big(F^{t_0}_{t_1}(x)\big)\right\vert_{t=t_1}\Big]\\
=\dd F^{t_0}_{t_1}(x)^*\big[(\mathbb L_{X_{t_1}}\tau)\big(F^{t_0}_{t_1}(x)\big)\big],
\end{multline*}
for all $x$ in the domain of $F^{t_0}_{t_1}$. The conclusion follows.
\end{proof}

\begin{defin}
We say that a smooth $(r,s)$-tensor field $\tau$ over a differentiable manifold $M$ is {\em invariant\/} under the
flow $F$ of a smooth vector field $X$ over $M$ if $F_t^*\tau$ is equal to (the restriction to the domain of $F_t$ of)
$\tau$, for all $t\in\R$. If $X$ is a smooth time-dependent vector field over $M$, we say that $\tau$ is invariant
under the flow of $X$ if $(F^{t_0}_t)^*\tau$ is equal to (the restriction to the domain of $F^{t_0}_t$ of) $\tau$,
for all $t_0,t\in\R$.
\end{defin}

\begin{prop}\label{thm:invariantflowLie}
Let $\tau$ be a smooth $(r,s)$-tensor field over a differentiable manifold $M$ and $X$ be a smooth vector field over $M$.
Then $\tau$ is invariant under the flow of $X$ if and only if $\mathbb L_X\tau=0$. If $X$ is a smooth time-dependent
vector field over $M$ then $\tau$ is invariant under the flow of $X$ if and only if $\mathbb L_{X_t}\tau=0$,
for all $t\in\R$, where $X_t=X(t,\cdot)$.
\end{prop}
\begin{proof}
It suffices to consider the time-dependent case. The tensor field $\tau$ is invariant under the flow of $X$ if and only
if:
\begin{equation}\label{eq:auxddtFttau}
\frac{\dd}{\dd t}(F^{t_0}_t)^*\tau=0,
\end{equation}
for all $t_0,t\in\R$. If $\mathbb L_{X_t}\tau=0$ for all $t\in\R$ then \eqref{eq:auxddtFttau} follows from \eqref{eq:Lietimedep}.
If \eqref{eq:auxddtFttau} holds for all $t\in\R$, then using \eqref{eq:Lietimedep} with $t=t_0$, we obtain
that $\mathbb L_{X_{t_0}}\tau=0$ for all $t_0\in\R$.
\end{proof}

{\em Exterior differentiation\/} is an operation that takes a smooth $k$-form $\kappa$ over a differential manifold to a smooth $(k+1)$-form
$\dd\kappa$ over that same manifold. Such operation is characterized by the following set of properties:
\begin{enumerate}
\item\label{itm:extdifopen}
exterior differentiation commutes with restriction to open sets, i.e., if $U$ is an open subset of a differentiable manifold $M$ and $\kappa$ is
a smooth differential form over $M$ then the exterior differential of the restriction of $\kappa$ to $U$ is the restriction of $\dd\kappa$ to $U$:
\[\dd(\kappa\vert_U)=(\dd\kappa)\vert_U.\]
\item Given a differentiable manifold $M$, then the map $\kappa\mapsto\dd\kappa$ that takes smooth $k$-forms over $M$ to smooth $(k+1)$-forms
over $M$ is linear (over the field of real numbers).
\item Exterior differentiation agrees with ordinary differentiation over smooth zero-forms (i.e., over smooth real valued functions).
\item\label{itm:ddzero} The exterior differential of the exterior differential of a smooth differential form $\kappa$ vanishes:
\[\dd(\dd\kappa)=0.\]
\item\label{itm:extdifwedge}
If $\kappa$ is a smooth $k$-form over a differentiable manifold $M$ and $\lambda$ is a smooth $l$-form over $M$ then:
\[\dd(\kappa\wedge\lambda)=(\dd\kappa)\wedge\lambda+(-1)^k\kappa\wedge\dd\lambda.\]
\end{enumerate}

Let us show how properties \eqref{itm:extdifopen}--\eqref{itm:extdifwedge} of exterior differentiation can be used to compute
the exterior differential of a differential form using a local chart.
Let $\varphi:U\subset M\to\widetilde U\subset\R^n$ be a local chart on a differentiable manifold $M$. If one denotes
by $x^i:U\to\R$, $i=1,\ldots,n$, the coordinate functions of $\varphi$ (so that $\varphi=(x^1,\ldots,x^n)$) then it
is customary to denote by $\frac{\partial}{\partial x^i}$, $i=1,\ldots,n$, the local frame over $U$ such that
$\frac{\partial}{\partial x^i}(p)$ is mapped by $\dd\varphi(p)$ to the $i$-th vector of the canonical basis of $\R^n$,
for all $p\in U$, $i=1,\ldots,n$. If $\dd x^i$ denotes the one-form which is the (exterior or ordinary) differential of
the scalar field $x^i$ then $\big(\dd x^1(p),\ldots,\dd x^n(p)\big)$ is the dual basis of
$\big(\frac{\partial}{\partial x^1}(p),\ldots,\frac{\partial}{\partial x^n}(p)\big)$, for all $p\in U$. If $\kappa$
is a $k$-form over $M$ then:
\[\kappa\vert_U=\sum_I\kappa_I\dd x^I,\]
where $I$ runs over the $k$-tuples $(i_1,\ldots,i_k)$ with $1\le i_1<\cdots<i_k\le n$ and:
\[\kappa_I=\kappa\big(\tfrac{\partial}{\partial x^{i_1}},\ldots,\tfrac{\partial}{\partial x^{i_k}}\big),
\quad\dd x^I=\dd x^{i_1}\wedge\cdots\wedge\dd x^{i_k}.\]
Using properties \eqref{itm:extdifopen}--\eqref{itm:extdifwedge} of exterior differentiation it follows that:
\[(\dd\kappa)\vert_U=\sum_I\dd\kappa_I\wedge\dd x^I.\]


Exterior differentiation commutes with pull-backs: if $M$, $N$ are differentiable manifolds,
$\varphi:N\to M$ is a smooth map and $\kappa$ is a smooth differential form over $M$ then:
\[\dd(\varphi^*\kappa)=\varphi^*\dd\kappa.\]
We have the following explicit formula for the exterior differential of a smooth $k$-form $\kappa$ over a differentiable manifold $M$:
\begin{multline}\label{eq:Cartanextdiff}
\dd\kappa(X_0,X_1,\ldots,X_k)=\sum_{i=0}^k(-1)^iX_i\big(\kappa(X_0,\ldots,\widehat{X_i},\ldots,X_k)\big)\\
+\sum_{i<j}(-1)^{i+j}\kappa\big([X_i,X_j],X_0,\ldots,\widehat{X_i},\ldots,\widehat{X_j},\ldots,X_k),
\end{multline}
for any smooth vector fields $X_0$, $X_1$, \dots, $X_k$ over $M$, where the hat indicates that the corresponding term was omitted from the sequence.

A smooth differential form is said to be {\em closed\/} if its exterior differential vanishes; it is said to be
{\em exact\/} if it is equal to the exterior differential of another smooth differential form. By property~\ref{itm:ddzero}
above, every smooth exact form is closed. {\em Poincar\'e Lemma\/} says that every smooth closed form is locally exact,
i.e., if $\kappa$ is a smooth closed form over $M$ then every point of $M$ has an open neighborhood $U$ such that the
restriction $\kappa\vert_U$ is exact (actually, $\kappa\vert_U$ is exact whenever $U$ is diffeomorphic to a star-shaped
open subset of $\R^n$ or, more generally, whenever $U$ is contractible).

There is a very nice formula that expresses the Lie derivative of a differential form in terms of exterior derivatives
and interior products. If $\kappa$ is a smooth differential form over a differentiable manifold $M$ and $X$ is a smooth vector field
over $M$ then:
\begin{equation}\label{eq:diid}
\mathbb L_X\kappa=\dd i_X\kappa+i_X\dd\kappa.
\end{equation}
In Exercise~\ref{exe:gendiid} (which uses Exercises~\ref{exe:timedepdiff} and \ref{exe:genintprod})
we ask the reader to prove a generalization of formula \eqref{eq:diid}.

\end{section}

\begin{section}{A little bit of Lie groups}
\label{sec:Liegroups}

A {\em Lie group\/} is a differentiable manifold $G$, endowed with a group structure, in such a way that both the multiplication map:
\[G\times G\ni(g,h)\longmapsto gh\in G\]
and the inversion map:
\[G\ni g\longmapsto g^{-1}\in G\]
are smooth (in fact, the smoothness of the multiplication map
implies the smoothness of the inversion map, by the implicit function theorem). The neutral element of a group $G$ will be denoted by $1$. The tangent
space $T_1G$ at the neutral element will be denoted by $\mathfrak g$. The space $\mathfrak g$ can be endowed with a binary operation (the Lie bracket) and
with such operation it becomes a Lie algebra and it is called the {\em Lie algebra\/} of the Lie group $G$
(more details are given below). To each $g\in G$, we can associate smooth diffeomorphisms:
\[L_g:G\ni x\longmapsto gx\in G,\quad R_g:G\ni x\longmapsto xg\in G,\]
known respectively as the {\em left translation map\/} and the {\em right translation map}. A vector field $X$ over $G$ is said to be {\em left invariant\/}
(resp., {\em right invariant}) if $X$ is $L_g$-related (resp., $R_g$-related) to $X$, for all $g\in G$ (recall Definition~\ref{thm:phirelated}).
A left invariant (resp., right invariant) vector field is automatically smooth and it is uniquely determined by its value at $1\in G$ by the formula:
\[X(g)=\dd L_g(1)X(1),\quad g\in G,\]
(resp., by the formula $X(g)=\dd R_g(1)X(1)$, $g\in G$). We use the following notation: letters like $X$, $Y$ denote elements of $\mathfrak g=T_1G$;
given $X\in\mathfrak g$, we denote by $X^L$ (resp., by $X^R$) the unique left invariant vector field (resp., the unique right invariant vector field) such
that $X^L(1)=X$ (resp., such that
$X^R(1)=X$). By Proposition~\ref{thm:brackphirel}, the Lie bracket of left invariant (resp., of right invariant) vector fields is left invariant
(resp., right invariant). We define the {\em Lie bracket\/} $[X,Y]\in\mathfrak g$ of elements $X,Y\in\mathfrak g$ by setting:
\[[X,Y]=[X^L,Y^L](1),\]
so that:
\[[X,Y]^L=[X^L,Y^L],\]
for all $X,Y\in\mathfrak g$. We have also:
\begin{equation}\label{eq:brackR}
[X,Y]^R=-[X^R,Y^R],
\end{equation}
for all $X,Y\in\mathfrak g$ (this follows from the observation that $X^L$ and $-X^R$ are related by the inversion map $g\mapsto g^{-1}$).
Endowed with the Lie bracket, the vector space $\mathfrak g$ becomes a {\em Lie algebra}, i.e., a vector space endowed with an anti-symmetric
bilinear binary operation $(X,Y)\mapsto[X,Y]$ satisfying the {\em Jacobi identity}:
\[[X,[Y,Z]]+[Y,[Z,X]]+[Z,[X,Y]]=0,\quad X,Y,Z\in\mathfrak g.\]
Given $X\in\mathfrak g$ then there exists a unique smooth group homomorphism:
\[\gamma_X:\R\longrightarrow G\]
such that $\gamma_X'(0)=X$ (notice that $\gamma_X(0)=1$,
so that $\gamma_X'(0)\in\mathfrak g$). The map $\gamma_X$ is also the maximal integral curve of $X^L$ (and also of $X^R$) passing through $1$ at $t=0$.
The map:
\[\exp:\mathfrak g\longrightarrow G\]
defined by:
\[\exp(X)=\gamma_X(1),\quad X\in\mathfrak g,\]
is smooth and it is called the {\em exponential map\/} of $G$. We have:
\[\gamma_X(t)=\exp(tX),\]
for all $t\in\R$, $X\in\mathfrak g$.

\subsection{Actions of Lie groups on manifolds}\label{sub:Lieactions}

Let $G$ be a Lie group, $M$ be a differentiable manifold and:
\[\rho:G\times M\longrightarrow M\]
be a smooth (left) {\em action\/} of $G$ on $M$, i.e., the map $\rho$ is smooth and the map:
\[G\ni g\longmapsto\rho_g\stackrel{\text{def}}=\rho(g,\cdot)\in\Diff(M)\]
is a homomorphism from $G$ to the group $\Diff(M)$ of all smooth diffeomorphisms of $M$. We write:
\[\rho(g,x)=g\cdot x,\]
for all $g\in G$, $x\in M$. Given $x\in M$, we obtain from the action $\rho$ a smooth map:
\[\beta_x:G\longrightarrow M\]
defined by $\beta_x(g)=g\cdot x$, for all $g\in G$. Given $X\in\mathfrak g$, we define a vector field $X^{\!M}$ over the manifold $M$ by setting:
\begin{equation}\label{eq:defXM}
X^{\!M}(x)=\dd\beta_x(1)X\in T_xM,\quad x\in M.
\end{equation}
The vector field $X^{\!M}$ is smooth. We have:
\begin{equation}\label{eq:XMexp}
X^{\!M}(x)=\left.\frac{\dd}{\dd t}\big(\!\exp(tX)\cdot x\big)\right\vert_{t=0},
\end{equation}
for all $x\in M$. Obviously, in \eqref{eq:XMexp} we can replace $\exp(tX)$ with $\gamma(t)$, where $\gamma$ is any differentiable curve in $G$
with $\gamma(0)=1$ and $\gamma'(0)=X$. It is easily checked that $X^{\!M}$ is the only vector field on $M$ that is $\beta_x$-related to the right invariant
vector field $X^R$, for all $x\in M$. It follows from \eqref{eq:brackR} and from Proposition~\ref{thm:brackphirel} that:
\begin{equation}\label{eq:brackXYM}
[X^{\!M},Y^{\!M}]=-[X,Y]^{\!M},
\end{equation}
for all $X,Y\in\mathfrak g$. In other words, the map $X\mapsto X^{\!M}$ is an {\em anti-homomor\-phism\/} from the Lie algebra $\mathfrak g$ to
the Lie algebra of smooth vector fields over $M$, endowed with the Lie bracket. For any $x\in M$, the curve:
\[\R\ni t\longmapsto\exp(tX)\cdot x\in M\]
is the maximal integral curve of $X^{\!M}$ passing through $x$ at $t=0$; thus, if $F$ denotes the flow of $X^{\!M}$, then:
\[F_t=\rho_{\exp(tX)},\]
for all $t\in\R$.

\begin{rem}
The group $\Diff(M)$ of smooth diffeomorphisms of a differentiable manifold $M$
isn't a (finite-dimensional) Lie group, but it can be endowed with the structure of
an infinite-dimensional Lie group (it is a Fr\'echet Lie group if $M$ is compact and, in general, it is a Lie group
modeled on a topological vector space which is an inductive limit of Fr\'echet spaces). Unfortunately,
given a smooth action $\rho:G\times M\to M$, then the group homomorphism:
\begin{equation}\label{eq:grouphomorho}
G\ni g\longmapsto\rho_g\in\Diff(M)
\end{equation}
is not smooth, unless $M$ is compact. Let us then assume that $M$ is compact. The differential at the neutral
element of a smooth homomorphism between Lie groups is a homomorphism between their Lie algebras.
The Lie algebra homomorphism obtained by differentiating \eqref{eq:grouphomorho} at $1\in G$ is precisely the map
$\mathfrak g\ni X\mapsto X^{\!M}$. But we have seen above that such map is not a Lie algebra homomorphism, but
an {\em anti}-homomorphism; what is going on here? It happens that the Lie algebra of the infinite-dimensional
Lie group $\Diff(M)$ is identified with the space of smooth vector fields over $M$ (vector fields of compact support,
if $M$ is not compact) endowed with the {\em negative\/} of the standard Lie bracket of vector fields!
\end{rem}

\end{section}

\goodbreak

\section*{Exercises}

\subsection*{Quick review of calculus on manifolds}

\begin{exercise}\label{exe:tildeFinstead}
Let $M$, $N$ be differentiable manifolds and
\[F:\Dom(F)\subset\R\times M\to N,\quad\widetilde F:\Dom(\widetilde F)\subset\R\times M\to N\]
be smooth maps defined over open subsets of $\R\times M$. For each $t\in\R$, we define maps:
\begin{gather*}
F_t:\Dom(F_t)=\big\{x\in M:(t,x)\in\Dom(F)\big\}\ni x\longmapsto F(t,x)\in N,\\
\widetilde F_t:\Dom(\widetilde F_t)=\big\{x\in M:(t,x)\in\Dom(\widetilde F)\big\}\ni x\longmapsto\widetilde F(t,x)\in N.
\end{gather*}
Assume that for a certain $t_0\in\R$ the maps $F_{t_0}$ and $\widetilde F_{t_0}$ are equal and that:
\[\left.\frac{\dd}{\dd t}F(t,x)\right\vert_{t=t_0}=\left.\frac{\dd}{\dd t}\widetilde F(t,x)\right\vert_{t=t_0},\]
for all $x\in\Dom(F_{t_0})=\Dom(\widetilde F_{t_0})$. Let $\tau$ be a smooth $(r,s)$-tensor field over $N$;
if $s\ne0$, assume that $F_t$ and $\widetilde F_t$ are local diffeomorphisms, for all $t\in\R$. Show that:
\[\left.\frac{\dd}{\dd t}F_t^*\tau\right\vert_{t=t_0}=\left.\frac{\dd}{\dd t}\widetilde F_t^*\tau\right\vert_{t=t_0}.\]
({\em hint}: there is no loss of generality in assuming that both $M$, $N$ are open subsets of Euclidean space.
Notice that we can write $(F_t^*\tau)(x)$ in the form $\alpha\big(F_t(x),\dd F_t(x)\big)$, for some smooth
map $\alpha$ and that the derivative of:
\[t\longmapsto\alpha\big(F_t(x),\dd F_t(x)\big)\]
at $t=t_0$ depends only on the value and derivative at $t=t_0$ of the map $t\mapsto F_t(x)$).
\end{exercise}

\begin{exercise}\label{exe:ddtFtkappat}
Let $M$, $N$ be differentiable manifolds and:
\[F:\Dom(F)\subset\R\times M\longrightarrow N\]
be a smooth map defined over some open subset $\Dom(F)$ of $\R\times M$. For $t\in\R$, denote by $F_t$ the smooth map
$F(t,\cdot)$ defined over the open set $\Dom(F_t)=\big\{x\in M:(t,x)\in\Dom(F)\big\}$. Let $\kappa$ be a smooth {\em time-dependent $k$-form\/} over $M$, i.e.,
$\kappa$ is a smooth map defined over some open subset $\Dom(\kappa)$ of $\R\times M$, associating an element $\kappa(t,x)$ of $\bigwedge_kT_xM^*$ to each
$(t,x)$ in $\Dom(\kappa)$. For each $t\in\R$, we have a smooth $k$-form $\kappa_t=\kappa(t,\cdot)$ over the open subset
$\Dom(\kappa_t)=\big\{x\in M:(t,x)\in\Dom(\kappa)\big\}$ of $M$. Given $t_0\in\R$, show that the following equality holds over the open set
$F_{t_0}^{-1}\big(\!\Dom(\kappa_{t_0})\big)$:
\[\left.\frac{\dd}{\dd t}(F_t^*\kappa_t)\right\vert_{t=t_0}=\left.\frac{\dd}{\dd t}(F_t^*\kappa_{t_0})\right\vert_{t=t_0}
+F_{t_0}^*\Big(\left.\frac{\dd}{\dd t}\kappa_t\right\vert_{t=t_0}\Big).\]
({\em hint}: for a fixed $x$, define $\phi(t,s)=(F_t^*\kappa_s)(x)$, and then write the derivative of $t\mapsto\phi(t,t)$
in terms of the partial derivatives of $\phi$).
\end{exercise}

\begin{exercise}\label{exe:timedepdiff}
Let $M$ be a differentiable manifold and $\kappa$ be a smooth time-dependent $k$-form over $M$. For $t\in\R$, set $\kappa_t=\kappa(t,\cdot)$
and denote by $\frac{\dd}{\dd t}\kappa_t$ the smooth $k$-form over $\Dom(\kappa_t)$ defined by:
\[\Big(\frac{\dd}{\dd t}\kappa_t\Big)(x)=\frac{\dd}{\dd t}\big(\kappa_t(x)\big),\quad x\in\Dom(\kappa_t).\]
Show that the operation $\frac{\dd}{\dd t}$ commutes with the exterior derivative, i.e.:
\[\dd\Big(\frac{\dd}{\dd t}\kappa_t\Big)=\frac{\dd}{\dd t}\,\dd\kappa_t,\]
for all $t\in\R$ ({\em hint}: write down the representation of $\kappa$ with respect to a local chart and observe
that the exterior derivative and the operation $\frac{\dd}{\dd t}$ involve partial derivatives with respect to {\em distinct\/}
variables).
\end{exercise}

\begin{exercise}\label{exe:genintprod}
Let $M$, $N$ be differentiable manifolds and:
\[F:\Dom(F)\subset\R\times M\longrightarrow N\]
be a smooth map defined over some open subset $\Dom(F)$ of $\R\times M$. For $t\in\R$, denote by $F_t$ the smooth map
$F(t,\cdot)$ defined over the open set $\Dom(F_t)=\big\{x\in M:(t,x)\in\Dom(F)\big\}$.
Given a $k$-form $\kappa$ over $N$ then the
{\em generalized interior product\/} $i(F,\kappa)$ is the time-dependent $(k-1)$-form over $M$:
\[i(F,\kappa):\Dom(F)\ni(t,x)\longmapsto i(F,\kappa)(t,x)\in\bigwedge_{k-1}T_xM^*\]
defined by:
\[i(F,\kappa)(t,x)=\dd F_t(x)^*\big[i_{v(t,x)}\kappa\big(F_t(x)\big)\big],\]
for all $(t,x)\in\Dom(F)$, where $v(t,x)=\frac{\dd}{\dd t}F_t(x)\in T_{F_t(x)}N$. For $t\in\R$, denote by
$i(F,\kappa;t)$ the $k$-form over $\Dom(F_t)$ defined by $i(F,\kappa;t)(x)=i(F,\kappa)(t,x)$.
Show that if $\kappa$ is a $k$-form over $N$ and $\lambda$ is an $l$-form over $N$ then:
\begin{equation}\label{eq:geniderivation}
i(F,\kappa\wedge\lambda;t)=i(F,\kappa;t)\wedge(F_t^*\lambda)+(-1)^k(F_t^*\kappa)\wedge i(F,\lambda;t),
\end{equation}
for all $(t,x)\in\Dom(F)$ ({\em hint}: use \eqref{eq:propiv}).
\end{exercise}

\begin{exercise}\label{exe:gendiid}
Let $M$, $N$ and $F$ be as in the statement of Exercise~\ref{exe:genintprod}. The goal of this exercise is to show
that for a smooth $k$-form $\kappa$ over $N$ the following formula holds:
\begin{equation}\label{eq:gendiid}
\frac{\dd}{\dd t}F_t^*\kappa=\dd\big(i(F,\kappa;t)\big)+i(F,\dd\kappa;t).
\end{equation}
Notice that if $N=M$ and $F$ is the flow of a smooth vector field $X$ over $M$ then formula \eqref{eq:gendiid} (with $t=0$) reduces to
\eqref{eq:diid}.
Let $t\in\R$ be fixed and for a smooth $k$-form $\kappa$ over $N$ consider the smooth $k$-forms over
$\Dom(F_t)\subset M$ defined by:
\[D_1(\kappa)=\frac{\dd}{\dd t}F_t^*\kappa,\quad
D_2(\kappa)=\dd\big(i(F,\kappa;t)\big)+i(F,\dd\kappa;t).\]
We have to show that $D_1(\kappa)=D_2(\kappa)$, for any smooth differential form $\kappa$ over $N$.
\begin{itemize}
\item[(a)] Check that the map $\kappa\mapsto D_i(\kappa)$ is linear (over the field of real numbers), for $i=1,2$.
\item[(b)] Check that, if $\kappa$ and $\lambda$ are smooth differential forms over $N$ then:
\[D_i(\kappa\wedge\lambda)=D_i(\kappa)\wedge(F_t^*\lambda)+(F_t^*\kappa)\wedge D_i(\lambda),\]
for $i=1,2$ ({\em hint}: for $i=2$ use formula \eqref{eq:geniderivation}).
\item[(c)] Check that both $D_1$ and $D_2$ commute with the exterior derivative, i.e.,
that $D_i(\dd\kappa)=\dd\big(D_i(\kappa)\big)$, $i=1,2$ ({\em hint}: for $i=1$ use
the result of Exercise~\ref{exe:timedepdiff}).
\item[(d)] Use the results of the items above to conclude that the set of smooth differential forms $\kappa$ over $N$
for which $D_1(\kappa)=D_2(\kappa)$ is a real vector subspace closed under exterior products and exterior derivatives.
\item[(e)] Check that $D_1(\kappa)=D_2(\kappa)$ when $\kappa$ is a smooth zero-form (i.e., a smooth real valued function)
over $N$.
\item[(f)] When $N$ admits a global chart then any smooth differential form over $N$ can be obtained from smooth
zero-forms using exterior derivatives, exterior products and sums. Explain why it suffices to prove \eqref{eq:gendiid}
in the case when $N$ admits a global chart. Conclude the proof of \eqref{eq:gendiid}.
\end{itemize}
\end{exercise}

\end{chapter}

\end{document}
